
% Default to the notebook output style

    


% Inherit from the specified cell style.




    
\documentclass[11pt]{article}

    
    
    \usepackage[T1]{fontenc}
    % Nicer default font (+ math font) than Computer Modern for most use cases
    \usepackage{mathpazo}

    % Basic figure setup, for now with no caption control since it's done
    % automatically by Pandoc (which extracts ![](path) syntax from Markdown).
    \usepackage{graphicx}
    % We will generate all images so they have a width \maxwidth. This means
    % that they will get their normal width if they fit onto the page, but
    % are scaled down if they would overflow the margins.
    \makeatletter
    \def\maxwidth{\ifdim\Gin@nat@width>\linewidth\linewidth
    \else\Gin@nat@width\fi}
    \makeatother
    \let\Oldincludegraphics\includegraphics
    % Set max figure width to be 80% of text width, for now hardcoded.
    \renewcommand{\includegraphics}[1]{\Oldincludegraphics[width=.8\maxwidth]{#1}}
    % Ensure that by default, figures have no caption (until we provide a
    % proper Figure object with a Caption API and a way to capture that
    % in the conversion process - todo).
    \usepackage{caption}
    \DeclareCaptionLabelFormat{nolabel}{}
    \captionsetup{labelformat=nolabel}

    \usepackage{adjustbox} % Used to constrain images to a maximum size 
    \usepackage{xcolor} % Allow colors to be defined
    \usepackage{enumerate} % Needed for markdown enumerations to work
    \usepackage{geometry} % Used to adjust the document margins
    \usepackage{amsmath} % Equations
    \usepackage{amssymb} % Equations
    \usepackage{textcomp} % defines textquotesingle
    % Hack from http://tex.stackexchange.com/a/47451/13684:
    \AtBeginDocument{%
        \def\PYZsq{\textquotesingle}% Upright quotes in Pygmentized code
    }
    \usepackage{upquote} % Upright quotes for verbatim code
    \usepackage{eurosym} % defines \euro
    \usepackage[mathletters]{ucs} % Extended unicode (utf-8) support
    \usepackage[utf8x]{inputenc} % Allow utf-8 characters in the tex document
    \usepackage{fancyvrb} % verbatim replacement that allows latex
    \usepackage{grffile} % extends the file name processing of package graphics 
                         % to support a larger range 
    % The hyperref package gives us a pdf with properly built
    % internal navigation ('pdf bookmarks' for the table of contents,
    % internal cross-reference links, web links for URLs, etc.)
    \usepackage{hyperref}
    \usepackage{longtable} % longtable support required by pandoc >1.10
    \usepackage{booktabs}  % table support for pandoc > 1.12.2
    \usepackage[inline]{enumitem} % IRkernel/repr support (it uses the enumerate* environment)
    \usepackage[normalem]{ulem} % ulem is needed to support strikethroughs (\sout)
                                % normalem makes italics be italics, not underlines
    \usepackage{mathrsfs}
    

    
    
    % Colors for the hyperref package
    \definecolor{urlcolor}{rgb}{0,.145,.698}
    \definecolor{linkcolor}{rgb}{.71,0.21,0.01}
    \definecolor{citecolor}{rgb}{.12,.54,.11}

    % ANSI colors
    \definecolor{ansi-black}{HTML}{3E424D}
    \definecolor{ansi-black-intense}{HTML}{282C36}
    \definecolor{ansi-red}{HTML}{E75C58}
    \definecolor{ansi-red-intense}{HTML}{B22B31}
    \definecolor{ansi-green}{HTML}{00A250}
    \definecolor{ansi-green-intense}{HTML}{007427}
    \definecolor{ansi-yellow}{HTML}{DDB62B}
    \definecolor{ansi-yellow-intense}{HTML}{B27D12}
    \definecolor{ansi-blue}{HTML}{208FFB}
    \definecolor{ansi-blue-intense}{HTML}{0065CA}
    \definecolor{ansi-magenta}{HTML}{D160C4}
    \definecolor{ansi-magenta-intense}{HTML}{A03196}
    \definecolor{ansi-cyan}{HTML}{60C6C8}
    \definecolor{ansi-cyan-intense}{HTML}{258F8F}
    \definecolor{ansi-white}{HTML}{C5C1B4}
    \definecolor{ansi-white-intense}{HTML}{A1A6B2}
    \definecolor{ansi-default-inverse-fg}{HTML}{FFFFFF}
    \definecolor{ansi-default-inverse-bg}{HTML}{000000}

    % commands and environments needed by pandoc snippets
    % extracted from the output of `pandoc -s`
    \providecommand{\tightlist}{%
      \setlength{\itemsep}{0pt}\setlength{\parskip}{0pt}}
    \DefineVerbatimEnvironment{Highlighting}{Verbatim}{commandchars=\\\{\}}
    % Add ',fontsize=\small' for more characters per line
    \newenvironment{Shaded}{}{}
    \newcommand{\KeywordTok}[1]{\textcolor[rgb]{0.00,0.44,0.13}{\textbf{{#1}}}}
    \newcommand{\DataTypeTok}[1]{\textcolor[rgb]{0.56,0.13,0.00}{{#1}}}
    \newcommand{\DecValTok}[1]{\textcolor[rgb]{0.25,0.63,0.44}{{#1}}}
    \newcommand{\BaseNTok}[1]{\textcolor[rgb]{0.25,0.63,0.44}{{#1}}}
    \newcommand{\FloatTok}[1]{\textcolor[rgb]{0.25,0.63,0.44}{{#1}}}
    \newcommand{\CharTok}[1]{\textcolor[rgb]{0.25,0.44,0.63}{{#1}}}
    \newcommand{\StringTok}[1]{\textcolor[rgb]{0.25,0.44,0.63}{{#1}}}
    \newcommand{\CommentTok}[1]{\textcolor[rgb]{0.38,0.63,0.69}{\textit{{#1}}}}
    \newcommand{\OtherTok}[1]{\textcolor[rgb]{0.00,0.44,0.13}{{#1}}}
    \newcommand{\AlertTok}[1]{\textcolor[rgb]{1.00,0.00,0.00}{\textbf{{#1}}}}
    \newcommand{\FunctionTok}[1]{\textcolor[rgb]{0.02,0.16,0.49}{{#1}}}
    \newcommand{\RegionMarkerTok}[1]{{#1}}
    \newcommand{\ErrorTok}[1]{\textcolor[rgb]{1.00,0.00,0.00}{\textbf{{#1}}}}
    \newcommand{\NormalTok}[1]{{#1}}
    
    % Additional commands for more recent versions of Pandoc
    \newcommand{\ConstantTok}[1]{\textcolor[rgb]{0.53,0.00,0.00}{{#1}}}
    \newcommand{\SpecialCharTok}[1]{\textcolor[rgb]{0.25,0.44,0.63}{{#1}}}
    \newcommand{\VerbatimStringTok}[1]{\textcolor[rgb]{0.25,0.44,0.63}{{#1}}}
    \newcommand{\SpecialStringTok}[1]{\textcolor[rgb]{0.73,0.40,0.53}{{#1}}}
    \newcommand{\ImportTok}[1]{{#1}}
    \newcommand{\DocumentationTok}[1]{\textcolor[rgb]{0.73,0.13,0.13}{\textit{{#1}}}}
    \newcommand{\AnnotationTok}[1]{\textcolor[rgb]{0.38,0.63,0.69}{\textbf{\textit{{#1}}}}}
    \newcommand{\CommentVarTok}[1]{\textcolor[rgb]{0.38,0.63,0.69}{\textbf{\textit{{#1}}}}}
    \newcommand{\VariableTok}[1]{\textcolor[rgb]{0.10,0.09,0.49}{{#1}}}
    \newcommand{\ControlFlowTok}[1]{\textcolor[rgb]{0.00,0.44,0.13}{\textbf{{#1}}}}
    \newcommand{\OperatorTok}[1]{\textcolor[rgb]{0.40,0.40,0.40}{{#1}}}
    \newcommand{\BuiltInTok}[1]{{#1}}
    \newcommand{\ExtensionTok}[1]{{#1}}
    \newcommand{\PreprocessorTok}[1]{\textcolor[rgb]{0.74,0.48,0.00}{{#1}}}
    \newcommand{\AttributeTok}[1]{\textcolor[rgb]{0.49,0.56,0.16}{{#1}}}
    \newcommand{\InformationTok}[1]{\textcolor[rgb]{0.38,0.63,0.69}{\textbf{\textit{{#1}}}}}
    \newcommand{\WarningTok}[1]{\textcolor[rgb]{0.38,0.63,0.69}{\textbf{\textit{{#1}}}}}
    
    
    % Define a nice break command that doesn't care if a line doesn't already
    % exist.
    \def\br{\hspace*{\fill} \\* }
    % Math Jax compatibility definitions
    \def\gt{>}
    \def\lt{<}
    \let\Oldtex\TeX
    \let\Oldlatex\LaTeX
    \renewcommand{\TeX}{\textrm{\Oldtex}}
    \renewcommand{\LaTeX}{\textrm{\Oldlatex}}
    % Document parameters
    % Document title
    \title{Lab3}
    
    
    
    
    

    % Pygments definitions
    
\makeatletter
\def\PY@reset{\let\PY@it=\relax \let\PY@bf=\relax%
    \let\PY@ul=\relax \let\PY@tc=\relax%
    \let\PY@bc=\relax \let\PY@ff=\relax}
\def\PY@tok#1{\csname PY@tok@#1\endcsname}
\def\PY@toks#1+{\ifx\relax#1\empty\else%
    \PY@tok{#1}\expandafter\PY@toks\fi}
\def\PY@do#1{\PY@bc{\PY@tc{\PY@ul{%
    \PY@it{\PY@bf{\PY@ff{#1}}}}}}}
\def\PY#1#2{\PY@reset\PY@toks#1+\relax+\PY@do{#2}}

\expandafter\def\csname PY@tok@w\endcsname{\def\PY@tc##1{\textcolor[rgb]{0.73,0.73,0.73}{##1}}}
\expandafter\def\csname PY@tok@c\endcsname{\let\PY@it=\textit\def\PY@tc##1{\textcolor[rgb]{0.25,0.50,0.50}{##1}}}
\expandafter\def\csname PY@tok@cp\endcsname{\def\PY@tc##1{\textcolor[rgb]{0.74,0.48,0.00}{##1}}}
\expandafter\def\csname PY@tok@k\endcsname{\let\PY@bf=\textbf\def\PY@tc##1{\textcolor[rgb]{0.00,0.50,0.00}{##1}}}
\expandafter\def\csname PY@tok@kp\endcsname{\def\PY@tc##1{\textcolor[rgb]{0.00,0.50,0.00}{##1}}}
\expandafter\def\csname PY@tok@kt\endcsname{\def\PY@tc##1{\textcolor[rgb]{0.69,0.00,0.25}{##1}}}
\expandafter\def\csname PY@tok@o\endcsname{\def\PY@tc##1{\textcolor[rgb]{0.40,0.40,0.40}{##1}}}
\expandafter\def\csname PY@tok@ow\endcsname{\let\PY@bf=\textbf\def\PY@tc##1{\textcolor[rgb]{0.67,0.13,1.00}{##1}}}
\expandafter\def\csname PY@tok@nb\endcsname{\def\PY@tc##1{\textcolor[rgb]{0.00,0.50,0.00}{##1}}}
\expandafter\def\csname PY@tok@nf\endcsname{\def\PY@tc##1{\textcolor[rgb]{0.00,0.00,1.00}{##1}}}
\expandafter\def\csname PY@tok@nc\endcsname{\let\PY@bf=\textbf\def\PY@tc##1{\textcolor[rgb]{0.00,0.00,1.00}{##1}}}
\expandafter\def\csname PY@tok@nn\endcsname{\let\PY@bf=\textbf\def\PY@tc##1{\textcolor[rgb]{0.00,0.00,1.00}{##1}}}
\expandafter\def\csname PY@tok@ne\endcsname{\let\PY@bf=\textbf\def\PY@tc##1{\textcolor[rgb]{0.82,0.25,0.23}{##1}}}
\expandafter\def\csname PY@tok@nv\endcsname{\def\PY@tc##1{\textcolor[rgb]{0.10,0.09,0.49}{##1}}}
\expandafter\def\csname PY@tok@no\endcsname{\def\PY@tc##1{\textcolor[rgb]{0.53,0.00,0.00}{##1}}}
\expandafter\def\csname PY@tok@nl\endcsname{\def\PY@tc##1{\textcolor[rgb]{0.63,0.63,0.00}{##1}}}
\expandafter\def\csname PY@tok@ni\endcsname{\let\PY@bf=\textbf\def\PY@tc##1{\textcolor[rgb]{0.60,0.60,0.60}{##1}}}
\expandafter\def\csname PY@tok@na\endcsname{\def\PY@tc##1{\textcolor[rgb]{0.49,0.56,0.16}{##1}}}
\expandafter\def\csname PY@tok@nt\endcsname{\let\PY@bf=\textbf\def\PY@tc##1{\textcolor[rgb]{0.00,0.50,0.00}{##1}}}
\expandafter\def\csname PY@tok@nd\endcsname{\def\PY@tc##1{\textcolor[rgb]{0.67,0.13,1.00}{##1}}}
\expandafter\def\csname PY@tok@s\endcsname{\def\PY@tc##1{\textcolor[rgb]{0.73,0.13,0.13}{##1}}}
\expandafter\def\csname PY@tok@sd\endcsname{\let\PY@it=\textit\def\PY@tc##1{\textcolor[rgb]{0.73,0.13,0.13}{##1}}}
\expandafter\def\csname PY@tok@si\endcsname{\let\PY@bf=\textbf\def\PY@tc##1{\textcolor[rgb]{0.73,0.40,0.53}{##1}}}
\expandafter\def\csname PY@tok@se\endcsname{\let\PY@bf=\textbf\def\PY@tc##1{\textcolor[rgb]{0.73,0.40,0.13}{##1}}}
\expandafter\def\csname PY@tok@sr\endcsname{\def\PY@tc##1{\textcolor[rgb]{0.73,0.40,0.53}{##1}}}
\expandafter\def\csname PY@tok@ss\endcsname{\def\PY@tc##1{\textcolor[rgb]{0.10,0.09,0.49}{##1}}}
\expandafter\def\csname PY@tok@sx\endcsname{\def\PY@tc##1{\textcolor[rgb]{0.00,0.50,0.00}{##1}}}
\expandafter\def\csname PY@tok@m\endcsname{\def\PY@tc##1{\textcolor[rgb]{0.40,0.40,0.40}{##1}}}
\expandafter\def\csname PY@tok@gh\endcsname{\let\PY@bf=\textbf\def\PY@tc##1{\textcolor[rgb]{0.00,0.00,0.50}{##1}}}
\expandafter\def\csname PY@tok@gu\endcsname{\let\PY@bf=\textbf\def\PY@tc##1{\textcolor[rgb]{0.50,0.00,0.50}{##1}}}
\expandafter\def\csname PY@tok@gd\endcsname{\def\PY@tc##1{\textcolor[rgb]{0.63,0.00,0.00}{##1}}}
\expandafter\def\csname PY@tok@gi\endcsname{\def\PY@tc##1{\textcolor[rgb]{0.00,0.63,0.00}{##1}}}
\expandafter\def\csname PY@tok@gr\endcsname{\def\PY@tc##1{\textcolor[rgb]{1.00,0.00,0.00}{##1}}}
\expandafter\def\csname PY@tok@ge\endcsname{\let\PY@it=\textit}
\expandafter\def\csname PY@tok@gs\endcsname{\let\PY@bf=\textbf}
\expandafter\def\csname PY@tok@gp\endcsname{\let\PY@bf=\textbf\def\PY@tc##1{\textcolor[rgb]{0.00,0.00,0.50}{##1}}}
\expandafter\def\csname PY@tok@go\endcsname{\def\PY@tc##1{\textcolor[rgb]{0.53,0.53,0.53}{##1}}}
\expandafter\def\csname PY@tok@gt\endcsname{\def\PY@tc##1{\textcolor[rgb]{0.00,0.27,0.87}{##1}}}
\expandafter\def\csname PY@tok@err\endcsname{\def\PY@bc##1{\setlength{\fboxsep}{0pt}\fcolorbox[rgb]{1.00,0.00,0.00}{1,1,1}{\strut ##1}}}
\expandafter\def\csname PY@tok@kc\endcsname{\let\PY@bf=\textbf\def\PY@tc##1{\textcolor[rgb]{0.00,0.50,0.00}{##1}}}
\expandafter\def\csname PY@tok@kd\endcsname{\let\PY@bf=\textbf\def\PY@tc##1{\textcolor[rgb]{0.00,0.50,0.00}{##1}}}
\expandafter\def\csname PY@tok@kn\endcsname{\let\PY@bf=\textbf\def\PY@tc##1{\textcolor[rgb]{0.00,0.50,0.00}{##1}}}
\expandafter\def\csname PY@tok@kr\endcsname{\let\PY@bf=\textbf\def\PY@tc##1{\textcolor[rgb]{0.00,0.50,0.00}{##1}}}
\expandafter\def\csname PY@tok@bp\endcsname{\def\PY@tc##1{\textcolor[rgb]{0.00,0.50,0.00}{##1}}}
\expandafter\def\csname PY@tok@fm\endcsname{\def\PY@tc##1{\textcolor[rgb]{0.00,0.00,1.00}{##1}}}
\expandafter\def\csname PY@tok@vc\endcsname{\def\PY@tc##1{\textcolor[rgb]{0.10,0.09,0.49}{##1}}}
\expandafter\def\csname PY@tok@vg\endcsname{\def\PY@tc##1{\textcolor[rgb]{0.10,0.09,0.49}{##1}}}
\expandafter\def\csname PY@tok@vi\endcsname{\def\PY@tc##1{\textcolor[rgb]{0.10,0.09,0.49}{##1}}}
\expandafter\def\csname PY@tok@vm\endcsname{\def\PY@tc##1{\textcolor[rgb]{0.10,0.09,0.49}{##1}}}
\expandafter\def\csname PY@tok@sa\endcsname{\def\PY@tc##1{\textcolor[rgb]{0.73,0.13,0.13}{##1}}}
\expandafter\def\csname PY@tok@sb\endcsname{\def\PY@tc##1{\textcolor[rgb]{0.73,0.13,0.13}{##1}}}
\expandafter\def\csname PY@tok@sc\endcsname{\def\PY@tc##1{\textcolor[rgb]{0.73,0.13,0.13}{##1}}}
\expandafter\def\csname PY@tok@dl\endcsname{\def\PY@tc##1{\textcolor[rgb]{0.73,0.13,0.13}{##1}}}
\expandafter\def\csname PY@tok@s2\endcsname{\def\PY@tc##1{\textcolor[rgb]{0.73,0.13,0.13}{##1}}}
\expandafter\def\csname PY@tok@sh\endcsname{\def\PY@tc##1{\textcolor[rgb]{0.73,0.13,0.13}{##1}}}
\expandafter\def\csname PY@tok@s1\endcsname{\def\PY@tc##1{\textcolor[rgb]{0.73,0.13,0.13}{##1}}}
\expandafter\def\csname PY@tok@mb\endcsname{\def\PY@tc##1{\textcolor[rgb]{0.40,0.40,0.40}{##1}}}
\expandafter\def\csname PY@tok@mf\endcsname{\def\PY@tc##1{\textcolor[rgb]{0.40,0.40,0.40}{##1}}}
\expandafter\def\csname PY@tok@mh\endcsname{\def\PY@tc##1{\textcolor[rgb]{0.40,0.40,0.40}{##1}}}
\expandafter\def\csname PY@tok@mi\endcsname{\def\PY@tc##1{\textcolor[rgb]{0.40,0.40,0.40}{##1}}}
\expandafter\def\csname PY@tok@il\endcsname{\def\PY@tc##1{\textcolor[rgb]{0.40,0.40,0.40}{##1}}}
\expandafter\def\csname PY@tok@mo\endcsname{\def\PY@tc##1{\textcolor[rgb]{0.40,0.40,0.40}{##1}}}
\expandafter\def\csname PY@tok@ch\endcsname{\let\PY@it=\textit\def\PY@tc##1{\textcolor[rgb]{0.25,0.50,0.50}{##1}}}
\expandafter\def\csname PY@tok@cm\endcsname{\let\PY@it=\textit\def\PY@tc##1{\textcolor[rgb]{0.25,0.50,0.50}{##1}}}
\expandafter\def\csname PY@tok@cpf\endcsname{\let\PY@it=\textit\def\PY@tc##1{\textcolor[rgb]{0.25,0.50,0.50}{##1}}}
\expandafter\def\csname PY@tok@c1\endcsname{\let\PY@it=\textit\def\PY@tc##1{\textcolor[rgb]{0.25,0.50,0.50}{##1}}}
\expandafter\def\csname PY@tok@cs\endcsname{\let\PY@it=\textit\def\PY@tc##1{\textcolor[rgb]{0.25,0.50,0.50}{##1}}}

\def\PYZbs{\char`\\}
\def\PYZus{\char`\_}
\def\PYZob{\char`\{}
\def\PYZcb{\char`\}}
\def\PYZca{\char`\^}
\def\PYZam{\char`\&}
\def\PYZlt{\char`\<}
\def\PYZgt{\char`\>}
\def\PYZsh{\char`\#}
\def\PYZpc{\char`\%}
\def\PYZdl{\char`\$}
\def\PYZhy{\char`\-}
\def\PYZsq{\char`\'}
\def\PYZdq{\char`\"}
\def\PYZti{\char`\~}
% for compatibility with earlier versions
\def\PYZat{@}
\def\PYZlb{[}
\def\PYZrb{]}
\makeatother


    % Exact colors from NB
    \definecolor{incolor}{rgb}{0.0, 0.0, 0.5}
    \definecolor{outcolor}{rgb}{0.545, 0.0, 0.0}



    
    % Prevent overflowing lines due to hard-to-break entities
    \sloppy 
    % Setup hyperref package
    \hypersetup{
      breaklinks=true,  % so long urls are correctly broken across lines
      colorlinks=true,
      urlcolor=urlcolor,
      linkcolor=linkcolor,
      citecolor=citecolor,
      }
    % Slightly bigger margins than the latex defaults
    
    \geometry{verbose,tmargin=1in,bmargin=1in,lmargin=1in,rmargin=1in}
    
    

    \begin{document}
    
    
    \maketitle
    
    

    
    \begin{verbatim}
Instructor(s): Dr. Kenneth Duru
First Semester 2019
Mathematical Sciences Institute
Australian National University
\end{verbatim}

\section{Math3511/6111, Scientific
Computing}\label{math35116111-scientific-computing}

This Lab book must be submitted by \textbf{6th May 5pm}. Late
Submissions will incur a 5\% penalty per working day. Assignment
submissions will close on the \textbf{13th May 5pm}. Submissions after
this time will be invalid.

\section{Lab 3: Interpolation}\label{lab-3-interpolation}

    \subsection{A. Background}\label{a.-background}

\subsubsection{A1. Summary}\label{a1.-summary}

In this lab you will explore the efficient evaluation of the polynomial
interpolant using the barycentric formula. You will learn about the
barycentric formula and Scipy routine for barycentric interpolation.
Background required are the lectures in the file Interpolation on Wattle
up to the Lagrange formula.

The barycentric formula will be derived in the tutorial, alternatively,
you can read up on it yourself using a search of the term barycentric
formula.

Your tasks are to implement the barycentric formula in Python. You will
measure the performance using the Python timeit tool. Finally, you will
provide code which utilises the Scipy barycentric interpolation codes.
Compare the performance of your codes and the Scipy codes. For extra
points provide some discussion on the difference in performance of the
Scipy and your code.

\begin{itemize}
\item
  starting point: Lagrangian interpolation formula for the data
  \(x_i, y_i\) with \(i=0,\ldots,n\) \[p(x) = \sum_{j=0}^n y_j l_j(x)\]
\item
  barycentric interpolation formula
  \[p(x) = \frac{\sum_{j=0}^n y_j \frac{c_j}{x-x_j}}{\sum_{j=0}^n \frac{c_j}{x-x_j}}\]
  which holds for \(x\neq x_j\) for \(j=0,\ldots,n\)
\end{itemize}

For those who would like to get faster code we have included some
\textbf{non-mandatory} material and coding ideas using Cython. Feel free
to talk to the tutors to get further information about Cython. However,
the Cython parts will not be assessed. Note that Cython might not work
on the lab computers. Typically, you will install it yourself with
Anaconda. However, you will require to use a C compiler which may cause
additional problems on Macs and Windows computers.

    \subsubsection{A2. Theory (by tutors)}\label{a2.-theory-by-tutors}

\begin{itemize}
\item
  Lagrangian functions given in lecture satisfy
  \[l_j(x) = w(x) \cdot \frac{c_j}{x-x_j}\] for \(x\neq x_j\) and
  \[w(x) = \prod_{k=0}^n (x-x_k)\] and
  \(c_j = \left(\prod_{k=0}^n (x_j-x_k)\right)^{-1}\)
\item
  Lagrangian functions satisfy for all \(x\) \[\sum_{j=0}^n l_j(x) = 1\]
\item
  Using these two properties of \(l_j\) one then gets for the
  interpolant for \(x\neq x_j\)
  \[p(x) = \frac{\sum_{j=0}^n y_j w(x) \frac{c_j}{x-x_j}}{\sum_{j=0}^n w(x) \frac{c_j}{x-x_j}}\]
  from which the barycentric formula follows directly
\end{itemize}

    \subsubsection{A3. Non-mandatory part on Cython (by tutors, on demand
only)}\label{a3.-non-mandatory-part-on-cython-by-tutors-on-demand-only}

\begin{itemize}
\tightlist
\item
  difference between Python and C:

  \begin{itemize}
  \tightlist
  \item
    Python is interpreted which C is compiled
  \item
    Python uses dynamical typing while C uses static typing
  \end{itemize}
\item
  Cython is basically Python code

  \begin{itemize}
  \tightlist
  \item
    which is converted to C and then compiled
  \item
    which includes a possibility of static typing
  \end{itemize}
\item
  Usage of Cython is easy with the \%\%cython magic of jupyter notebooks

  \begin{itemize}
  \tightlist
  \item
    load cython at the beginning with
  \end{itemize}

\begin{Shaded}
\begin{Highlighting}[]
\OperatorTok{%}\NormalTok{load_ext cython}
\end{Highlighting}
\end{Shaded}

  \begin{itemize}
  \tightlist
  \item
    define a Cython cell with
  \end{itemize}

\begin{Shaded}
\begin{Highlighting}[]
\OperatorTok\NormalTok{cython}
\end{Highlighting}
\end{Shaded}

  (\texttt{\%\%cython\ -a} for annotation)

  \begin{itemize}
  \tightlist
  \item
    use ordinary Python code in the Cython cell
  \item
    add static typing with commmands like
  \end{itemize}

\begin{Shaded}
\begin{Highlighting}[]
\NormalTok{cdef }\BuiltInTok{int}\NormalTok{ i,n}
\NormalTok{cdef numpy.ndarray[numpy.float64_t, ndim}\OperatorTok{=}\DecValTok{1}\NormalTok{] x}
\end{Highlighting}
\end{Shaded}

  \begin{itemize}
  \item
    import numpy data for Cython and Python by

\begin{Shaded}
\begin{Highlighting}[]
\ImportTok{import}\NormalTok{ numpy    }
\NormalTok{cimport numpy}
\end{Highlighting}
\end{Shaded}
  \end{itemize}
\end{itemize}

    \begin{Verbatim}[commandchars=\\\{\}]
{\color{incolor}In [{\color{incolor}2}]:} \PY{o}{\PYZpc{}}\PY{k}{load\PYZus{}ext} cython
\end{Verbatim}

    \begin{Verbatim}[commandchars=\\\{\}]
{\color{incolor}In [{\color{incolor}3}]:} \PY{o}{\PYZpc{}}\PY{o}{\PYZpc{}}\PY{n}{cython} \PY{o}{\PYZhy{}}\PY{n}{a}
        \PY{c}{\PYZsh{}\PYZsh{}\PYZsh{} Example of a Cython cell (tutors)}
        
        \PY{k}{import} \PY{n+nn}{numpy}
        \PY{k}{cimport} \PY{n+nn}{numpy}
        
        \PY{k}{def} \PY{n+nf}{addC}\PY{p}{(}\PY{n}{numpy}\PY{o}{.}\PY{n}{ndarray}\PY{p}{[}\PY{n}{numpy}\PY{o}{.}\PY{n}{float64\PYZus{}t}\PY{p}{,}\PY{n}{ndim}\PY{o}{=}\PY{l+m+mf}{1}\PY{p}{]} \PY{n}{x}\PY{p}{,} \PY{n}{numpy}\PY{o}{.}\PY{n}{ndarray}\PY{p}{[}\PY{n}{numpy}\PY{o}{.}\PY{n}{float64\PYZus{}t}\PY{p}{,}\PY{n}{ndim}\PY{o}{=}\PY{l+m+mf}{1}\PY{p}{]} \PY{n}{y}\PY{p}{)}\PY{p}{:}
            \PY{c}{\PYZsh{} Python code with static typing, compiled}
            \PY{k}{cdef} \PY{k+kt}{int} \PY{n+nf}{i}
            \PY{k}{cdef} \PY{k+kt}{int} \PY{n+nf}{n} \PY{o}{=} \PY{n}{x}\PY{o}{.}\PY{n}{shape}\PY{p}{[}\PY{l+m+mf}{0}\PY{p}{]}
            \PY{k}{cdef} \PY{k+kt}{numpy}.\PY{k+kt}{ndarray}[\PY{k+kt}{numpy}.\PY{n+nf}{float64\PYZus{}t}\PY{p}{,}\PY{n+nf}{ndim}\PY{o}{=}\PY{l+m+mf}{1}\PY{p}{]} \PY{n}{z} \PY{o}{=} \PY{n}{numpy}\PY{o}{.}\PY{n}{zeros}\PY{p}{(}\PY{n}{n}\PY{p}{)}
            \PY{k}{for} \PY{n}{i} \PY{o+ow}{in} \PY{n+nb}{range}\PY{p}{(}\PY{n}{n}\PY{p}{)}\PY{p}{:}
                \PY{n}{z}\PY{p}{[}\PY{n}{i}\PY{p}{]} \PY{o}{=} \PY{n}{x}\PY{p}{[}\PY{n}{i}\PY{p}{]} \PY{o}{+} \PY{n}{y}\PY{p}{[}\PY{n}{i}\PY{p}{]}
            \PY{k}{return} \PY{n}{z}
            
        \PY{k}{def} \PY{n+nf}{addP}\PY{p}{(}\PY{n}{x}\PY{p}{,}\PY{n}{y}\PY{p}{)}\PY{p}{:}   \PY{c}{\PYZsh{} Python code with dynamic typing, compiled}
            \PY{n}{n} \PY{o}{=} \PY{n}{x}\PY{o}{.}\PY{n}{shape}\PY{p}{[}\PY{l+m+mf}{0}\PY{p}{]}
            \PY{n}{z} \PY{o}{=} \PY{n}{numpy}\PY{o}{.}\PY{n}{zeros}\PY{p}{(}\PY{n}{n}\PY{p}{)}
            \PY{k}{for} \PY{n}{i} \PY{o+ow}{in} \PY{n+nb}{range}\PY{p}{(}\PY{n}{n}\PY{p}{)}\PY{p}{:}
                \PY{n}{z}\PY{p}{[}\PY{n}{i}\PY{p}{]} \PY{o}{=} \PY{n}{x}\PY{p}{[}\PY{n}{i}\PY{p}{]} \PY{o}{+} \PY{n}{y}\PY{p}{[}\PY{n}{i}\PY{p}{]}
            \PY{k}{return} \PY{n}{z}
        
        \PY{n}{x} \PY{o}{=} \PY{n}{numpy}\PY{o}{.}\PY{n}{array}\PY{p}{(}\PY{p}{(}\PY{l+m+mf}{2.0}\PY{p}{,} \PY{l+m+mf}{11}\PY{p}{,} \PY{l+m+mf}{100}\PY{p}{)}\PY{p}{)}
        \PY{n}{y} \PY{o}{=} \PY{n}{numpy}\PY{o}{.}\PY{n}{array}\PY{p}{(}\PY{p}{(}\PY{l+m+mf}{3.0}\PY{p}{,} \PY{l+m+mf}{17}\PY{p}{,} \PY{l+m+mf}{2}\PY{p}{)}\PY{p}{)}
        \PY{n}{z} \PY{o}{=} \PY{n}{addC}\PY{p}{(}\PY{n}{x}\PY{p}{,}\PY{n}{y}\PY{p}{)}
        \PY{k}{print}\PY{p}{(}\PY{n}{z}\PY{p}{)}
\end{Verbatim}

    \begin{Verbatim}[commandchars=\\\{\}]
[   5.   28.  102.]

    \end{Verbatim}

\begin{Verbatim}[commandchars=\\\{\}]
{\color{outcolor}Out[{\color{outcolor}3}]:} <IPython.core.display.HTML object>
\end{Verbatim}
            
    \begin{Verbatim}[commandchars=\\\{\}]
{\color{incolor}In [{\color{incolor}35}]:} \PY{c+c1}{\PYZsh{} timing (tutors)}
         \PY{k+kn}{from} \PY{n+nn}{numpy}\PY{n+nn}{.}\PY{n+nn}{random} \PY{k}{import} \PY{n}{random}
         \PY{n}{x} \PY{o}{=} \PY{n}{random}\PY{p}{(}\PY{l+m+mi}{20000}\PY{p}{)}
         \PY{n}{y} \PY{o}{=} \PY{n}{random}\PY{p}{(}\PY{l+m+mi}{20000}\PY{p}{)}
         \PY{o}{\PYZpc{}}\PY{k}{timeit} z = addP(x,y)
         \PY{o}{\PYZpc{}}\PY{k}{timeit} z = addC(x,y)
         \PY{o}{\PYZpc{}}\PY{k}{timeit} z = x+y    \PYZsh{} numpy code
\end{Verbatim}

    \begin{Verbatim}[commandchars=\\\{\}]
18.4 ms ± 148 µs per loop (mean ± std. dev. of 7 runs, 10 loops each)
151 µs ± 487 ns per loop (mean ± std. dev. of 7 runs, 10000 loops each)
37.3 µs ± 2.96 µs per loop (mean ± std. dev. of 7 runs, 10000 loops each)

    \end{Verbatim}

    \subsection{B. Labbook -\/- here comes the part which you include or
modify
(student)}\label{b.-labbook----here-comes-the-part-which-you-include-or-modify-student}

    \subsubsection{B1. Lagrange interpolation (student)
{[}40pts{]}}\label{b1.-lagrange-interpolation-student-40pts}

\begin{enumerate}
\def\labelenumi{\arabic{enumi}.}
\tightlist
\item
  In the next 2 cells you find an implementation of the Lagrange
  interpolation formula and a timing and accuracy study. Carefully read
  through the code until you understand what is going on. Write a short
  description on what the various procedures do. Then run the next cell
  multiple times. Discuss what you observe in terms of accuracy and
  time. Develop a simple model for the time spent in the various
  routines. Does this explain the growth of time with increasing n?
\end{enumerate}

Note that you can improve the performance of the Lagrangian function by
precomputing the (constant) denominator. How much does this save you?

Write your report (1/2 page) here:

\subsubsection{My report}\label{my-report}

Your discussion goes here

    what the various procedures do:

In the following code:

First, we input the data points of exponential function with equal.

Second, we define the caridnal function(langrangua function) by the
formula: \[l_j(x) = w(x) \cdot \frac{c_j}{x-x_j}\] for \(x\neq x_j\) and
\(w(x) = \prod_{k=0 k\neq j}^n (x-x_k)\) and
\(c_j = \left(\prod_{k=0}^n (x_j-x_k)\right)^{-1}\)

Third, we perfrom lagrangian interpolation
\[p(x) = \sum_{j=0}^n y_j l_j(x)\]

    \begin{Verbatim}[commandchars=\\\{\}]
{\color{incolor}In [{\color{incolor}1}]:} \PY{o}{\PYZpc{}}\PY{k}{matplotlib} inline
        \PY{k+kn}{import} \PY{n+nn}{numpy}
        \PY{k+kn}{import} \PY{n+nn}{pylab} \PY{k}{as} \PY{n+nn}{plt}
        
        \PY{c+c1}{\PYZsh{}data}
        \PY{n}{n} \PY{o}{=} \PY{l+m+mi}{5}
        \PY{n}{xpts} \PY{o}{=} \PY{n}{numpy}\PY{o}{.}\PY{n}{linspace}\PY{p}{(}\PY{l+m+mi}{0}\PY{p}{,}\PY{l+m+mi}{1}\PY{p}{,}\PY{n}{n}\PY{p}{)}
        \PY{n}{f} \PY{o}{=} \PY{k}{lambda} \PY{n}{x}\PY{p}{:} \PY{n}{numpy}\PY{o}{.}\PY{n}{exp}\PY{p}{(}\PY{o}{\PYZhy{}}\PY{l+m+mi}{10}\PY{o}{*}\PY{n}{x}\PY{p}{)}
        \PY{n}{ypts} \PY{o}{=} \PY{n}{f}\PY{p}{(}\PY{n}{xpts}\PY{p}{)}
        
        \PY{l+s+sd}{\PYZsq{}\PYZsq{}\PYZsq{}}
        \PY{l+s+sd}{\PYZsh{} improved Lagrangian function}
        \PY{l+s+sd}{d=numpy.zeros(n)}
        \PY{l+s+sd}{for i in range(n):}
        \PY{l+s+sd}{    if (i != j):}
        \PY{l+s+sd}{        d[j] *= (xpts[j]\PYZhy{}xpts[i])}
        \PY{l+s+sd}{            }
        \PY{l+s+sd}{def l(x, xpts, j):}
        \PY{l+s+sd}{    y = 1.0}
        \PY{l+s+sd}{    d = 1.0}
        \PY{l+s+sd}{    n = xpts.shape[0]}
        \PY{l+s+sd}{    for i in range(n):}
        \PY{l+s+sd}{        if (i != j):}
        \PY{l+s+sd}{            d *= (xpts[j]\PYZhy{}xpts[i])}
        \PY{l+s+sd}{    for i in range(n):}
        \PY{l+s+sd}{        if (i != j): }
        \PY{l+s+sd}{            y *= (x\PYZhy{}xpts[i])}
        \PY{l+s+sd}{    y=y/d[j]}
        \PY{l+s+sd}{    return y}
        \PY{l+s+sd}{\PYZsq{}\PYZsq{}\PYZsq{}}
        
        \PY{c+c1}{\PYZsh{} Lagrangian function}
        \PY{k}{def} \PY{n+nf}{l}\PY{p}{(}\PY{n}{x}\PY{p}{,} \PY{n}{xpts}\PY{p}{,} \PY{n}{j}\PY{p}{)}\PY{p}{:}
            \PY{n}{y} \PY{o}{=} \PY{l+m+mf}{1.0}
            \PY{n}{n} \PY{o}{=} \PY{n}{xpts}\PY{o}{.}\PY{n}{shape}\PY{p}{[}\PY{l+m+mi}{0}\PY{p}{]}
            \PY{k}{for} \PY{n}{i} \PY{o+ow}{in} \PY{n+nb}{range}\PY{p}{(}\PY{n}{n}\PY{p}{)}\PY{p}{:}
                \PY{k}{if} \PY{p}{(}\PY{n}{i} \PY{o}{!=} \PY{n}{j}\PY{p}{)}\PY{p}{:} 
                    \PY{n}{y} \PY{o}{*}\PY{o}{=} \PY{p}{(}\PY{n}{x}\PY{o}{\PYZhy{}}\PY{n}{xpts}\PY{p}{[}\PY{n}{i}\PY{p}{]}\PY{p}{)}\PY{o}{/}\PY{p}{(}\PY{n}{xpts}\PY{p}{[}\PY{n}{j}\PY{p}{]}\PY{o}{\PYZhy{}}\PY{n}{xpts}\PY{p}{[}\PY{n}{i}\PY{p}{]}\PY{p}{)}
            \PY{k}{return} \PY{n}{y}
        
        \PY{c+c1}{\PYZsh{} Lagrangian interpolant}
        \PY{k}{def} \PY{n+nf}{p}\PY{p}{(}\PY{n}{x}\PY{p}{,} \PY{n}{xpts}\PY{o}{=}\PY{n}{xpts}\PY{p}{,} \PY{n}{ypts}\PY{o}{=}\PY{n}{ypts}\PY{p}{,} \PY{n}{l}\PY{o}{=}\PY{n}{l}\PY{p}{)}\PY{p}{:}
            \PY{n}{y} \PY{o}{=} \PY{l+m+mf}{0.0}
            \PY{n}{n} \PY{o}{=} \PY{n}{xpts}\PY{o}{.}\PY{n}{shape}\PY{p}{[}\PY{l+m+mi}{0}\PY{p}{]}
            \PY{k}{for} \PY{n}{j} \PY{o+ow}{in} \PY{n+nb}{range}\PY{p}{(}\PY{n}{n}\PY{p}{)}\PY{p}{:}
                \PY{n}{y} \PY{o}{+}\PY{o}{=} \PY{n}{ypts}\PY{p}{[}\PY{n}{j}\PY{p}{]}\PY{o}{*}\PY{n}{l}\PY{p}{(}\PY{n}{x}\PY{p}{,}\PY{n}{xpts}\PY{p}{,}\PY{n}{j}\PY{p}{)}
            \PY{k}{return} \PY{n}{y}
        
        \PY{c+c1}{\PYZsh{} plot result}
        \PY{n}{xg} \PY{o}{=} \PY{n}{numpy}\PY{o}{.}\PY{n}{linspace}\PY{p}{(}\PY{l+m+mi}{0}\PY{p}{,}\PY{l+m+mi}{1}\PY{p}{,}\PY{l+m+mi}{128}\PY{p}{)}
        \PY{n}{yg} \PY{o}{=} \PY{n}{p}\PY{p}{(}\PY{n}{xg}\PY{p}{)}
        \PY{n}{plt}\PY{o}{.}\PY{n}{plot}\PY{p}{(}\PY{n}{xg}\PY{p}{,}\PY{n}{yg}\PY{p}{,}\PY{n}{xg}\PY{p}{,}\PY{n}{f}\PY{p}{(}\PY{n}{xg}\PY{p}{)}\PY{p}{)}\PY{p}{;}
\end{Verbatim}

    \begin{center}
    \adjustimage{max size={0.9\linewidth}{0.9\paperheight}}{output_10_0.png}
    \end{center}
    { \hspace*{\fill} \\}
    
    \begin{Verbatim}[commandchars=\\\{\}]
{\color{incolor}In [{\color{incolor}36}]:} \PY{c+c1}{\PYZsh{} timing and accuracy study }
         \PY{k+kn}{from} \PY{n+nn}{numpy}\PY{n+nn}{.}\PY{n+nn}{random} \PY{k}{import} \PY{n}{random}
         \PY{n}{x} \PY{o}{=} \PY{n}{random}\PY{p}{(}\PY{p}{)}
         \PY{n+nb}{print}\PY{p}{(}\PY{n}{x}\PY{p}{,} \PY{n}{f}\PY{p}{(}\PY{n}{x}\PY{p}{)}\PY{p}{)}
         
         \PY{k}{for} \PY{n}{n} \PY{o+ow}{in}  \PY{p}{(}\PY{l+m+mi}{3}\PY{p}{,} \PY{l+m+mi}{5}\PY{p}{,} \PY{l+m+mi}{9}\PY{p}{,} \PY{l+m+mi}{17}\PY{p}{,} \PY{l+m+mi}{33}\PY{p}{,} \PY{l+m+mi}{65}\PY{p}{)}\PY{p}{:}
             \PY{n}{xpts} \PY{o}{=} \PY{n}{numpy}\PY{o}{.}\PY{n}{linspace}\PY{p}{(}\PY{l+m+mi}{0}\PY{p}{,}\PY{l+m+mi}{1}\PY{p}{,}\PY{n}{n}\PY{p}{)}
             \PY{n}{ypts} \PY{o}{=} \PY{n}{f}\PY{p}{(}\PY{n}{xpts}\PY{p}{)}
         
             \PY{n+nb}{print}\PY{p}{(}\PY{l+s+s2}{\PYZdq{}}\PY{l+s+s2}{n= }\PY{l+s+s2}{\PYZdq{}}\PY{p}{,} \PY{n}{n}\PY{p}{,} \PY{l+s+s2}{\PYZdq{}}\PY{l+s+s2}{ p(x) \PYZhy{} f(x) = }\PY{l+s+s2}{\PYZdq{}}\PY{p}{,} \PY{n}{p}\PY{p}{(}\PY{n}{x}\PY{p}{,}\PY{n}{xpts}\PY{p}{,}\PY{n}{ypts}\PY{p}{)} \PY{o}{\PYZhy{}} \PY{n}{f}\PY{p}{(}\PY{n}{x}\PY{p}{)}\PY{p}{)}
             
             \PY{o}{\PYZpc{}}\PY{k}{timeit} y = p(x,xpts,ypts)
\end{Verbatim}

    \begin{Verbatim}[commandchars=\\\{\}]
0.9289107503866519 9.242551289317041e-05
n=  3  p(x) - f(x) =  -0.05925835920813797
15 µs ± 465 ns per loop (mean ± std. dev. of 7 runs, 100000 loops each)
n=  5  p(x) - f(x) =  -0.024831423960763148
42.7 µs ± 1.61 µs per loop (mean ± std. dev. of 7 runs, 10000 loops each)
n=  9  p(x) - f(x) =  -0.0005053421039129846
142 µs ± 24.2 µs per loop (mean ± std. dev. of 7 runs, 10000 loops each)
n=  17  p(x) - f(x) =  8.360079957676122e-10
508 µs ± 34 µs per loop (mean ± std. dev. of 7 runs, 1000 loops each)
n=  33  p(x) - f(x) =  6.209022553917143e-14
1.98 ms ± 199 µs per loop (mean ± std. dev. of 7 runs, 1000 loops each)
n=  65  p(x) - f(x) =  8.501533292017865e-08
6.8 ms ± 274 µs per loop (mean ± std. dev. of 7 runs, 100 loops each)

    \end{Verbatim}

    performance in time

    \begin{Verbatim}[commandchars=\\\{\}]
{\color{incolor}In [{\color{incolor}27}]:} \PY{k+kn}{import} \PY{n+nn}{timeit} \PY{k}{as} \PY{n+nn}{ti}
         \PY{k+kn}{import} \PY{n+nn}{time}
         \PY{n}{x} \PY{o}{=} \PY{n}{random}\PY{p}{(}\PY{p}{)}
         \PY{n+nb}{print}\PY{p}{(}\PY{n}{x}\PY{p}{,} \PY{n}{f}\PY{p}{(}\PY{n}{x}\PY{p}{)}\PY{p}{)}
         \PY{n}{N}\PY{o}{=}\PY{l+m+mi}{10}
         \PY{n}{T}\PY{o}{=}\PY{p}{[}\PY{p}{]}
         \PY{n}{Narr}\PY{o}{=} \PY{n}{numpy}\PY{o}{.}\PY{n}{linspace}\PY{p}{(}\PY{l+m+mi}{3}\PY{p}{,}\PY{l+m+mi}{199}\PY{p}{,}\PY{l+m+mi}{50}\PY{p}{)}
         \PY{k}{for} \PY{n}{n} \PY{o+ow}{in} \PY{n}{Narr}\PY{p}{:}
             \PY{n}{xpts} \PY{o}{=} \PY{n}{numpy}\PY{o}{.}\PY{n}{linspace}\PY{p}{(}\PY{l+m+mi}{0}\PY{p}{,}\PY{l+m+mi}{1}\PY{p}{,}\PY{n}{n}\PY{p}{)}
             \PY{n}{ypts} \PY{o}{=} \PY{n}{f}\PY{p}{(}\PY{n}{xpts}\PY{p}{)}
             \PY{n}{y} \PY{o}{=} \PY{n}{p}\PY{p}{(}\PY{n}{x}\PY{p}{,}\PY{n}{xpts}\PY{p}{,}\PY{n}{ypts}\PY{p}{)}
             \PY{n}{T}\PY{o}{.}\PY{n}{append}\PY{p}{(}\PY{n}{ti}\PY{o}{.}\PY{n}{timeit}\PY{p}{(}\PY{l+s+s1}{\PYZsq{}}\PY{l+s+s1}{y = p(x,xpts,ypts)}\PY{l+s+s1}{\PYZsq{}}\PY{p}{,}
                          \PY{n}{setup}\PY{o}{=}\PY{l+s+s1}{\PYZsq{}}\PY{l+s+s1}{from \PYZus{}\PYZus{}main\PYZus{}\PYZus{} import p,x,xpts,ypts}\PY{l+s+s1}{\PYZsq{}}\PY{p}{,}\PY{n}{number}\PY{o}{=}\PY{l+m+mi}{20}\PY{p}{)}\PY{p}{)}
         \PY{n}{plt}\PY{o}{.}\PY{n}{plot}\PY{p}{(}\PY{n}{Narr}\PY{p}{,}\PY{n}{T}\PY{p}{)}
         \PY{n}{n}\PY{o}{=}\PY{n}{numpy}\PY{o}{.}\PY{n}{linspace}\PY{p}{(}\PY{l+m+mi}{3}\PY{p}{,} \PY{l+m+mi}{202}\PY{p}{,} \PY{l+m+mi}{200}\PY{p}{)}
         \PY{n}{y} \PY{o}{=} \PY{l+m+mf}{4e\PYZhy{}5}\PY{o}{*}\PY{n}{n}\PY{o}{*}\PY{o}{*}\PY{l+m+mi}{2}
         \PY{n}{plt}\PY{o}{.}\PY{n}{plot}\PY{p}{(}\PY{n}{n}\PY{p}{,}\PY{n}{y}\PY{p}{)}
         \PY{n}{plt}\PY{o}{.}\PY{n}{xlabel}\PY{p}{(}\PY{l+s+s1}{\PYZsq{}}\PY{l+s+s1}{n}\PY{l+s+s1}{\PYZsq{}}\PY{p}{)}
         \PY{n}{plt}\PY{o}{.}\PY{n}{ylabel}\PY{p}{(}\PY{l+s+s1}{\PYZsq{}}\PY{l+s+s1}{time}\PY{l+s+s1}{\PYZsq{}}\PY{p}{)}
\end{Verbatim}

    \begin{Verbatim}[commandchars=\\\{\}]
0.2702047026725797 0.06706808196927055

    \end{Verbatim}

    \begin{Verbatim}[commandchars=\\\{\}]
C:\textbackslash{}Users\textbackslash{}YYX\textbackslash{}Anaconda3\textbackslash{}lib\textbackslash{}site-packages\textbackslash{}ipykernel\_launcher.py:9: DeprecationWarning: object of type <class 'numpy.float64'> cannot be safely interpreted as an integer.
  if \_\_name\_\_ == '\_\_main\_\_':

    \end{Verbatim}

\begin{Verbatim}[commandchars=\\\{\}]
{\color{outcolor}Out[{\color{outcolor}27}]:} Text(0,0.5,'time')
\end{Verbatim}
            
    \begin{center}
    \adjustimage{max size={0.9\linewidth}{0.9\paperheight}}{output_13_3.png}
    \end{center}
    { \hspace*{\fill} \\}
    
    performance in accuracy

    \begin{Verbatim}[commandchars=\\\{\}]
{\color{incolor}In [{\color{incolor}48}]:} \PY{n}{x} \PY{o}{=} \PY{n}{random}\PY{p}{(}\PY{p}{)}
         \PY{n+nb}{print}\PY{p}{(}\PY{n}{x}\PY{p}{,} \PY{n}{f}\PY{p}{(}\PY{n}{x}\PY{p}{)}\PY{p}{)}
         
         \PY{n}{y}\PY{o}{=}\PY{p}{[}\PY{p}{]}
         
         \PY{n}{Narr}\PY{o}{=} \PY{n}{numpy}\PY{o}{.}\PY{n}{linspace}\PY{p}{(}\PY{l+m+mi}{3}\PY{p}{,}\PY{l+m+mi}{80}\PY{p}{,}\PY{l+m+mi}{78}\PY{p}{)}
         
         \PY{k}{for} \PY{n}{n} \PY{o+ow}{in} \PY{n}{Narr}\PY{p}{:}
             \PY{n}{xpts} \PY{o}{=} \PY{n}{numpy}\PY{o}{.}\PY{n}{linspace}\PY{p}{(}\PY{l+m+mi}{0}\PY{p}{,}\PY{l+m+mi}{1}\PY{p}{,}\PY{n}{n}\PY{p}{)}
             \PY{n}{ypts} \PY{o}{=} \PY{n}{f}\PY{p}{(}\PY{n}{xpts}\PY{p}{)}
             \PY{n}{yy} \PY{o}{=} \PY{n}{numpy}\PY{o}{.}\PY{n}{abs}\PY{p}{(}\PY{n}{p}\PY{p}{(}\PY{n}{x}\PY{p}{,}\PY{n}{xpts}\PY{p}{,}\PY{n}{ypts}\PY{p}{)} \PY{o}{\PYZhy{}} \PY{n}{f}\PY{p}{(}\PY{n}{x}\PY{p}{)}\PY{p}{)}
             \PY{n}{y}\PY{o}{.}\PY{n}{append}\PY{p}{(}\PY{n}{yy}\PY{p}{)}
         \PY{n}{plt}\PY{o}{.}\PY{n}{plot}\PY{p}{(}\PY{n}{Narr}\PY{p}{,}\PY{n}{y}\PY{p}{)}
         \PY{n}{plt}\PY{o}{.}\PY{n}{yscale}\PY{p}{(}\PY{l+s+s2}{\PYZdq{}}\PY{l+s+s2}{log}\PY{l+s+s2}{\PYZdq{}}\PY{p}{)}
         \PY{n}{plt}\PY{o}{.}\PY{n}{xlabel}\PY{p}{(}\PY{l+s+s1}{\PYZsq{}}\PY{l+s+s1}{n}\PY{l+s+s1}{\PYZsq{}}\PY{p}{)}
         \PY{n}{plt}\PY{o}{.}\PY{n}{ylabel}\PY{p}{(}\PY{l+s+s1}{\PYZsq{}}\PY{l+s+s1}{error}\PY{l+s+s1}{\PYZsq{}}\PY{p}{)}
\end{Verbatim}

    \begin{Verbatim}[commandchars=\\\{\}]
0.808336483651065 0.0003086307941147965

    \end{Verbatim}

    \begin{Verbatim}[commandchars=\\\{\}]
C:\textbackslash{}Users\textbackslash{}YYX\textbackslash{}Anaconda3\textbackslash{}lib\textbackslash{}site-packages\textbackslash{}ipykernel\_launcher.py:9: DeprecationWarning: object of type <class 'numpy.float64'> cannot be safely interpreted as an integer.
  if \_\_name\_\_ == '\_\_main\_\_':

    \end{Verbatim}

\begin{Verbatim}[commandchars=\\\{\}]
{\color{outcolor}Out[{\color{outcolor}48}]:} Text(0,0.5,'error')
\end{Verbatim}
            
    \begin{center}
    \adjustimage{max size={0.9\linewidth}{0.9\paperheight}}{output_15_3.png}
    \end{center}
    { \hspace*{\fill} \\}
    
    We observe that:

The first plot show us that T=\(4\times10^{-5} n^2\) is a good fit of
the relationship between calculation time and n. And we can see that the
calculation time is of order \(n^2\).

The second plot show us that the error reaches its minimum at around 27,
which means the accuaracy of the calculation is best when n is around
27.

    a simple model for the time spent in the various routines:

The time spent in every calculate langrangian function needs n division
and n-1 multiplication.

The time spent in intepolation needs to calculate n+1 langrangian
function.

The total calculation time is roughly \(2n^2 t_0\), where \(t_0\) is the
single calculation time for multiplication or division.

    By precomputing the (constant) denominator:

If we want to calculate the intepolation function in a region of x. Then
when calculating the langrangian function for all values of x, the
constant denominators are not changed for every x, so we can calculate
them in advance. Then the time spent in calculate langrangian function
will be roughly half beacuse we do not need to do n division but just 1.
Hence the whole calculating time will be half compared to the previous
method.

    \subsubsection{\texorpdfstring{B2. Performance improvement using Cython
(\textbf{not assessed},
student)}{B2. Performance improvement using Cython (not assessed, student)}}\label{b2.-performance-improvement-using-cython-not-assessed-student}

Most of the time spent in Lagrangian interpolation is spent in
evaluating the Lagrange functions l(x,y,j). * Can you confirm this with
your model?

Use Cython to improve performance, first by only including compilation
and in a second step by adding static typing. Redo the timings by using
the two new Cython variants of the Lagrange functions. Compare the
performance with the pure Python performance. Note that you do not need
to change the code for the interpolation function p.

\subsubsection{My report}\label{my-report}

your discussion of the results and your code goes here

    \begin{Verbatim}[commandchars=\\\{\}]
{\color{incolor}In [{\color{incolor}38}]:} \PY{c+c1}{\PYZsh{} Your Cython code goes here for the compiling only variant (not assessed)}
\end{Verbatim}

    \begin{Verbatim}[commandchars=\\\{\}]
{\color{incolor}In [{\color{incolor}39}]:} \PY{c+c1}{\PYZsh{} Your timing code goes here (not assessed)}
\end{Verbatim}

    \begin{Verbatim}[commandchars=\\\{\}]
{\color{incolor}In [{\color{incolor}40}]:} \PY{c+c1}{\PYZsh{} Your Cython code goes here for the compiling + static typing code (not assessed)}
\end{Verbatim}

    \begin{Verbatim}[commandchars=\\\{\}]
{\color{incolor}In [{\color{incolor}41}]:} \PY{c+c1}{\PYZsh{} Your timing code goes here (not assessed)}
\end{Verbatim}

    \subsubsection{B3. Barycentric formula (student)
{[}30pts{]}}\label{b3.-barycentric-formula-student-30pts}

\begin{enumerate}
\def\labelenumi{\arabic{enumi}.}
\setcounter{enumi}{1}
\tightlist
\item
  Repeat the performance study from point 1 replacing the Lagrange
  formula with the barycentric formula. Observe how the performance
  changes. Use a pure Python implementation. Your results are to be
  included here like in point 1.
\end{enumerate}

Note that here there are two components to the performance * for the
computation of the coefficients \(c_j\) * for the evaluation of the
polynomial with given \(c_j\)

    what the various procedures do:

In the following code:

First, we input the data points of exponential function with equal.

Second, we calculate \(c_j\) by the formula:
\(c_j = \left(\prod_{k=0}^n (x_j-x_k)\right)^{-1}\)

Third, we perfrom Barycentric formula:
\[p(x) = \frac{\sum_{j=0}^n y_j \frac{c_j}{x-x_j}}{\sum_{j=0}^n \frac{c_j}{x-x_j}}\]

    \begin{Verbatim}[commandchars=\\\{\}]
{\color{incolor}In [{\color{incolor}26}]:} \PY{o}{\PYZpc{}}\PY{k}{matplotlib} inline
         \PY{k+kn}{import} \PY{n+nn}{numpy}
         \PY{k+kn}{import} \PY{n+nn}{pylab} \PY{k}{as} \PY{n+nn}{plt}
         
         \PY{c+c1}{\PYZsh{}data}
         \PY{n}{n} \PY{o}{=} \PY{l+m+mi}{5}
         \PY{n}{xpts} \PY{o}{=} \PY{n}{numpy}\PY{o}{.}\PY{n}{linspace}\PY{p}{(}\PY{l+m+mi}{0}\PY{p}{,}\PY{l+m+mi}{1}\PY{p}{,}\PY{n}{n}\PY{p}{)}
         \PY{n}{f} \PY{o}{=} \PY{k}{lambda} \PY{n}{x}\PY{p}{:} \PY{n}{numpy}\PY{o}{.}\PY{n}{exp}\PY{p}{(}\PY{o}{\PYZhy{}}\PY{l+m+mi}{10}\PY{o}{*}\PY{n}{x}\PY{p}{)}
         \PY{n}{ypts} \PY{o}{=} \PY{n}{f}\PY{p}{(}\PY{n}{xpts}\PY{p}{)}
         
         \PY{c+c1}{\PYZsh{}c\PYZus{}j}
         \PY{k}{def} \PY{n+nf}{c}\PY{p}{(}\PY{n}{x}\PY{p}{,} \PY{n}{xpts}\PY{p}{,} \PY{n}{j}\PY{p}{)}\PY{p}{:}
             \PY{n}{y} \PY{o}{=} \PY{l+m+mf}{1.}
             \PY{n}{n} \PY{o}{=} \PY{n}{xpts}\PY{o}{.}\PY{n}{shape}\PY{p}{[}\PY{l+m+mi}{0}\PY{p}{]}
             \PY{k}{for} \PY{n}{i} \PY{o+ow}{in} \PY{n+nb}{range}\PY{p}{(}\PY{n}{n}\PY{p}{)}\PY{p}{:}
                 \PY{k}{if} \PY{p}{(}\PY{n}{i} \PY{o}{!=} \PY{n}{j}\PY{p}{)}\PY{p}{:}
                     \PY{n}{y} \PY{o}{*}\PY{o}{=} \PY{p}{(}\PY{n}{xpts}\PY{p}{[}\PY{n}{j}\PY{p}{]}\PY{o}{\PYZhy{}}\PY{n}{xpts}\PY{p}{[}\PY{n}{i}\PY{p}{]}\PY{p}{)}
             \PY{n}{y}\PY{o}{=}\PY{l+m+mi}{1}\PY{o}{/}\PY{n}{y}
             \PY{k}{return} \PY{n}{y}
         
         
         
         \PY{c+c1}{\PYZsh{} Barcentric interpolant}
         \PY{k}{def} \PY{n+nf}{b}\PY{p}{(}\PY{n}{x}\PY{p}{,} \PY{n}{xpts}\PY{o}{=}\PY{n}{xpts}\PY{p}{,} \PY{n}{ypts}\PY{o}{=}\PY{n}{ypts}\PY{p}{,} \PY{n}{l}\PY{o}{=}\PY{n}{l}\PY{p}{)}\PY{p}{:}
             \PY{n}{y} \PY{o}{=} \PY{l+m+mf}{0.0}
             \PY{n}{d} \PY{o}{=} \PY{l+m+mf}{0.0}
             \PY{n}{n} \PY{o}{=} \PY{n}{xpts}\PY{o}{.}\PY{n}{shape}\PY{p}{[}\PY{l+m+mi}{0}\PY{p}{]}
             \PY{n}{cc} \PY{o}{=} \PY{n}{numpy}\PY{o}{.}\PY{n}{zeros}\PY{p}{(}\PY{n}{n}\PY{p}{)}
             \PY{k}{for} \PY{n}{j} \PY{o+ow}{in} \PY{n+nb}{range}\PY{p}{(}\PY{n}{n}\PY{p}{)}\PY{p}{:}
                 \PY{n}{cc}\PY{p}{[}\PY{n}{j}\PY{p}{]} \PY{o}{=} \PY{n}{c}\PY{p}{(}\PY{n}{x}\PY{p}{,}\PY{n}{xpts}\PY{p}{,}\PY{n}{j}\PY{p}{)}
             \PY{k}{for} \PY{n}{j} \PY{o+ow}{in} \PY{n+nb}{range}\PY{p}{(}\PY{n}{n}\PY{p}{)}\PY{p}{:}
                 \PY{n}{y} \PY{o}{+}\PY{o}{=} \PY{n}{ypts}\PY{p}{[}\PY{n}{j}\PY{p}{]}\PY{o}{*}\PY{n}{cc}\PY{p}{[}\PY{n}{j}\PY{p}{]}\PY{o}{/}\PY{p}{(}\PY{n}{x}\PY{o}{\PYZhy{}}\PY{n}{xpts}\PY{p}{[}\PY{n}{j}\PY{p}{]}\PY{p}{)}
             \PY{k}{for} \PY{n}{j} \PY{o+ow}{in} \PY{n+nb}{range}\PY{p}{(}\PY{n}{n}\PY{p}{)}\PY{p}{:}
                 \PY{n}{d} \PY{o}{+}\PY{o}{=} \PY{n}{cc}\PY{p}{[}\PY{n}{j}\PY{p}{]}\PY{o}{/}\PY{p}{(}\PY{n}{x}\PY{o}{\PYZhy{}}\PY{n}{xpts}\PY{p}{[}\PY{n}{j}\PY{p}{]}\PY{p}{)}
             \PY{n}{y} \PY{o}{=} \PY{n}{y}\PY{o}{/}\PY{n}{d}
             \PY{k}{return} \PY{n}{y}
         
         \PY{c+c1}{\PYZsh{} plot result}
         \PY{n}{xg} \PY{o}{=} \PY{n}{numpy}\PY{o}{.}\PY{n}{linspace}\PY{p}{(}\PY{l+m+mi}{0}\PY{p}{,}\PY{l+m+mi}{1}\PY{p}{,}\PY{l+m+mi}{128}\PY{p}{)}
         \PY{n}{yg} \PY{o}{=} \PY{n}{b}\PY{p}{(}\PY{n}{xg}\PY{p}{)}
         \PY{n}{plt}\PY{o}{.}\PY{n}{plot}\PY{p}{(}\PY{n}{xg}\PY{p}{,}\PY{n}{yg}\PY{p}{,}\PY{n}{xg}\PY{p}{,}\PY{n}{f}\PY{p}{(}\PY{n}{xg}\PY{p}{)}\PY{p}{)}\PY{p}{;}
\end{Verbatim}

    \begin{Verbatim}[commandchars=\\\{\}]
C:\textbackslash{}Users\textbackslash{}YYX\textbackslash{}Anaconda3\textbackslash{}lib\textbackslash{}site-packages\textbackslash{}ipykernel\_launcher.py:32: RuntimeWarning: divide by zero encountered in true\_divide
C:\textbackslash{}Users\textbackslash{}YYX\textbackslash{}Anaconda3\textbackslash{}lib\textbackslash{}site-packages\textbackslash{}ipykernel\_launcher.py:34: RuntimeWarning: divide by zero encountered in true\_divide
C:\textbackslash{}Users\textbackslash{}YYX\textbackslash{}Anaconda3\textbackslash{}lib\textbackslash{}site-packages\textbackslash{}ipykernel\_launcher.py:35: RuntimeWarning: invalid value encountered in true\_divide

    \end{Verbatim}

    \begin{center}
    \adjustimage{max size={0.9\linewidth}{0.9\paperheight}}{output_26_1.png}
    \end{center}
    { \hspace*{\fill} \\}
    
    \begin{Verbatim}[commandchars=\\\{\}]
{\color{incolor}In [{\color{incolor}10}]:} \PY{c+c1}{\PYZsh{} timing and accuracy study }
         
         \PY{n}{x} \PY{o}{=} \PY{n}{random}\PY{p}{(}\PY{p}{)}
         \PY{n+nb}{print}\PY{p}{(}\PY{n}{x}\PY{p}{,} \PY{n}{f}\PY{p}{(}\PY{n}{x}\PY{p}{)}\PY{p}{)}
         
         \PY{k}{for} \PY{n}{n} \PY{o+ow}{in}  \PY{p}{(}\PY{l+m+mi}{3}\PY{p}{,} \PY{l+m+mi}{5}\PY{p}{,} \PY{l+m+mi}{9}\PY{p}{,} \PY{l+m+mi}{17}\PY{p}{,} \PY{l+m+mi}{33}\PY{p}{,} \PY{l+m+mi}{65}\PY{p}{)}\PY{p}{:}
             \PY{n}{xpts} \PY{o}{=} \PY{n}{numpy}\PY{o}{.}\PY{n}{linspace}\PY{p}{(}\PY{l+m+mi}{0}\PY{p}{,}\PY{l+m+mi}{1}\PY{p}{,}\PY{n}{n}\PY{p}{)}
             \PY{n}{ypts} \PY{o}{=} \PY{n}{f}\PY{p}{(}\PY{n}{xpts}\PY{p}{)}
         
             \PY{n+nb}{print}\PY{p}{(}\PY{l+s+s2}{\PYZdq{}}\PY{l+s+s2}{n= }\PY{l+s+s2}{\PYZdq{}}\PY{p}{,} \PY{n}{n}\PY{p}{,} \PY{l+s+s2}{\PYZdq{}}\PY{l+s+s2}{ b(x) \PYZhy{} f(x) = }\PY{l+s+s2}{\PYZdq{}}\PY{p}{,} \PY{n}{b}\PY{p}{(}\PY{n}{x}\PY{p}{,}\PY{n}{xpts}\PY{p}{,}\PY{n}{ypts}\PY{p}{)} \PY{o}{\PYZhy{}} \PY{n}{f}\PY{p}{(}\PY{n}{x}\PY{p}{)}\PY{p}{)}
             
             \PY{o}{\PYZpc{}}\PY{k}{timeit} y = b(x,xpts,ypts)
\end{Verbatim}

    \begin{Verbatim}[commandchars=\\\{\}]
0.0650774212793307 0.521641758660128
n=  3  b(x) - f(x) =  0.29323335918847604
26.4 µs ± 1.8 µs per loop (mean ± std. dev. of 7 runs, 100000 loops each)
n=  5  b(x) - f(x) =  0.08797347251311394
51.2 µs ± 6.03 µs per loop (mean ± std. dev. of 7 runs, 10000 loops each)
n=  9  b(x) - f(x) =  0.001310196727692059
128 µs ± 25.8 µs per loop (mean ± std. dev. of 7 runs, 10000 loops each)
n=  17  b(x) - f(x) =  -5.162064109498488e-10
322 µs ± 16.6 µs per loop (mean ± std. dev. of 7 runs, 1000 loops each)
n=  33  b(x) - f(x) =  4.667377595524158e-13
1.16 ms ± 102 µs per loop (mean ± std. dev. of 7 runs, 1000 loops each)
n=  65  b(x) - f(x) =  1.0048724943256815e-06
4.31 ms ± 33.9 µs per loop (mean ± std. dev. of 7 runs, 100 loops each)

    \end{Verbatim}

    performance in time

    \begin{Verbatim}[commandchars=\\\{\}]
{\color{incolor}In [{\color{incolor}32}]:} \PY{n}{x} \PY{o}{=} \PY{n}{random}\PY{p}{(}\PY{p}{)}
         \PY{n+nb}{print}\PY{p}{(}\PY{n}{x}\PY{p}{,} \PY{n}{f}\PY{p}{(}\PY{n}{x}\PY{p}{)}\PY{p}{)}
         \PY{n}{T}\PY{o}{=}\PY{p}{[}\PY{p}{]}
         \PY{n}{Narr}\PY{o}{=} \PY{n}{numpy}\PY{o}{.}\PY{n}{linspace}\PY{p}{(}\PY{l+m+mi}{3}\PY{p}{,}\PY{l+m+mi}{199}\PY{p}{,}\PY{l+m+mi}{50}\PY{p}{)}
         \PY{k}{for} \PY{n}{n} \PY{o+ow}{in} \PY{n}{Narr}\PY{p}{:}
             \PY{n}{xpts} \PY{o}{=} \PY{n}{numpy}\PY{o}{.}\PY{n}{linspace}\PY{p}{(}\PY{l+m+mi}{0}\PY{p}{,}\PY{l+m+mi}{1}\PY{p}{,}\PY{n}{n}\PY{p}{)}
             \PY{n}{ypts} \PY{o}{=} \PY{n}{f}\PY{p}{(}\PY{n}{xpts}\PY{p}{)}
             \PY{n}{y} \PY{o}{=} \PY{n}{b}\PY{p}{(}\PY{n}{x}\PY{p}{,}\PY{n}{xpts}\PY{p}{,}\PY{n}{ypts}\PY{p}{)}
             \PY{n}{T}\PY{o}{.}\PY{n}{append}\PY{p}{(}\PY{n}{ti}\PY{o}{.}\PY{n}{timeit}\PY{p}{(}\PY{l+s+s1}{\PYZsq{}}\PY{l+s+s1}{y = b(x,xpts,ypts)}\PY{l+s+s1}{\PYZsq{}}\PY{p}{,}
                          \PY{n}{setup}\PY{o}{=}\PY{l+s+s1}{\PYZsq{}}\PY{l+s+s1}{from \PYZus{}\PYZus{}main\PYZus{}\PYZus{} import b,x,xpts,ypts}\PY{l+s+s1}{\PYZsq{}}\PY{p}{,}\PY{n}{number}\PY{o}{=}\PY{l+m+mi}{20}\PY{p}{)}\PY{p}{)}
         \PY{n}{plt}\PY{o}{.}\PY{n}{plot}\PY{p}{(}\PY{n}{Narr}\PY{p}{,}\PY{n}{T}\PY{p}{)}
         \PY{n}{n}\PY{o}{=}\PY{n}{numpy}\PY{o}{.}\PY{n}{linspace}\PY{p}{(}\PY{l+m+mi}{3}\PY{p}{,} \PY{l+m+mi}{201}\PY{p}{,} \PY{l+m+mi}{100}\PY{p}{)}
         \PY{n}{y} \PY{o}{=} \PY{l+m+mf}{2.1e\PYZhy{}5}\PY{o}{*}\PY{n}{n}\PY{o}{*}\PY{o}{*}\PY{l+m+mi}{2}
         \PY{n}{plt}\PY{o}{.}\PY{n}{plot}\PY{p}{(}\PY{n}{n}\PY{p}{,}\PY{n}{y}\PY{p}{)}
         \PY{n}{plt}\PY{o}{.}\PY{n}{xlabel}\PY{p}{(}\PY{l+s+s1}{\PYZsq{}}\PY{l+s+s1}{n}\PY{l+s+s1}{\PYZsq{}}\PY{p}{)}
         \PY{n}{plt}\PY{o}{.}\PY{n}{ylabel}\PY{p}{(}\PY{l+s+s1}{\PYZsq{}}\PY{l+s+s1}{time}\PY{l+s+s1}{\PYZsq{}}\PY{p}{)}
\end{Verbatim}

    \begin{Verbatim}[commandchars=\\\{\}]
0.761234354879148 0.0004943120555604486

    \end{Verbatim}

    \begin{Verbatim}[commandchars=\\\{\}]
C:\textbackslash{}Users\textbackslash{}YYX\textbackslash{}Anaconda3\textbackslash{}lib\textbackslash{}site-packages\textbackslash{}ipykernel\_launcher.py:6: DeprecationWarning: object of type <class 'numpy.float64'> cannot be safely interpreted as an integer.
  

    \end{Verbatim}

\begin{Verbatim}[commandchars=\\\{\}]
{\color{outcolor}Out[{\color{outcolor}32}]:} Text(0,0.5,'time')
\end{Verbatim}
            
    \begin{center}
    \adjustimage{max size={0.9\linewidth}{0.9\paperheight}}{output_29_3.png}
    \end{center}
    { \hspace*{\fill} \\}
    
    performance in accuracy

    \begin{Verbatim}[commandchars=\\\{\}]
{\color{incolor}In [{\color{incolor}50}]:} \PY{n}{x} \PY{o}{=} \PY{n}{random}\PY{p}{(}\PY{p}{)}
         \PY{n+nb}{print}\PY{p}{(}\PY{n}{x}\PY{p}{,} \PY{n}{f}\PY{p}{(}\PY{n}{x}\PY{p}{)}\PY{p}{)}
         
         \PY{n}{y}\PY{o}{=}\PY{p}{[}\PY{p}{]}
         
         \PY{n}{Narr}\PY{o}{=} \PY{n}{numpy}\PY{o}{.}\PY{n}{linspace}\PY{p}{(}\PY{l+m+mi}{3}\PY{p}{,}\PY{l+m+mi}{80}\PY{p}{,}\PY{l+m+mi}{78}\PY{p}{)}
         
         \PY{k}{for} \PY{n}{n} \PY{o+ow}{in} \PY{n}{Narr}\PY{p}{:}
             \PY{n}{xpts} \PY{o}{=} \PY{n}{numpy}\PY{o}{.}\PY{n}{linspace}\PY{p}{(}\PY{l+m+mi}{0}\PY{p}{,}\PY{l+m+mi}{1}\PY{p}{,}\PY{n}{n}\PY{p}{)}
             \PY{n}{ypts} \PY{o}{=} \PY{n}{f}\PY{p}{(}\PY{n}{xpts}\PY{p}{)}
             \PY{n}{yy} \PY{o}{=} \PY{n}{numpy}\PY{o}{.}\PY{n}{abs}\PY{p}{(}\PY{n}{b}\PY{p}{(}\PY{n}{x}\PY{p}{,}\PY{n}{xpts}\PY{p}{,}\PY{n}{ypts}\PY{p}{)} \PY{o}{\PYZhy{}} \PY{n}{f}\PY{p}{(}\PY{n}{x}\PY{p}{)}\PY{p}{)}
             \PY{n}{y}\PY{o}{.}\PY{n}{append}\PY{p}{(}\PY{n}{yy}\PY{p}{)}
         \PY{n}{plt}\PY{o}{.}\PY{n}{plot}\PY{p}{(}\PY{n}{Narr}\PY{p}{,}\PY{n}{y}\PY{p}{)}
         \PY{n}{plt}\PY{o}{.}\PY{n}{yscale}\PY{p}{(}\PY{l+s+s2}{\PYZdq{}}\PY{l+s+s2}{log}\PY{l+s+s2}{\PYZdq{}}\PY{p}{)}
         \PY{n}{plt}\PY{o}{.}\PY{n}{xlabel}\PY{p}{(}\PY{l+s+s1}{\PYZsq{}}\PY{l+s+s1}{n}\PY{l+s+s1}{\PYZsq{}}\PY{p}{)}
         \PY{n}{plt}\PY{o}{.}\PY{n}{ylabel}\PY{p}{(}\PY{l+s+s1}{\PYZsq{}}\PY{l+s+s1}{error}\PY{l+s+s1}{\PYZsq{}}\PY{p}{)}
\end{Verbatim}

    \begin{Verbatim}[commandchars=\\\{\}]
0.15288993785390115 0.21677412231879992

    \end{Verbatim}

    \begin{Verbatim}[commandchars=\\\{\}]
C:\textbackslash{}Users\textbackslash{}YYX\textbackslash{}Anaconda3\textbackslash{}lib\textbackslash{}site-packages\textbackslash{}ipykernel\_launcher.py:9: DeprecationWarning: object of type <class 'numpy.float64'> cannot be safely interpreted as an integer.
  if \_\_name\_\_ == '\_\_main\_\_':

    \end{Verbatim}

\begin{Verbatim}[commandchars=\\\{\}]
{\color{outcolor}Out[{\color{outcolor}50}]:} Text(0,0.5,'error')
\end{Verbatim}
            
    \begin{center}
    \adjustimage{max size={0.9\linewidth}{0.9\paperheight}}{output_31_3.png}
    \end{center}
    { \hspace*{\fill} \\}
    
    We observe that:

The first plot show us that T=\(2.1\times10^{-5} n^2\) is a good fit of
the relationship between calculation time and n. And we can see that the
calculation time is of order \(n^2\).

The second plot show us that the error reaches its minimum at around 27.
The accuaracy of the calculation is best when n is around 27.

    Comparison between lagrangian formula with baycentric formula:

For the computation of the coefficients \(c_j\): there are n
multiplication and 1 division.

For the evaluation of the intepolation: we have to calculate n+1 \(c_j\)

The total calculation time is roughly \(n^2 t_0\). The time required for
computation is nearly the half the time of time of previous method.

The performance of accuracy has not changed much.

    \subsubsection{B4. Barycentric formula with Cython (student, not
assessed)}\label{b4.-barycentric-formula-with-cython-student-not-assessed}

Repeat the performance study from point 2 for the barycentric formula,
again considering both the effect of compilation and of static typing
separately.


    % Add a bibliography block to the postdoc
    
    
    
    \end{document}
