
% Default to the notebook output style

    


% Inherit from the specified cell style.




    
\documentclass[11pt]{article}

    
    
    \usepackage[T1]{fontenc}
    % Nicer default font (+ math font) than Computer Modern for most use cases
    \usepackage{mathpazo}

    % Basic figure setup, for now with no caption control since it's done
    % automatically by Pandoc (which extracts ![](path) syntax from Markdown).
    \usepackage{graphicx}
    % We will generate all images so they have a width \maxwidth. This means
    % that they will get their normal width if they fit onto the page, but
    % are scaled down if they would overflow the margins.
    \makeatletter
    \def\maxwidth{\ifdim\Gin@nat@width>\linewidth\linewidth
    \else\Gin@nat@width\fi}
    \makeatother
    \let\Oldincludegraphics\includegraphics
    % Set max figure width to be 80% of text width, for now hardcoded.
    \renewcommand{\includegraphics}[1]{\Oldincludegraphics[width=.8\maxwidth]{#1}}
    % Ensure that by default, figures have no caption (until we provide a
    % proper Figure object with a Caption API and a way to capture that
    % in the conversion process - todo).
    \usepackage{caption}
    \DeclareCaptionLabelFormat{nolabel}{}
    \captionsetup{labelformat=nolabel}

    \usepackage{adjustbox} % Used to constrain images to a maximum size 
    \usepackage{xcolor} % Allow colors to be defined
    \usepackage{enumerate} % Needed for markdown enumerations to work
    \usepackage{geometry} % Used to adjust the document margins
    \usepackage{amsmath} % Equations
    \usepackage{amssymb} % Equations
    \usepackage{textcomp} % defines textquotesingle
    % Hack from http://tex.stackexchange.com/a/47451/13684:
    \AtBeginDocument{%
        \def\PYZsq{\textquotesingle}% Upright quotes in Pygmentized code
    }
    \usepackage{upquote} % Upright quotes for verbatim code
    \usepackage{eurosym} % defines \euro
    \usepackage[mathletters]{ucs} % Extended unicode (utf-8) support
    \usepackage[utf8x]{inputenc} % Allow utf-8 characters in the tex document
    \usepackage{fancyvrb} % verbatim replacement that allows latex
    \usepackage{grffile} % extends the file name processing of package graphics 
                         % to support a larger range 
    % The hyperref package gives us a pdf with properly built
    % internal navigation ('pdf bookmarks' for the table of contents,
    % internal cross-reference links, web links for URLs, etc.)
    \usepackage{hyperref}
    \usepackage{longtable} % longtable support required by pandoc >1.10
    \usepackage{booktabs}  % table support for pandoc > 1.12.2
    \usepackage[inline]{enumitem} % IRkernel/repr support (it uses the enumerate* environment)
    \usepackage[normalem]{ulem} % ulem is needed to support strikethroughs (\sout)
                                % normalem makes italics be italics, not underlines
    

    
    
    % Colors for the hyperref package
    \definecolor{urlcolor}{rgb}{0,.145,.698}
    \definecolor{linkcolor}{rgb}{.71,0.21,0.01}
    \definecolor{citecolor}{rgb}{.12,.54,.11}

    % ANSI colors
    \definecolor{ansi-black}{HTML}{3E424D}
    \definecolor{ansi-black-intense}{HTML}{282C36}
    \definecolor{ansi-red}{HTML}{E75C58}
    \definecolor{ansi-red-intense}{HTML}{B22B31}
    \definecolor{ansi-green}{HTML}{00A250}
    \definecolor{ansi-green-intense}{HTML}{007427}
    \definecolor{ansi-yellow}{HTML}{DDB62B}
    \definecolor{ansi-yellow-intense}{HTML}{B27D12}
    \definecolor{ansi-blue}{HTML}{208FFB}
    \definecolor{ansi-blue-intense}{HTML}{0065CA}
    \definecolor{ansi-magenta}{HTML}{D160C4}
    \definecolor{ansi-magenta-intense}{HTML}{A03196}
    \definecolor{ansi-cyan}{HTML}{60C6C8}
    \definecolor{ansi-cyan-intense}{HTML}{258F8F}
    \definecolor{ansi-white}{HTML}{C5C1B4}
    \definecolor{ansi-white-intense}{HTML}{A1A6B2}

    % commands and environments needed by pandoc snippets
    % extracted from the output of `pandoc -s`
    \providecommand{\tightlist}{%
      \setlength{\itemsep}{0pt}\setlength{\parskip}{0pt}}
    \DefineVerbatimEnvironment{Highlighting}{Verbatim}{commandchars=\\\{\}}
    % Add ',fontsize=\small' for more characters per line
    \newenvironment{Shaded}{}{}
    \newcommand{\KeywordTok}[1]{\textcolor[rgb]{0.00,0.44,0.13}{\textbf{{#1}}}}
    \newcommand{\DataTypeTok}[1]{\textcolor[rgb]{0.56,0.13,0.00}{{#1}}}
    \newcommand{\DecValTok}[1]{\textcolor[rgb]{0.25,0.63,0.44}{{#1}}}
    \newcommand{\BaseNTok}[1]{\textcolor[rgb]{0.25,0.63,0.44}{{#1}}}
    \newcommand{\FloatTok}[1]{\textcolor[rgb]{0.25,0.63,0.44}{{#1}}}
    \newcommand{\CharTok}[1]{\textcolor[rgb]{0.25,0.44,0.63}{{#1}}}
    \newcommand{\StringTok}[1]{\textcolor[rgb]{0.25,0.44,0.63}{{#1}}}
    \newcommand{\CommentTok}[1]{\textcolor[rgb]{0.38,0.63,0.69}{\textit{{#1}}}}
    \newcommand{\OtherTok}[1]{\textcolor[rgb]{0.00,0.44,0.13}{{#1}}}
    \newcommand{\AlertTok}[1]{\textcolor[rgb]{1.00,0.00,0.00}{\textbf{{#1}}}}
    \newcommand{\FunctionTok}[1]{\textcolor[rgb]{0.02,0.16,0.49}{{#1}}}
    \newcommand{\RegionMarkerTok}[1]{{#1}}
    \newcommand{\ErrorTok}[1]{\textcolor[rgb]{1.00,0.00,0.00}{\textbf{{#1}}}}
    \newcommand{\NormalTok}[1]{{#1}}
    
    % Additional commands for more recent versions of Pandoc
    \newcommand{\ConstantTok}[1]{\textcolor[rgb]{0.53,0.00,0.00}{{#1}}}
    \newcommand{\SpecialCharTok}[1]{\textcolor[rgb]{0.25,0.44,0.63}{{#1}}}
    \newcommand{\VerbatimStringTok}[1]{\textcolor[rgb]{0.25,0.44,0.63}{{#1}}}
    \newcommand{\SpecialStringTok}[1]{\textcolor[rgb]{0.73,0.40,0.53}{{#1}}}
    \newcommand{\ImportTok}[1]{{#1}}
    \newcommand{\DocumentationTok}[1]{\textcolor[rgb]{0.73,0.13,0.13}{\textit{{#1}}}}
    \newcommand{\AnnotationTok}[1]{\textcolor[rgb]{0.38,0.63,0.69}{\textbf{\textit{{#1}}}}}
    \newcommand{\CommentVarTok}[1]{\textcolor[rgb]{0.38,0.63,0.69}{\textbf{\textit{{#1}}}}}
    \newcommand{\VariableTok}[1]{\textcolor[rgb]{0.10,0.09,0.49}{{#1}}}
    \newcommand{\ControlFlowTok}[1]{\textcolor[rgb]{0.00,0.44,0.13}{\textbf{{#1}}}}
    \newcommand{\OperatorTok}[1]{\textcolor[rgb]{0.40,0.40,0.40}{{#1}}}
    \newcommand{\BuiltInTok}[1]{{#1}}
    \newcommand{\ExtensionTok}[1]{{#1}}
    \newcommand{\PreprocessorTok}[1]{\textcolor[rgb]{0.74,0.48,0.00}{{#1}}}
    \newcommand{\AttributeTok}[1]{\textcolor[rgb]{0.49,0.56,0.16}{{#1}}}
    \newcommand{\InformationTok}[1]{\textcolor[rgb]{0.38,0.63,0.69}{\textbf{\textit{{#1}}}}}
    \newcommand{\WarningTok}[1]{\textcolor[rgb]{0.38,0.63,0.69}{\textbf{\textit{{#1}}}}}
    
    
    % Define a nice break command that doesn't care if a line doesn't already
    % exist.
    \def\br{\hspace*{\fill} \\* }
    % Math Jax compatability definitions
    \def\gt{>}
    \def\lt{<}
    % Document parameters
    \title{DirectSolvers}
    
    
    

    % Pygments definitions
    
\makeatletter
\def\PY@reset{\let\PY@it=\relax \let\PY@bf=\relax%
    \let\PY@ul=\relax \let\PY@tc=\relax%
    \let\PY@bc=\relax \let\PY@ff=\relax}
\def\PY@tok#1{\csname PY@tok@#1\endcsname}
\def\PY@toks#1+{\ifx\relax#1\empty\else%
    \PY@tok{#1}\expandafter\PY@toks\fi}
\def\PY@do#1{\PY@bc{\PY@tc{\PY@ul{%
    \PY@it{\PY@bf{\PY@ff{#1}}}}}}}
\def\PY#1#2{\PY@reset\PY@toks#1+\relax+\PY@do{#2}}

\expandafter\def\csname PY@tok@w\endcsname{\def\PY@tc##1{\textcolor[rgb]{0.73,0.73,0.73}{##1}}}
\expandafter\def\csname PY@tok@c\endcsname{\let\PY@it=\textit\def\PY@tc##1{\textcolor[rgb]{0.25,0.50,0.50}{##1}}}
\expandafter\def\csname PY@tok@cp\endcsname{\def\PY@tc##1{\textcolor[rgb]{0.74,0.48,0.00}{##1}}}
\expandafter\def\csname PY@tok@k\endcsname{\let\PY@bf=\textbf\def\PY@tc##1{\textcolor[rgb]{0.00,0.50,0.00}{##1}}}
\expandafter\def\csname PY@tok@kp\endcsname{\def\PY@tc##1{\textcolor[rgb]{0.00,0.50,0.00}{##1}}}
\expandafter\def\csname PY@tok@kt\endcsname{\def\PY@tc##1{\textcolor[rgb]{0.69,0.00,0.25}{##1}}}
\expandafter\def\csname PY@tok@o\endcsname{\def\PY@tc##1{\textcolor[rgb]{0.40,0.40,0.40}{##1}}}
\expandafter\def\csname PY@tok@ow\endcsname{\let\PY@bf=\textbf\def\PY@tc##1{\textcolor[rgb]{0.67,0.13,1.00}{##1}}}
\expandafter\def\csname PY@tok@nb\endcsname{\def\PY@tc##1{\textcolor[rgb]{0.00,0.50,0.00}{##1}}}
\expandafter\def\csname PY@tok@nf\endcsname{\def\PY@tc##1{\textcolor[rgb]{0.00,0.00,1.00}{##1}}}
\expandafter\def\csname PY@tok@nc\endcsname{\let\PY@bf=\textbf\def\PY@tc##1{\textcolor[rgb]{0.00,0.00,1.00}{##1}}}
\expandafter\def\csname PY@tok@nn\endcsname{\let\PY@bf=\textbf\def\PY@tc##1{\textcolor[rgb]{0.00,0.00,1.00}{##1}}}
\expandafter\def\csname PY@tok@ne\endcsname{\let\PY@bf=\textbf\def\PY@tc##1{\textcolor[rgb]{0.82,0.25,0.23}{##1}}}
\expandafter\def\csname PY@tok@nv\endcsname{\def\PY@tc##1{\textcolor[rgb]{0.10,0.09,0.49}{##1}}}
\expandafter\def\csname PY@tok@no\endcsname{\def\PY@tc##1{\textcolor[rgb]{0.53,0.00,0.00}{##1}}}
\expandafter\def\csname PY@tok@nl\endcsname{\def\PY@tc##1{\textcolor[rgb]{0.63,0.63,0.00}{##1}}}
\expandafter\def\csname PY@tok@ni\endcsname{\let\PY@bf=\textbf\def\PY@tc##1{\textcolor[rgb]{0.60,0.60,0.60}{##1}}}
\expandafter\def\csname PY@tok@na\endcsname{\def\PY@tc##1{\textcolor[rgb]{0.49,0.56,0.16}{##1}}}
\expandafter\def\csname PY@tok@nt\endcsname{\let\PY@bf=\textbf\def\PY@tc##1{\textcolor[rgb]{0.00,0.50,0.00}{##1}}}
\expandafter\def\csname PY@tok@nd\endcsname{\def\PY@tc##1{\textcolor[rgb]{0.67,0.13,1.00}{##1}}}
\expandafter\def\csname PY@tok@s\endcsname{\def\PY@tc##1{\textcolor[rgb]{0.73,0.13,0.13}{##1}}}
\expandafter\def\csname PY@tok@sd\endcsname{\let\PY@it=\textit\def\PY@tc##1{\textcolor[rgb]{0.73,0.13,0.13}{##1}}}
\expandafter\def\csname PY@tok@si\endcsname{\let\PY@bf=\textbf\def\PY@tc##1{\textcolor[rgb]{0.73,0.40,0.53}{##1}}}
\expandafter\def\csname PY@tok@se\endcsname{\let\PY@bf=\textbf\def\PY@tc##1{\textcolor[rgb]{0.73,0.40,0.13}{##1}}}
\expandafter\def\csname PY@tok@sr\endcsname{\def\PY@tc##1{\textcolor[rgb]{0.73,0.40,0.53}{##1}}}
\expandafter\def\csname PY@tok@ss\endcsname{\def\PY@tc##1{\textcolor[rgb]{0.10,0.09,0.49}{##1}}}
\expandafter\def\csname PY@tok@sx\endcsname{\def\PY@tc##1{\textcolor[rgb]{0.00,0.50,0.00}{##1}}}
\expandafter\def\csname PY@tok@m\endcsname{\def\PY@tc##1{\textcolor[rgb]{0.40,0.40,0.40}{##1}}}
\expandafter\def\csname PY@tok@gh\endcsname{\let\PY@bf=\textbf\def\PY@tc##1{\textcolor[rgb]{0.00,0.00,0.50}{##1}}}
\expandafter\def\csname PY@tok@gu\endcsname{\let\PY@bf=\textbf\def\PY@tc##1{\textcolor[rgb]{0.50,0.00,0.50}{##1}}}
\expandafter\def\csname PY@tok@gd\endcsname{\def\PY@tc##1{\textcolor[rgb]{0.63,0.00,0.00}{##1}}}
\expandafter\def\csname PY@tok@gi\endcsname{\def\PY@tc##1{\textcolor[rgb]{0.00,0.63,0.00}{##1}}}
\expandafter\def\csname PY@tok@gr\endcsname{\def\PY@tc##1{\textcolor[rgb]{1.00,0.00,0.00}{##1}}}
\expandafter\def\csname PY@tok@ge\endcsname{\let\PY@it=\textit}
\expandafter\def\csname PY@tok@gs\endcsname{\let\PY@bf=\textbf}
\expandafter\def\csname PY@tok@gp\endcsname{\let\PY@bf=\textbf\def\PY@tc##1{\textcolor[rgb]{0.00,0.00,0.50}{##1}}}
\expandafter\def\csname PY@tok@go\endcsname{\def\PY@tc##1{\textcolor[rgb]{0.53,0.53,0.53}{##1}}}
\expandafter\def\csname PY@tok@gt\endcsname{\def\PY@tc##1{\textcolor[rgb]{0.00,0.27,0.87}{##1}}}
\expandafter\def\csname PY@tok@err\endcsname{\def\PY@bc##1{\setlength{\fboxsep}{0pt}\fcolorbox[rgb]{1.00,0.00,0.00}{1,1,1}{\strut ##1}}}
\expandafter\def\csname PY@tok@kc\endcsname{\let\PY@bf=\textbf\def\PY@tc##1{\textcolor[rgb]{0.00,0.50,0.00}{##1}}}
\expandafter\def\csname PY@tok@kd\endcsname{\let\PY@bf=\textbf\def\PY@tc##1{\textcolor[rgb]{0.00,0.50,0.00}{##1}}}
\expandafter\def\csname PY@tok@kn\endcsname{\let\PY@bf=\textbf\def\PY@tc##1{\textcolor[rgb]{0.00,0.50,0.00}{##1}}}
\expandafter\def\csname PY@tok@kr\endcsname{\let\PY@bf=\textbf\def\PY@tc##1{\textcolor[rgb]{0.00,0.50,0.00}{##1}}}
\expandafter\def\csname PY@tok@bp\endcsname{\def\PY@tc##1{\textcolor[rgb]{0.00,0.50,0.00}{##1}}}
\expandafter\def\csname PY@tok@fm\endcsname{\def\PY@tc##1{\textcolor[rgb]{0.00,0.00,1.00}{##1}}}
\expandafter\def\csname PY@tok@vc\endcsname{\def\PY@tc##1{\textcolor[rgb]{0.10,0.09,0.49}{##1}}}
\expandafter\def\csname PY@tok@vg\endcsname{\def\PY@tc##1{\textcolor[rgb]{0.10,0.09,0.49}{##1}}}
\expandafter\def\csname PY@tok@vi\endcsname{\def\PY@tc##1{\textcolor[rgb]{0.10,0.09,0.49}{##1}}}
\expandafter\def\csname PY@tok@vm\endcsname{\def\PY@tc##1{\textcolor[rgb]{0.10,0.09,0.49}{##1}}}
\expandafter\def\csname PY@tok@sa\endcsname{\def\PY@tc##1{\textcolor[rgb]{0.73,0.13,0.13}{##1}}}
\expandafter\def\csname PY@tok@sb\endcsname{\def\PY@tc##1{\textcolor[rgb]{0.73,0.13,0.13}{##1}}}
\expandafter\def\csname PY@tok@sc\endcsname{\def\PY@tc##1{\textcolor[rgb]{0.73,0.13,0.13}{##1}}}
\expandafter\def\csname PY@tok@dl\endcsname{\def\PY@tc##1{\textcolor[rgb]{0.73,0.13,0.13}{##1}}}
\expandafter\def\csname PY@tok@s2\endcsname{\def\PY@tc##1{\textcolor[rgb]{0.73,0.13,0.13}{##1}}}
\expandafter\def\csname PY@tok@sh\endcsname{\def\PY@tc##1{\textcolor[rgb]{0.73,0.13,0.13}{##1}}}
\expandafter\def\csname PY@tok@s1\endcsname{\def\PY@tc##1{\textcolor[rgb]{0.73,0.13,0.13}{##1}}}
\expandafter\def\csname PY@tok@mb\endcsname{\def\PY@tc##1{\textcolor[rgb]{0.40,0.40,0.40}{##1}}}
\expandafter\def\csname PY@tok@mf\endcsname{\def\PY@tc##1{\textcolor[rgb]{0.40,0.40,0.40}{##1}}}
\expandafter\def\csname PY@tok@mh\endcsname{\def\PY@tc##1{\textcolor[rgb]{0.40,0.40,0.40}{##1}}}
\expandafter\def\csname PY@tok@mi\endcsname{\def\PY@tc##1{\textcolor[rgb]{0.40,0.40,0.40}{##1}}}
\expandafter\def\csname PY@tok@il\endcsname{\def\PY@tc##1{\textcolor[rgb]{0.40,0.40,0.40}{##1}}}
\expandafter\def\csname PY@tok@mo\endcsname{\def\PY@tc##1{\textcolor[rgb]{0.40,0.40,0.40}{##1}}}
\expandafter\def\csname PY@tok@ch\endcsname{\let\PY@it=\textit\def\PY@tc##1{\textcolor[rgb]{0.25,0.50,0.50}{##1}}}
\expandafter\def\csname PY@tok@cm\endcsname{\let\PY@it=\textit\def\PY@tc##1{\textcolor[rgb]{0.25,0.50,0.50}{##1}}}
\expandafter\def\csname PY@tok@cpf\endcsname{\let\PY@it=\textit\def\PY@tc##1{\textcolor[rgb]{0.25,0.50,0.50}{##1}}}
\expandafter\def\csname PY@tok@c1\endcsname{\let\PY@it=\textit\def\PY@tc##1{\textcolor[rgb]{0.25,0.50,0.50}{##1}}}
\expandafter\def\csname PY@tok@cs\endcsname{\let\PY@it=\textit\def\PY@tc##1{\textcolor[rgb]{0.25,0.50,0.50}{##1}}}

\def\PYZbs{\char`\\}
\def\PYZus{\char`\_}
\def\PYZob{\char`\{}
\def\PYZcb{\char`\}}
\def\PYZca{\char`\^}
\def\PYZam{\char`\&}
\def\PYZlt{\char`\<}
\def\PYZgt{\char`\>}
\def\PYZsh{\char`\#}
\def\PYZpc{\char`\%}
\def\PYZdl{\char`\$}
\def\PYZhy{\char`\-}
\def\PYZsq{\char`\'}
\def\PYZdq{\char`\"}
\def\PYZti{\char`\~}
% for compatibility with earlier versions
\def\PYZat{@}
\def\PYZlb{[}
\def\PYZrb{]}
\makeatother


    % Exact colors from NB
    \definecolor{incolor}{rgb}{0.0, 0.0, 0.5}
    \definecolor{outcolor}{rgb}{0.545, 0.0, 0.0}



    
    % Prevent overflowing lines due to hard-to-break entities
    \sloppy 
    % Setup hyperref package
    \hypersetup{
      breaklinks=true,  % so long urls are correctly broken across lines
      colorlinks=true,
      urlcolor=urlcolor,
      linkcolor=linkcolor,
      citecolor=citecolor,
      }
    % Slightly bigger margins than the latex defaults
    
    \geometry{verbose,tmargin=1in,bmargin=1in,lmargin=1in,rmargin=1in}
    
    

    \begin{document}
    
    
    \maketitle
    
    

    
    \section{2.1 Direct Solvers}\label{direct-solvers}

    \begin{Verbatim}[commandchars=\\\{\}]
{\color{incolor}In [{\color{incolor}1}]:} \PY{k+kn}{import} \PY{n+nn}{numpy} \PY{k}{as} \PY{n+nn}{np}
        \PY{k+kn}{import} \PY{n+nn}{scipy}\PY{n+nn}{.}\PY{n+nn}{linalg} \PY{k}{as} \PY{n+nn}{la}
\end{Verbatim}


    \subsection{\texorpdfstring{Linear systems of equations
\(Ax = b\)}{Linear systems of equations Ax = b}}\label{linear-systems-of-equations-ax-b}

\[\renewcommand{\R}{\mathbb{R}}\]

\[\begin{aligned}
    A = \left[\begin{array}{ccccc}
    a_{1,1} & a_{1,2} & \cdots & a_{1,n}\\
    a_{2,1} & a_{2,2} & \cdots & a_{2,n}\\
    \vdots  & \vdots  & \vdots & \vdots\\
    a_{n,1} & a_{n,2} & \cdots & a_{n,n}
    \end{array}\right],
    \quad x = \left[\begin{array}{cccc}
    x_1\\
    x_2\\
    \vdots\\
    x_n
    \end{array}\right],
    \quad b = \left[\begin{array}{ccccc}
    b_1\\
    b_2\\
    \vdots\\
    b_n
    \end{array}\right]\end{aligned}\]

\begin{itemize}
\tightlist
\item
  known \(A\in \R^{n,n}\) and \(b\in\R^n\)
\item
  \(A\) \emph{invertible}, i.e., inverse \(A^{-1}\in\R^{n,n}\) exists
\item
  unknown \(x\in \R^n\)
\end{itemize}

\begin{center}\rule{0.5\linewidth}{\linethickness}\end{center}

\[\begin{aligned}
    a_{1,1} x_1 + a_{1,2} x_2 + \cdots + a_{1,n} x_n & = b_1\\
    a_{2,1} x_1 + a_{2,2} x_2 + \cdots + a_{2,n} x_n & = b_2\\
    \vdots \qquad\qquad \quad &\\
    a_{n,1} x_1 + a_{n,2} x_2 + \cdots + a_{n,n} x_n & = b_n\end{aligned}\]

    \subsection{Questions}\label{questions}

\begin{itemize}
\tightlist
\item
  Find examples of invertible matrices
\item
  Find examples of not-invertible (singular) matrices
\item
  Discuss invertibility and solution of \(Ax=0\)
\item
  What is geometric interpretation of determinant \(\det(A)\) and
  application to systems of equations
\item
  Discuss systems \(Ax=b\) with non-square matrix \(A\)
\end{itemize}

\textbf{Think about the questions. Write down answers. Discuss with
colleagues. Assess your answers. Find your own questions.}

    \subsection{Applications}\label{applications}

\begin{itemize}
\item
  Linear systems appear in many applications:

  \begin{itemize}
  \item
    Regression by least squares method in statistics
  \item
    Linear programming in optimization
  \item
    Numerical solutions of ordinary differential equations
  \item
    Numerical solutions of partial differential equations
  \item
    Solving nonlinear equations by linearization
  \item
    \(\cdots\)
  \end{itemize}
\item
  The size of \(A x =b\) is usually huge in applications such that it is
  not possible to solve "by hand"
\item
  Need to develop algorithms to let computer do the job.
\end{itemize}

    \subsection{Algorithms}\label{algorithms}

\begin{itemize}
\item
  The algorithms for solving linear systems fall into two categories:

  \begin{itemize}
  \item
    Direct methods
  \item
    Iterative methods
  \end{itemize}
\item
  Direct methods produce (exact) solution using a finite number of
  arithmetic operations
\item
  Most common method: Gaussian elimination
\item
  Basic idea: Reduce system \(Ax=b\) to equivalent \(Ux=y\) where \(U\)
  is upper triangular
\item
  We will see: Gaussian elimination leads to matrix factorisation
  \(A=LU\) where \(L\) is lower triangular
\end{itemize}

\textbf{Question:} Write down examples of upper and lower triangular
matrices.

\begin{center}\rule{0.5\linewidth}{\linethickness}\end{center}

\begin{itemize}
\item
  If we know \(A=LU\), we can solve two systems to get solution of
  \(Ax=b\) \[Ly = b, \quad \text{and} \quad Ux = y\]
\item
  Gaussian elimination can break down and LU factorisation may not
  exist, use factorisation \[A = PLU\] with permutation matrix \(P\)
\item
  Alternative: QR factorisation \(A=QR\), with upper triangular \(R\)
  and orthogonal \(Q\)
\end{itemize}

\textbf{Questions:}

\begin{itemize}
\tightlist
\item
  Show that \(Ax=b\) if \(Ly=b\) and \(Ux=y\) and \(A=LU\)
\item
  How would you solve \(Qy=b\)?
\end{itemize}

\begin{center}\rule{0.5\linewidth}{\linethickness}\end{center}

    \begin{Verbatim}[commandchars=\\\{\}]
{\color{incolor}In [{\color{incolor}2}]:} \PY{c+c1}{\PYZsh{} Solving linear systems in Python}
        
        \PY{n}{A} \PY{o}{=} \PY{n}{np}\PY{o}{.}\PY{n}{array}\PY{p}{(}\PY{p}{[}\PY{p}{[}\PY{l+m+mi}{2}\PY{p}{,}\PY{l+m+mi}{3}\PY{p}{]}\PY{p}{,}\PY{p}{[}\PY{l+m+mi}{5}\PY{p}{,}\PY{l+m+mi}{9}\PY{p}{]}\PY{p}{]}\PY{p}{)}
        \PY{n+nb}{print}\PY{p}{(}\PY{l+s+s2}{\PYZdq{}}\PY{l+s+s2}{A = }\PY{l+s+s2}{\PYZdq{}}\PY{p}{,} \PY{n}{A}\PY{p}{,} \PY{l+s+s2}{\PYZdq{}}\PY{l+s+se}{\PYZbs{}n}\PY{l+s+s2}{\PYZdq{}}\PY{p}{)}
        
        \PY{n}{b} \PY{o}{=} \PY{n}{np}\PY{o}{.}\PY{n}{array}\PY{p}{(}\PY{p}{[}\PY{l+m+mi}{12}\PY{p}{,}\PY{l+m+mi}{33}\PY{p}{]}\PY{p}{)}
        \PY{n+nb}{print}\PY{p}{(}\PY{l+s+s2}{\PYZdq{}}\PY{l+s+s2}{b = }\PY{l+s+s2}{\PYZdq{}}\PY{p}{,} \PY{n}{b}\PY{p}{,} \PY{l+s+s2}{\PYZdq{}}\PY{l+s+se}{\PYZbs{}n}\PY{l+s+s2}{\PYZdq{}}\PY{p}{)}
        
        \PY{n}{x} \PY{o}{=} \PY{n}{la}\PY{o}{.}\PY{n}{solve}\PY{p}{(}\PY{n}{A}\PY{p}{,}\PY{n}{b}\PY{p}{)}
        \PY{n+nb}{print}\PY{p}{(}\PY{l+s+s2}{\PYZdq{}}\PY{l+s+s2}{x = }\PY{l+s+s2}{\PYZdq{}}\PY{p}{,} \PY{n}{x}\PY{p}{,} \PY{l+s+s2}{\PYZdq{}}\PY{l+s+se}{\PYZbs{}n}\PY{l+s+s2}{\PYZdq{}}\PY{p}{)}
        
        \PY{n}{np}\PY{o}{.}\PY{n}{allclose}\PY{p}{(}\PY{n}{np}\PY{o}{.}\PY{n}{dot}\PY{p}{(}\PY{n}{A}\PY{p}{,}\PY{n}{x}\PY{p}{)}\PY{p}{,} \PY{n}{b}\PY{p}{)}
\end{Verbatim}


    \begin{Verbatim}[commandchars=\\\{\}]
A =  [[2 3]
 [5 9]] 

b =  [12 33] 

x =  [3. 2.] 


    \end{Verbatim}

\begin{Verbatim}[commandchars=\\\{\}]
{\color{outcolor}Out[{\color{outcolor}2}]:} True
\end{Verbatim}
            
    \subsection{Question:}\label{question}

\emph{Check the documentation of numpy.linalg.solve. Can you find out
what method is used? The following code determines the LU factorisation
with partial pivoting. Try this out. Can you find examples where no LU
factorisation exists but where the code below till gives a solution?}

\begin{center}\rule{0.5\linewidth}{\linethickness}\end{center}

    \begin{Verbatim}[commandchars=\\\{\}]
{\color{incolor}In [{\color{incolor}3}]:} \PY{c+c1}{\PYZsh{} Solving linear systems of equations in Python}
        \PY{c+c1}{\PYZsh{} good for solving system with same A and different b}
        
        \PY{k+kn}{import} \PY{n+nn}{numpy} \PY{k}{as} \PY{n+nn}{np}
        \PY{k+kn}{import} \PY{n+nn}{scipy}\PY{n+nn}{.}\PY{n+nn}{linalg} \PY{k}{as} \PY{n+nn}{la}
        
        \PY{n}{A} \PY{o}{=} \PY{n}{np}\PY{o}{.}\PY{n}{array}\PY{p}{(}\PY{p}{[}\PY{p}{[}\PY{l+m+mi}{2}\PY{p}{,}\PY{l+m+mi}{3}\PY{p}{]}\PY{p}{,}\PY{p}{[}\PY{l+m+mi}{5}\PY{p}{,}\PY{l+m+mi}{9}\PY{p}{]}\PY{p}{]}\PY{p}{)}
        \PY{n}{b} \PY{o}{=} \PY{n}{np}\PY{o}{.}\PY{n}{array}\PY{p}{(}\PY{p}{[}\PY{l+m+mi}{12}\PY{p}{,}\PY{l+m+mi}{33}\PY{p}{]}\PY{p}{)}
        
        \PY{n}{lu}\PY{p}{,} \PY{n}{p} \PY{o}{=} \PY{n}{la}\PY{o}{.}\PY{n}{lu\PYZus{}factor}\PY{p}{(}\PY{n}{A}\PY{p}{)}
        \PY{n}{x} \PY{o}{=} \PY{n}{la}\PY{o}{.}\PY{n}{lu\PYZus{}solve}\PY{p}{(}\PY{p}{(}\PY{n}{lu}\PY{p}{,}\PY{n}{p}\PY{p}{)}\PY{p}{,} \PY{n}{b}\PY{p}{)}
        
        \PY{n}{np}\PY{o}{.}\PY{n}{allclose}\PY{p}{(}\PY{n}{np}\PY{o}{.}\PY{n}{dot}\PY{p}{(}\PY{n}{A}\PY{p}{,}\PY{n}{x}\PY{p}{)}\PY{p}{,} \PY{n}{b}\PY{p}{)}
\end{Verbatim}


\begin{Verbatim}[commandchars=\\\{\}]
{\color{outcolor}Out[{\color{outcolor}3}]:} True
\end{Verbatim}
            
    \subsection{Elimination by Elementary
Operations}\label{elimination-by-elementary-operations}

\begin{itemize}
\item
  We will use combinations of three types of elementary row operations:

  \begin{itemize}
  \item
    adding a multiple of one row to another row
  \item
    swapping two rows
  \item
    multiplying a row by a non-zero number
  \end{itemize}
\end{itemize}

Convert \(a=[a_1, a_2, a_3, a_4]^T\) with \(a_2\ne 0\) into
\([a_1, a_2, 0, 0]^T\) by

\begin{itemize}
\item
  multiplying the second row by \(-\frac{a_3}{a_2}\) and adding to the
  third row;
\item
  multiplying the second row by \(-\frac{a_4}{a_2}\) and adding to the
  fourth row.
\end{itemize}

\[\left[\begin{array}{cccc}
1 & 0 & 0 & 0\\
0 & 1 & 0 & 0\\
0 & {-\frac{a_3}{a_2}} & 1 & 0\\
0 & {-\frac{a_4}{a_2}} & 0 & 1
\end{array}\right]
\left[\begin{array}{cccc}
a_1\\
a_2\\
a_3\\
a_4
\end{array}\right]=\left[\begin{array}{cccc}
a_1\\
a_2\\
0\\
0
\end{array}\right]\]

\textbf{Question:} What are the matrices of the other two elementary
operations (row swapping and scaling)?

    \subsection{Elementary Matrices
(Multipliers)}\label{elementary-matrices-multipliers}

\begin{itemize}
\tightlist
\item
  vector \(a = [a_1, a_2, \cdots, a_n]^T\) with \(a_k\ne 0\)
\item
  action of elementary matrix \[E_k a=\left[\begin{array}{cccccc}
  1&\cdots&0&0&\cdots&0\\
  \vdots&\ddots&\vdots&\vdots&\ddots&\vdots\\
  0&\cdots&1&0&\cdots&0\\
  0&\cdots&{-m_{k+1}}&1&\cdots&0\\
  \vdots&&\vdots&\vdots&\ddots&\vdots\\
  0&\cdots&{-m_n}&0&\cdots&1
  \end{array}\right]\left[
  \begin{array}{c}a_1\\\vdots\\a_k\\a_{k+1}\\\vdots\\a_n\end{array}\right]
  =\left[\begin{array}{c}
  a_1\\\vdots\\a_k\\0\\\vdots\\0\end{array}\right]\]
\item
  \(E_k\) is designed to nullify all of the elements below \(a_k\) in
  \(a\), and \(m_j = \frac{a_j}{a_k}\)
\end{itemize}

    \subsection{Properties of Elementary
Matrices}\label{properties-of-elementary-matrices}

Let \(e_k\) denote the column vector with \(1\) on spot \(k\) and \(0\)
elsewhere.

\begin{enumerate}
\def\labelenumi{\arabic{enumi}.}
\item
  \(E_k\) is lower triangular with unit main diagonal.
\item
  \(E_k=I-m_k e_k^T\), where
  \(m_k=[0\ \cdots\ 0\ m_{k+1}\ \cdots\ m_n]^T\).
\item
  \(E_k^{-1}=I + m_ke_k^T\) ~(\(E_k^{-1}\) will be denoted by \(L_k\)).
\item
  If \(k < j\) then \(E_k E_j=I-m_ke_k^T-m_je_j^T\).
\item
  \(E_1E_2\cdots E_{n-1}=I-\sum_{k=1}^{n-1}m_ke_k^T\qquad\) -- lower
  triangular matrix.
\end{enumerate}

The first two items are obvious. For the third item, use \(e_k^T m_k=0\)
and \[\begin{aligned}
\left(I-m_k e_k^T\right) \left(I+ m_k e_k^T\right)
&= I -\left(m_k e_k^T\right)\left( m_k e_k^T\right)\\
&= I -m_k \left(e_k^T m_k\right) e_k^T = I.\end{aligned}\] The last two
items can be checked by using \(e_k^T m_j=0\) for \(k< j\).

    \subsection{Elementary Matrices -\/-
examples}\label{elementary-matrices----examples}

Given \(a = [ 2 \ \ \ 4 \ -\!2]^T\), we have \[\begin{aligned}
E_1 a &= \left[\begin{array}{rrr}
1&0&0\\
-2&1&0\\
1&0&1
\end{array}\right] \left[\begin{array}{r}2\\4\\-2\end{array}\right]
=\left[\begin{array}{r}2\\0\\0\end{array}\right], \\
E_2 a &= \left[\begin{array}{rrr}
1&0&0\\
0&1&0\\
0&\frac{1}{2}&1
\end{array}\right]\left[\begin{array}{r}2\\4\\-2\end{array}\right]
=\left[\begin{array}{r}2\\4\\0\end{array}\right], \\
L_1 &=E_1^{-1}=\left[\begin{array}{rrr}
1&0&0\\
2&1&0\\
-1&0&1
\end{array}\right],\\
L_2 &=E_2^{-1}=\left[\begin{array}{rrr}
1&0&0\\
0&1&0\\
0&-\frac{1}{2}&1
\end{array}\right] .\end{aligned}\]

    \subsection{LU Factorisation using Elementary
Matrices}\label{lu-factorisation-using-elementary-matrices}

\subsubsection{\texorpdfstring{Algorithm for matrix
\(A =(a_{i,j})\).}{Algorithm for matrix A =(a\_\{i,j\}).}}\label{algorithm-for-matrix-a-a_ij.}

\begin{enumerate}
\def\labelenumi{\arabic{enumi}.}
\tightlist
\item
  Gaussian elimination first column of \(A\) to make elements below
  \(a_{1,1}\) zero
\item
  Gaussian elimination on second column to make elements below
  \(a_{2,2}\) zero
\item
  continue this procedure until last column
\end{enumerate}

This leads to \(E_{n-1}\cdots E_2E_1 A=U\) and

\[\begin{aligned}
A & = & E_1^{-1}E_2^{-1}\cdots E_{n-1}^{-1}U\\
  & = &L_1L_2\cdots L_{n-1}U\\
  & = &LU.\end{aligned}\]

\begin{center}\rule{0.5\linewidth}{\linethickness}\end{center}

\(L\) can be obtained easily once \(E_1, \cdots, E_{n-1}\) are
available. Indeed \[\begin{aligned}
& E_k = \left[\begin{array}{cccccc}
1&\cdots&0&0&\cdots&0\\
\vdots&\ddots&\vdots&\vdots&\ddots&\vdots\\
0&\cdots&1&0&\cdots&0\\
0&\cdots&-l_{k+1, k}&1&\cdots&0\\
\vdots&&\vdots&\vdots&\ddots&\vdots\\
0&\cdots&-l_{n,k}&0&\cdots&1
\end{array}\right]\\
& \Longrightarrow L_k = E_k^{-1} =\left[\begin{array}{cccccc}
1&\cdots&0&0&\cdots&0\\
\vdots&\ddots&\vdots&\vdots&\ddots&\vdots\\
0&\cdots&1&0&\cdots&0\\
0&\cdots&l_{k+1,k}&1&\cdots&0\\
\vdots&&\vdots&\vdots&\ddots&\vdots\\
0&\cdots&l_{n,k}&0&\cdots&1
\end{array}\right]\end{aligned}\]

\begin{center}\rule{0.5\linewidth}{\linethickness}\end{center}

    \[\begin{aligned}
\Longrightarrow L = L_1 L_2 \cdots L_{n-1} = \left[\begin{array}{cccccc}
1&   &  &   &   \\
l_{2,1} & 1 &    & &\\
\vdots & l_{3,2} & \ddots &\\
\vdots       & \vdots      &    \ddots    &   1 &  \\
l_{n,1} & l_{n,2} & \cdots & l_{n,n-1} & 1
\end{array}\right]\end{aligned}\]

Hence \(L\) is a lower triangular matrix with unit main diagonal.

    \subsection{LU factorisation -\/-
example}\label{lu-factorisation----example}

Consider the matrix
\[A = \left[\begin{array}{rrr}2&4&-2\\4&9&-3\\-2&-3&7\end{array}\right].\]

We have \[\begin{aligned}
E_1 A &=\left[\begin{array}{rrr}
1&0&0\\
{-2} & 1 & 0\\
{1} &  0 & 1
\end{array} \right] \left[\begin{array}{rrr}2&4&-2\\4&9&-3\\-2&-3&7\end{array}
\right]=\left[\begin{array}{rrr}2&4&-2\\0&1&1\\0&1&5\end{array}
\right].\end{aligned}\]

\[\begin{aligned}
E_2 E_1 A&=\left[\begin{array}{rrr}
1 & 0 & 0\\
0 & 1 & 0\\
0 &{-1} & 1
\end{array} \right]\left[\begin{array}{rrr}2&4&-2\\0&1&1\\0&1&5\end{array}
\right]=\left[\begin{array}{rrr}2&4&-2\\0&1&1\\0&0&4\end{array}
\right] = U.\end{aligned}\]

\begin{center}\rule{0.5\linewidth}{\linethickness}\end{center}

Therefore, we obtain the LU Factorisation \(A = L U\), where
\[L=L_1 L_2 = E_1^{-1} E_2^{-1} =
\left[\begin{array}{rrr}1&0&0\\2&1&0\\-1&0&1\end{array}\right]
\left[\begin{array}{rrr}1&0&0\\0&1&0\\0&1&1\end{array}\right]
=\left[\begin{array}{rrr}
1 & 0 & 0\\
{2} & 1 & 0\\
{-1} & {1} & 1\end{array}\right].\]

    \subsection{Algorithm (LU factorisation -\/- pseudo code
version)}\label{algorithm-lu-factorisation----pseudo-code-version}

\begin{verbatim}
L = I
for k=1:n-1
  for i=k+1:n
    L(i,k) = A(i,k)/A(k,k)
    A(i,k) = 0.0
    for j=k+1:n
       A(i,j)=A(i,j)-L(i,k)*A(k,j)

U = A
\end{verbatim}

\begin{center}\rule{0.5\linewidth}{\linethickness}\end{center}

    \begin{Verbatim}[commandchars=\\\{\}]
{\color{incolor}In [{\color{incolor}3}]:} \PY{c+c1}{\PYZsh{}\PYZsh{} Algorithm (LU factorisation \PYZhy{}\PYZhy{} Python version)}
        \PY{c+c1}{\PYZsh{}\PYZsh{} deals with exact breakdown and non\PYZhy{}square A}
        
        \PY{k}{def} \PY{n+nf}{LU}\PY{p}{(}\PY{n}{A}\PY{p}{)}\PY{p}{:}
            \PY{p}{(}\PY{n}{n}\PY{p}{,}\PY{n}{m}\PY{p}{)} \PY{o}{=} \PY{n}{A}\PY{o}{.}\PY{n}{shape}
            \PY{n}{s} \PY{o}{=} \PY{n+nb}{min}\PY{p}{(}\PY{n}{n}\PY{p}{,}\PY{n}{m}\PY{p}{)}
            \PY{n}{L} \PY{o}{=} \PY{n}{np}\PY{o}{.}\PY{n}{eye}\PY{p}{(}\PY{n}{n}\PY{p}{)}
            \PY{n}{U} \PY{o}{=} \PY{n}{A}\PY{o}{.}\PY{n}{copy}\PY{p}{(}\PY{p}{)}
            
            \PY{k}{for} \PY{n}{k} \PY{o+ow}{in} \PY{n+nb}{range}\PY{p}{(}\PY{n}{s}\PY{o}{\PYZhy{}}\PY{l+m+mi}{1}\PY{p}{)}\PY{p}{:}
                \PY{k}{if} \PY{p}{(}\PY{n}{U}\PY{p}{[}\PY{n}{k}\PY{p}{,}\PY{n}{k}\PY{p}{]}\PY{o}{!=}\PY{l+m+mi}{0}\PY{p}{)}\PY{p}{:}
                    \PY{n}{L}\PY{p}{[}\PY{n}{k}\PY{o}{+}\PY{l+m+mi}{1}\PY{p}{:}\PY{p}{,}\PY{n}{k}\PY{p}{]} \PY{o}{=} \PY{n}{U}\PY{p}{[}\PY{n}{k}\PY{o}{+}\PY{l+m+mi}{1}\PY{p}{:}\PY{p}{,}\PY{n}{k}\PY{p}{]}\PY{o}{/}\PY{n}{U}\PY{p}{[}\PY{n}{k}\PY{p}{,}\PY{n}{k}\PY{p}{]}  \PY{c+c1}{\PYZsh{} multipliers}
                \PY{k}{elif} \PY{p}{(}\PY{n}{np}\PY{o}{.}\PY{n}{sum}\PY{p}{(}\PY{n+nb}{abs}\PY{p}{(}\PY{n}{U}\PY{p}{[}\PY{n}{k}\PY{o}{+}\PY{l+m+mi}{1}\PY{p}{:}\PY{p}{,}\PY{n}{k}\PY{p}{]}\PY{p}{)}\PY{p}{)} \PY{o}{!=} \PY{l+m+mi}{0}\PY{p}{)}\PY{p}{:} \PY{c+c1}{\PYZsh{} zero pivot}
                    \PY{k}{raise} \PY{n+ne}{RuntimeError}\PY{p}{(}\PY{l+s+s1}{\PYZsq{}}\PY{l+s+s1}{LU breakdown}\PY{l+s+s1}{\PYZsq{}}\PY{p}{)}
                \PY{n}{U}\PY{p}{[}\PY{n}{k}\PY{o}{+}\PY{l+m+mi}{1}\PY{p}{:}\PY{p}{,}\PY{n}{k}\PY{o}{+}\PY{l+m+mi}{1}\PY{p}{:}\PY{p}{]} \PY{o}{\PYZhy{}}\PY{o}{=} \PY{n}{np}\PY{o}{.}\PY{n}{outer}\PY{p}{(}\PY{n}{L}\PY{p}{[}\PY{n}{k}\PY{o}{+}\PY{l+m+mi}{1}\PY{p}{:}\PY{p}{,}\PY{n}{k}\PY{p}{]}\PY{p}{,}\PY{n}{U}\PY{p}{[}\PY{n}{k}\PY{p}{,}\PY{n}{k}\PY{o}{+}\PY{l+m+mi}{1}\PY{p}{:}\PY{p}{]}\PY{p}{)}
                \PY{n}{U}\PY{p}{[}\PY{n}{k}\PY{o}{+}\PY{l+m+mi}{1}\PY{p}{:}\PY{p}{,}\PY{n}{k}\PY{p}{]} \PY{o}{=} \PY{l+m+mi}{0}
            
            \PY{k}{return} \PY{n}{L}\PY{p}{,} \PY{n}{U}
\end{Verbatim}


    \begin{center}\rule{0.5\linewidth}{\linethickness}\end{center}

    \begin{Verbatim}[commandchars=\\\{\}]
{\color{incolor}In [{\color{incolor}4}]:} \PY{n}{A} \PY{o}{=} \PY{n}{np}\PY{o}{.}\PY{n}{array}\PY{p}{(}\PY{p}{(}\PY{p}{(}\PY{l+m+mf}{3.0}\PY{p}{,} \PY{l+m+mf}{4.0}\PY{p}{,} \PY{l+m+mi}{6}\PY{p}{)}\PY{p}{,} \PY{p}{(}\PY{l+m+mf}{3.0}\PY{p}{,} \PY{l+m+mf}{5.0}\PY{p}{,} \PY{l+m+mi}{2}\PY{p}{)}\PY{p}{,} \PY{p}{(}\PY{l+m+mi}{3}\PY{p}{,}\PY{l+m+mi}{1}\PY{p}{,}\PY{l+m+mi}{4}\PY{p}{)}\PY{p}{)}\PY{p}{)}
        \PY{c+c1}{\PYZsh{}A = np.array(((0.0, 1.0), (1.0, 0.0)))  \PYZsh{} implement partial pivoting ...}
        \PY{c+c1}{\PYZsh{}A = np.array(((0.0, 0.0), (0.0, 0.0)))  \PYZsh{} no elimination required for Aik = 0}
        \PY{c+c1}{\PYZsh{}A = np.ones((3,2))}
        
        \PY{n+nb}{print}\PY{p}{(}\PY{n}{A}\PY{p}{)}
        \PY{n}{L}\PY{p}{,} \PY{n}{U} \PY{o}{=} \PY{n}{LU}\PY{p}{(}\PY{n}{A}\PY{o}{.}\PY{n}{copy}\PY{p}{(}\PY{p}{)}\PY{p}{)}
        \PY{n+nb}{print}\PY{p}{(}\PY{n}{L}\PY{p}{)}
        \PY{n+nb}{print}\PY{p}{(}\PY{n}{U}\PY{p}{)}
\end{Verbatim}


    \begin{Verbatim}[commandchars=\\\{\}]
[[3. 4. 6.]
 [3. 5. 2.]
 [3. 1. 4.]]
[[ 1.  0.  0.]
 [ 1.  1.  0.]
 [ 1. -3.  1.]]
[[  3.   4.   6.]
 [  0.   1.  -4.]
 [  0.   0. -14.]]

    \end{Verbatim}

    \subsection{Existence and Uniqueness}\label{existence-and-uniqueness}

\begin{itemize}
\item
  \(\left[\begin{array}{ccc} 0 & 1\\ 1 & 0 \end{array}\right]\) does not
  have an LU factorisation
\item
  Let \(A^r\) be \(r \times r\) submatrix with first \(r\) rows and
  columns of \(A\).
\item
  The \(r\)-th \textbf{principle minor} of \(A\) is the determinant
  \(\det(A^r)\).
\item
  \textbf{existence of \(LU\) factorisation}: if first \(n-1\) principle
  subminors of \(A\) do not vanish then the LU factorisation \(A=LU\)
  exists and is unique
\item
  Every symmetric positive definite matrix has LU factorisation.
\item
  Proof: By induction or by constructing the elimination process and
  observing that the process may be continued as long as the pivots are
  non-zero. The pivots will be non-zero since the product of the first
  \(j\) pivots is equal to \(\det(A^j) \ne 0\).
\end{itemize}

    \subsection{QR decomposition}\label{qr-decomposition}

\begin{itemize}
\item
  The QR decomposition also uses an elimination process and (in this
  case orthogonal) elmentary matrices. The most commonly used matrices
  are reflections or Householder matrices of the form
  \(H = I - 2 u u^T\) where \(u\) have length one.
\item
  Rotation (Givens or Jacobi) matrices are also used and even the
  Gram-Schmidt process can be implemented numerically.
\end{itemize}

    \subsection{Solution using the LU
factorisation}\label{solution-using-the-lu-factorisation}

\begin{itemize}
\item
  Let \(A=LU\) with \[L = \left[\begin{array}{cccccc}
  l_{1,1} &   &  &     \\
  l_{2,1} & l_{2,2} &     &\\
  \vdots       & \vdots      &    \ddots      &\\
  l_{n,1} & l_{n,2} & \cdots  & l_{n,n}
  \end{array}\right], \quad
  U = \left[ \begin{array}{cccccc}
  u_{1,1} & u_{1,2} & \cdots & u_{1,n} \\
      & u_{2,2} & \cdots & u_{2,n} \\
      &         & \ddots &  \vdots \\
      &         &        & u_{n,n}
  \end{array}\right].\]
\item
  Solving \(Ax=b\) in 2 steps:

  \begin{enumerate}
  \def\labelenumi{\arabic{enumi}.}
  \item
    Solve the lower triangular system \(Ly =b\) for \(y\) by forward
    substitution.
  \item
    Solve the upper triangular system \(U x = y\) for \(x\) by back
    substitution.
  \end{enumerate}
\end{itemize}

    \subsection{Forward Substitution}\label{forward-substitution}

The lower triangular system \(L y =b\) takes the form \[\begin{aligned}
\begin{array}{ccccccc}
l_{1,1}y_1 & &           &         &          & =&  b_1    \\
l_{2,1}y_1 &+&l_{2,2}y_2    &         &          & =&  b_2    \\
\vdots     & &\vdots     & \ddots  &          & =&  \vdots \\
l_{n,1}y_1 &+&l_{n,2}y_2 &+ \ldots +&l_{n,n}y_n&    =& b_n
\end{array}\end{aligned}\] By the first equation we can obtain
\[y_1 = \frac{b_1}{l_{1,1}}.\] Inductively, the solution is given by
\[y_i = {{b_i - \sum_{k=1}^{i-1} l_{i,k} y_k } \over l_{i,i}}, \quad
i = 1, 2, \ldots , n.\]

    \subsection{Backward Substitution}\label{backward-substitution}

The upper triangular system \(Ux = y\) takes the form
\[\begin{array}{ccccccc}
u_{1,1}x_1 &+&u_{1,2}x_2  &+ \ldots +&u_{1,n}x_n  &=  &y_1 \\
           & &u_{2,2}x_2  &+ \ldots +&u_{2,n}x_n  &=  &y_2 \\
           & &            & \ddots   & \vdots   &\vdots &\vdots \\
           & &            &          &  u_{n,n}x_n&=   &y_n
\end{array}\] By the last equation we obtain
\[x_n = { y_n \over u_{n,n}}.\] Inductively, the solution \(x\) is given
by \[x_i = {{y_i - \sum_{k=i+1}^n u_{i,k} x_k } \over u_{i,i}}, \quad
 i = n, n-1,..., 1.\]

    \subsection{\texorpdfstring{Solving \(Ly = b\) while computing
\(LU=A\)}{Solving Ly = b while computing LU=A}}\label{solving-ly-b-while-computing-lua}

\begin{itemize}
\tightlist
\item
  forward substitution can be done simultaneously as LU factorisation:

  \begin{itemize}
  \tightlist
  \item
    factorise the augmented matrix \([A\ \ b]\)
    \[\begin{aligned} E_{n-1}\dots E_1 [A \ \ b] &= 
    [E_{n-1} \dots E_1 A \ \ E_{n-1}\dots E_1 b] \\
    &= [U \ \ L^{-1} b] \\ &= [U \ \ y].\end{aligned}\]
  \end{itemize}
\end{itemize}

    \subsection{Example}\label{example}

Consider the linear system \(A x =b\) where
\[A = \left[\begin{array}{rrrr}
1  &  3  & 1 \\
1  & -2  & -1\\
2  & 1  & 2
\end{array}\right], \quad
b = \left[ \begin{array}{rrrr}
10\\ -6 \\ 10\\
\end{array}\right].\] We work on the augmented matrix \([A \ \ b]\).
Then \[\begin{aligned}
E_1 [A \ \ b]\! &= \!\!\left[\begin{array}{rrrr}
1 & 0 & 0\\
-1 & 1 & 0 \\
-2 & 0 & 1
\end{array}\right] \left[\begin{array}{rrrr}
1  &  3  & 1 & 10\\
1  & -2  & -1 & -6\\
2  & 1  & 2  &  10
\end{array}\right]
\!\!=\!\! \left[\begin{array}{rrrr}
1  &  3  & 1 & 10\\
0  & -5  & -2 & -16\\
0  & -5  &  0 & -10
\end{array}\right],\end{aligned}\]

\begin{center}\rule{0.5\linewidth}{\linethickness}\end{center}

\[\begin{aligned}
E_2 E_1 [A \ \ b] & = \left[\begin{array}{rrrr}
1 & 0 & 0\\
0 & 1 & 0 \\
0 & -1 & 1
\end{array}\right] \left[\begin{array}{rrrr}
1  &  3  & 1 & 10\\
0  & -5  & -2 & -16\\
0  & -5  & 0  &  -10
\end{array}\right]\\
&= \left[\begin{array}{rrrr}
1  &  3  & 1 & 10\\
0  & -5  & -2 & -16\\
0  & 0  &  2 & 6
\end{array}\right].\end{aligned}\]

By back substitution we obtain the solution

\[\begin{bmatrix} x_1\\x_2\\x_3 \end{bmatrix} = \begin{bmatrix} 1\\2\\3 \end{bmatrix}.\]

\subsection{Algorithm (Solve linear system by LU
factorisation)}\label{algorithm-solve-linear-system-by-lu-factorisation}

\begin{verbatim}
M = [A  b]
for k=1:n-1
  for i=k+1:n
     q = M(i,k)/M(k,k)
     for j = k:n+1
        M(i,j) = M(i, j) - q*M(k,j)
     end
  end
end
x(n) = M(n,n+1)/M(n,n)
for i = n-1:-1:1
   z = 0
   for j = i+1:n
      z = z + M(i,j)*x(j)
   end
   x(i) = (M(i,n+1)-z)/M(i,i)
end
\end{verbatim}

\subsection{Flops - Gaussian
Elimination}\label{flops---gaussian-elimination}

The most expensive part of the algorithm on LU factorisation involves
the row operations which can be written as three nested for loops

\begin{verbatim}
L = I
for k=1:n-1
  for i=k+1:n
    L(i,k) = A(i,k)/A(k,k)
    A(i,k) = 0.0
    for j=k+1:n
       A(i,j)=A(i,j)-L(i,k)*A(k,j)

U = A
\end{verbatim}


    % Add a bibliography block to the postdoc
    
    
    
    \end{document}
