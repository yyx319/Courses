
% Default to the notebook output style

    


% Inherit from the specified cell style.




    
\documentclass[11pt]{article}

    
    
    \usepackage[T1]{fontenc}
    % Nicer default font (+ math font) than Computer Modern for most use cases
    \usepackage{mathpazo}

    % Basic figure setup, for now with no caption control since it's done
    % automatically by Pandoc (which extracts ![](path) syntax from Markdown).
    \usepackage{graphicx}
    % We will generate all images so they have a width \maxwidth. This means
    % that they will get their normal width if they fit onto the page, but
    % are scaled down if they would overflow the margins.
    \makeatletter
    \def\maxwidth{\ifdim\Gin@nat@width>\linewidth\linewidth
    \else\Gin@nat@width\fi}
    \makeatother
    \let\Oldincludegraphics\includegraphics
    % Set max figure width to be 80% of text width, for now hardcoded.
    \renewcommand{\includegraphics}[1]{\Oldincludegraphics[width=.8\maxwidth]{#1}}
    % Ensure that by default, figures have no caption (until we provide a
    % proper Figure object with a Caption API and a way to capture that
    % in the conversion process - todo).
    \usepackage{caption}
    \DeclareCaptionLabelFormat{nolabel}{}
    \captionsetup{labelformat=nolabel}

    \usepackage{adjustbox} % Used to constrain images to a maximum size 
    \usepackage{xcolor} % Allow colors to be defined
    \usepackage{enumerate} % Needed for markdown enumerations to work
    \usepackage{geometry} % Used to adjust the document margins
    \usepackage{amsmath} % Equations
    \usepackage{amssymb} % Equations
    \usepackage{textcomp} % defines textquotesingle
    % Hack from http://tex.stackexchange.com/a/47451/13684:
    \AtBeginDocument{%
        \def\PYZsq{\textquotesingle}% Upright quotes in Pygmentized code
    }
    \usepackage{upquote} % Upright quotes for verbatim code
    \usepackage{eurosym} % defines \euro
    \usepackage[mathletters]{ucs} % Extended unicode (utf-8) support
    \usepackage[utf8x]{inputenc} % Allow utf-8 characters in the tex document
    \usepackage{fancyvrb} % verbatim replacement that allows latex
    \usepackage{grffile} % extends the file name processing of package graphics 
                         % to support a larger range 
    % The hyperref package gives us a pdf with properly built
    % internal navigation ('pdf bookmarks' for the table of contents,
    % internal cross-reference links, web links for URLs, etc.)
    \usepackage{hyperref}
    \usepackage{longtable} % longtable support required by pandoc >1.10
    \usepackage{booktabs}  % table support for pandoc > 1.12.2
    \usepackage[inline]{enumitem} % IRkernel/repr support (it uses the enumerate* environment)
    \usepackage[normalem]{ulem} % ulem is needed to support strikethroughs (\sout)
                                % normalem makes italics be italics, not underlines
    

    
    
    % Colors for the hyperref package
    \definecolor{urlcolor}{rgb}{0,.145,.698}
    \definecolor{linkcolor}{rgb}{.71,0.21,0.01}
    \definecolor{citecolor}{rgb}{.12,.54,.11}

    % ANSI colors
    \definecolor{ansi-black}{HTML}{3E424D}
    \definecolor{ansi-black-intense}{HTML}{282C36}
    \definecolor{ansi-red}{HTML}{E75C58}
    \definecolor{ansi-red-intense}{HTML}{B22B31}
    \definecolor{ansi-green}{HTML}{00A250}
    \definecolor{ansi-green-intense}{HTML}{007427}
    \definecolor{ansi-yellow}{HTML}{DDB62B}
    \definecolor{ansi-yellow-intense}{HTML}{B27D12}
    \definecolor{ansi-blue}{HTML}{208FFB}
    \definecolor{ansi-blue-intense}{HTML}{0065CA}
    \definecolor{ansi-magenta}{HTML}{D160C4}
    \definecolor{ansi-magenta-intense}{HTML}{A03196}
    \definecolor{ansi-cyan}{HTML}{60C6C8}
    \definecolor{ansi-cyan-intense}{HTML}{258F8F}
    \definecolor{ansi-white}{HTML}{C5C1B4}
    \definecolor{ansi-white-intense}{HTML}{A1A6B2}

    % commands and environments needed by pandoc snippets
    % extracted from the output of `pandoc -s`
    \providecommand{\tightlist}{%
      \setlength{\itemsep}{0pt}\setlength{\parskip}{0pt}}
    \DefineVerbatimEnvironment{Highlighting}{Verbatim}{commandchars=\\\{\}}
    % Add ',fontsize=\small' for more characters per line
    \newenvironment{Shaded}{}{}
    \newcommand{\KeywordTok}[1]{\textcolor[rgb]{0.00,0.44,0.13}{\textbf{{#1}}}}
    \newcommand{\DataTypeTok}[1]{\textcolor[rgb]{0.56,0.13,0.00}{{#1}}}
    \newcommand{\DecValTok}[1]{\textcolor[rgb]{0.25,0.63,0.44}{{#1}}}
    \newcommand{\BaseNTok}[1]{\textcolor[rgb]{0.25,0.63,0.44}{{#1}}}
    \newcommand{\FloatTok}[1]{\textcolor[rgb]{0.25,0.63,0.44}{{#1}}}
    \newcommand{\CharTok}[1]{\textcolor[rgb]{0.25,0.44,0.63}{{#1}}}
    \newcommand{\StringTok}[1]{\textcolor[rgb]{0.25,0.44,0.63}{{#1}}}
    \newcommand{\CommentTok}[1]{\textcolor[rgb]{0.38,0.63,0.69}{\textit{{#1}}}}
    \newcommand{\OtherTok}[1]{\textcolor[rgb]{0.00,0.44,0.13}{{#1}}}
    \newcommand{\AlertTok}[1]{\textcolor[rgb]{1.00,0.00,0.00}{\textbf{{#1}}}}
    \newcommand{\FunctionTok}[1]{\textcolor[rgb]{0.02,0.16,0.49}{{#1}}}
    \newcommand{\RegionMarkerTok}[1]{{#1}}
    \newcommand{\ErrorTok}[1]{\textcolor[rgb]{1.00,0.00,0.00}{\textbf{{#1}}}}
    \newcommand{\NormalTok}[1]{{#1}}
    
    % Additional commands for more recent versions of Pandoc
    \newcommand{\ConstantTok}[1]{\textcolor[rgb]{0.53,0.00,0.00}{{#1}}}
    \newcommand{\SpecialCharTok}[1]{\textcolor[rgb]{0.25,0.44,0.63}{{#1}}}
    \newcommand{\VerbatimStringTok}[1]{\textcolor[rgb]{0.25,0.44,0.63}{{#1}}}
    \newcommand{\SpecialStringTok}[1]{\textcolor[rgb]{0.73,0.40,0.53}{{#1}}}
    \newcommand{\ImportTok}[1]{{#1}}
    \newcommand{\DocumentationTok}[1]{\textcolor[rgb]{0.73,0.13,0.13}{\textit{{#1}}}}
    \newcommand{\AnnotationTok}[1]{\textcolor[rgb]{0.38,0.63,0.69}{\textbf{\textit{{#1}}}}}
    \newcommand{\CommentVarTok}[1]{\textcolor[rgb]{0.38,0.63,0.69}{\textbf{\textit{{#1}}}}}
    \newcommand{\VariableTok}[1]{\textcolor[rgb]{0.10,0.09,0.49}{{#1}}}
    \newcommand{\ControlFlowTok}[1]{\textcolor[rgb]{0.00,0.44,0.13}{\textbf{{#1}}}}
    \newcommand{\OperatorTok}[1]{\textcolor[rgb]{0.40,0.40,0.40}{{#1}}}
    \newcommand{\BuiltInTok}[1]{{#1}}
    \newcommand{\ExtensionTok}[1]{{#1}}
    \newcommand{\PreprocessorTok}[1]{\textcolor[rgb]{0.74,0.48,0.00}{{#1}}}
    \newcommand{\AttributeTok}[1]{\textcolor[rgb]{0.49,0.56,0.16}{{#1}}}
    \newcommand{\InformationTok}[1]{\textcolor[rgb]{0.38,0.63,0.69}{\textbf{\textit{{#1}}}}}
    \newcommand{\WarningTok}[1]{\textcolor[rgb]{0.38,0.63,0.69}{\textbf{\textit{{#1}}}}}
    
    
    % Define a nice break command that doesn't care if a line doesn't already
    % exist.
    \def\br{\hspace*{\fill} \\* }
    % Math Jax compatability definitions
    \def\gt{>}
    \def\lt{<}
    % Document parameters
    \title{Error}
    
    
    

    % Pygments definitions
    
\makeatletter
\def\PY@reset{\let\PY@it=\relax \let\PY@bf=\relax%
    \let\PY@ul=\relax \let\PY@tc=\relax%
    \let\PY@bc=\relax \let\PY@ff=\relax}
\def\PY@tok#1{\csname PY@tok@#1\endcsname}
\def\PY@toks#1+{\ifx\relax#1\empty\else%
    \PY@tok{#1}\expandafter\PY@toks\fi}
\def\PY@do#1{\PY@bc{\PY@tc{\PY@ul{%
    \PY@it{\PY@bf{\PY@ff{#1}}}}}}}
\def\PY#1#2{\PY@reset\PY@toks#1+\relax+\PY@do{#2}}

\expandafter\def\csname PY@tok@w\endcsname{\def\PY@tc##1{\textcolor[rgb]{0.73,0.73,0.73}{##1}}}
\expandafter\def\csname PY@tok@c\endcsname{\let\PY@it=\textit\def\PY@tc##1{\textcolor[rgb]{0.25,0.50,0.50}{##1}}}
\expandafter\def\csname PY@tok@cp\endcsname{\def\PY@tc##1{\textcolor[rgb]{0.74,0.48,0.00}{##1}}}
\expandafter\def\csname PY@tok@k\endcsname{\let\PY@bf=\textbf\def\PY@tc##1{\textcolor[rgb]{0.00,0.50,0.00}{##1}}}
\expandafter\def\csname PY@tok@kp\endcsname{\def\PY@tc##1{\textcolor[rgb]{0.00,0.50,0.00}{##1}}}
\expandafter\def\csname PY@tok@kt\endcsname{\def\PY@tc##1{\textcolor[rgb]{0.69,0.00,0.25}{##1}}}
\expandafter\def\csname PY@tok@o\endcsname{\def\PY@tc##1{\textcolor[rgb]{0.40,0.40,0.40}{##1}}}
\expandafter\def\csname PY@tok@ow\endcsname{\let\PY@bf=\textbf\def\PY@tc##1{\textcolor[rgb]{0.67,0.13,1.00}{##1}}}
\expandafter\def\csname PY@tok@nb\endcsname{\def\PY@tc##1{\textcolor[rgb]{0.00,0.50,0.00}{##1}}}
\expandafter\def\csname PY@tok@nf\endcsname{\def\PY@tc##1{\textcolor[rgb]{0.00,0.00,1.00}{##1}}}
\expandafter\def\csname PY@tok@nc\endcsname{\let\PY@bf=\textbf\def\PY@tc##1{\textcolor[rgb]{0.00,0.00,1.00}{##1}}}
\expandafter\def\csname PY@tok@nn\endcsname{\let\PY@bf=\textbf\def\PY@tc##1{\textcolor[rgb]{0.00,0.00,1.00}{##1}}}
\expandafter\def\csname PY@tok@ne\endcsname{\let\PY@bf=\textbf\def\PY@tc##1{\textcolor[rgb]{0.82,0.25,0.23}{##1}}}
\expandafter\def\csname PY@tok@nv\endcsname{\def\PY@tc##1{\textcolor[rgb]{0.10,0.09,0.49}{##1}}}
\expandafter\def\csname PY@tok@no\endcsname{\def\PY@tc##1{\textcolor[rgb]{0.53,0.00,0.00}{##1}}}
\expandafter\def\csname PY@tok@nl\endcsname{\def\PY@tc##1{\textcolor[rgb]{0.63,0.63,0.00}{##1}}}
\expandafter\def\csname PY@tok@ni\endcsname{\let\PY@bf=\textbf\def\PY@tc##1{\textcolor[rgb]{0.60,0.60,0.60}{##1}}}
\expandafter\def\csname PY@tok@na\endcsname{\def\PY@tc##1{\textcolor[rgb]{0.49,0.56,0.16}{##1}}}
\expandafter\def\csname PY@tok@nt\endcsname{\let\PY@bf=\textbf\def\PY@tc##1{\textcolor[rgb]{0.00,0.50,0.00}{##1}}}
\expandafter\def\csname PY@tok@nd\endcsname{\def\PY@tc##1{\textcolor[rgb]{0.67,0.13,1.00}{##1}}}
\expandafter\def\csname PY@tok@s\endcsname{\def\PY@tc##1{\textcolor[rgb]{0.73,0.13,0.13}{##1}}}
\expandafter\def\csname PY@tok@sd\endcsname{\let\PY@it=\textit\def\PY@tc##1{\textcolor[rgb]{0.73,0.13,0.13}{##1}}}
\expandafter\def\csname PY@tok@si\endcsname{\let\PY@bf=\textbf\def\PY@tc##1{\textcolor[rgb]{0.73,0.40,0.53}{##1}}}
\expandafter\def\csname PY@tok@se\endcsname{\let\PY@bf=\textbf\def\PY@tc##1{\textcolor[rgb]{0.73,0.40,0.13}{##1}}}
\expandafter\def\csname PY@tok@sr\endcsname{\def\PY@tc##1{\textcolor[rgb]{0.73,0.40,0.53}{##1}}}
\expandafter\def\csname PY@tok@ss\endcsname{\def\PY@tc##1{\textcolor[rgb]{0.10,0.09,0.49}{##1}}}
\expandafter\def\csname PY@tok@sx\endcsname{\def\PY@tc##1{\textcolor[rgb]{0.00,0.50,0.00}{##1}}}
\expandafter\def\csname PY@tok@m\endcsname{\def\PY@tc##1{\textcolor[rgb]{0.40,0.40,0.40}{##1}}}
\expandafter\def\csname PY@tok@gh\endcsname{\let\PY@bf=\textbf\def\PY@tc##1{\textcolor[rgb]{0.00,0.00,0.50}{##1}}}
\expandafter\def\csname PY@tok@gu\endcsname{\let\PY@bf=\textbf\def\PY@tc##1{\textcolor[rgb]{0.50,0.00,0.50}{##1}}}
\expandafter\def\csname PY@tok@gd\endcsname{\def\PY@tc##1{\textcolor[rgb]{0.63,0.00,0.00}{##1}}}
\expandafter\def\csname PY@tok@gi\endcsname{\def\PY@tc##1{\textcolor[rgb]{0.00,0.63,0.00}{##1}}}
\expandafter\def\csname PY@tok@gr\endcsname{\def\PY@tc##1{\textcolor[rgb]{1.00,0.00,0.00}{##1}}}
\expandafter\def\csname PY@tok@ge\endcsname{\let\PY@it=\textit}
\expandafter\def\csname PY@tok@gs\endcsname{\let\PY@bf=\textbf}
\expandafter\def\csname PY@tok@gp\endcsname{\let\PY@bf=\textbf\def\PY@tc##1{\textcolor[rgb]{0.00,0.00,0.50}{##1}}}
\expandafter\def\csname PY@tok@go\endcsname{\def\PY@tc##1{\textcolor[rgb]{0.53,0.53,0.53}{##1}}}
\expandafter\def\csname PY@tok@gt\endcsname{\def\PY@tc##1{\textcolor[rgb]{0.00,0.27,0.87}{##1}}}
\expandafter\def\csname PY@tok@err\endcsname{\def\PY@bc##1{\setlength{\fboxsep}{0pt}\fcolorbox[rgb]{1.00,0.00,0.00}{1,1,1}{\strut ##1}}}
\expandafter\def\csname PY@tok@kc\endcsname{\let\PY@bf=\textbf\def\PY@tc##1{\textcolor[rgb]{0.00,0.50,0.00}{##1}}}
\expandafter\def\csname PY@tok@kd\endcsname{\let\PY@bf=\textbf\def\PY@tc##1{\textcolor[rgb]{0.00,0.50,0.00}{##1}}}
\expandafter\def\csname PY@tok@kn\endcsname{\let\PY@bf=\textbf\def\PY@tc##1{\textcolor[rgb]{0.00,0.50,0.00}{##1}}}
\expandafter\def\csname PY@tok@kr\endcsname{\let\PY@bf=\textbf\def\PY@tc##1{\textcolor[rgb]{0.00,0.50,0.00}{##1}}}
\expandafter\def\csname PY@tok@bp\endcsname{\def\PY@tc##1{\textcolor[rgb]{0.00,0.50,0.00}{##1}}}
\expandafter\def\csname PY@tok@fm\endcsname{\def\PY@tc##1{\textcolor[rgb]{0.00,0.00,1.00}{##1}}}
\expandafter\def\csname PY@tok@vc\endcsname{\def\PY@tc##1{\textcolor[rgb]{0.10,0.09,0.49}{##1}}}
\expandafter\def\csname PY@tok@vg\endcsname{\def\PY@tc##1{\textcolor[rgb]{0.10,0.09,0.49}{##1}}}
\expandafter\def\csname PY@tok@vi\endcsname{\def\PY@tc##1{\textcolor[rgb]{0.10,0.09,0.49}{##1}}}
\expandafter\def\csname PY@tok@vm\endcsname{\def\PY@tc##1{\textcolor[rgb]{0.10,0.09,0.49}{##1}}}
\expandafter\def\csname PY@tok@sa\endcsname{\def\PY@tc##1{\textcolor[rgb]{0.73,0.13,0.13}{##1}}}
\expandafter\def\csname PY@tok@sb\endcsname{\def\PY@tc##1{\textcolor[rgb]{0.73,0.13,0.13}{##1}}}
\expandafter\def\csname PY@tok@sc\endcsname{\def\PY@tc##1{\textcolor[rgb]{0.73,0.13,0.13}{##1}}}
\expandafter\def\csname PY@tok@dl\endcsname{\def\PY@tc##1{\textcolor[rgb]{0.73,0.13,0.13}{##1}}}
\expandafter\def\csname PY@tok@s2\endcsname{\def\PY@tc##1{\textcolor[rgb]{0.73,0.13,0.13}{##1}}}
\expandafter\def\csname PY@tok@sh\endcsname{\def\PY@tc##1{\textcolor[rgb]{0.73,0.13,0.13}{##1}}}
\expandafter\def\csname PY@tok@s1\endcsname{\def\PY@tc##1{\textcolor[rgb]{0.73,0.13,0.13}{##1}}}
\expandafter\def\csname PY@tok@mb\endcsname{\def\PY@tc##1{\textcolor[rgb]{0.40,0.40,0.40}{##1}}}
\expandafter\def\csname PY@tok@mf\endcsname{\def\PY@tc##1{\textcolor[rgb]{0.40,0.40,0.40}{##1}}}
\expandafter\def\csname PY@tok@mh\endcsname{\def\PY@tc##1{\textcolor[rgb]{0.40,0.40,0.40}{##1}}}
\expandafter\def\csname PY@tok@mi\endcsname{\def\PY@tc##1{\textcolor[rgb]{0.40,0.40,0.40}{##1}}}
\expandafter\def\csname PY@tok@il\endcsname{\def\PY@tc##1{\textcolor[rgb]{0.40,0.40,0.40}{##1}}}
\expandafter\def\csname PY@tok@mo\endcsname{\def\PY@tc##1{\textcolor[rgb]{0.40,0.40,0.40}{##1}}}
\expandafter\def\csname PY@tok@ch\endcsname{\let\PY@it=\textit\def\PY@tc##1{\textcolor[rgb]{0.25,0.50,0.50}{##1}}}
\expandafter\def\csname PY@tok@cm\endcsname{\let\PY@it=\textit\def\PY@tc##1{\textcolor[rgb]{0.25,0.50,0.50}{##1}}}
\expandafter\def\csname PY@tok@cpf\endcsname{\let\PY@it=\textit\def\PY@tc##1{\textcolor[rgb]{0.25,0.50,0.50}{##1}}}
\expandafter\def\csname PY@tok@c1\endcsname{\let\PY@it=\textit\def\PY@tc##1{\textcolor[rgb]{0.25,0.50,0.50}{##1}}}
\expandafter\def\csname PY@tok@cs\endcsname{\let\PY@it=\textit\def\PY@tc##1{\textcolor[rgb]{0.25,0.50,0.50}{##1}}}

\def\PYZbs{\char`\\}
\def\PYZus{\char`\_}
\def\PYZob{\char`\{}
\def\PYZcb{\char`\}}
\def\PYZca{\char`\^}
\def\PYZam{\char`\&}
\def\PYZlt{\char`\<}
\def\PYZgt{\char`\>}
\def\PYZsh{\char`\#}
\def\PYZpc{\char`\%}
\def\PYZdl{\char`\$}
\def\PYZhy{\char`\-}
\def\PYZsq{\char`\'}
\def\PYZdq{\char`\"}
\def\PYZti{\char`\~}
% for compatibility with earlier versions
\def\PYZat{@}
\def\PYZlb{[}
\def\PYZrb{]}
\makeatother


    % Exact colors from NB
    \definecolor{incolor}{rgb}{0.0, 0.0, 0.5}
    \definecolor{outcolor}{rgb}{0.545, 0.0, 0.0}



    
    % Prevent overflowing lines due to hard-to-break entities
    \sloppy 
    % Setup hyperref package
    \hypersetup{
      breaklinks=true,  % so long urls are correctly broken across lines
      colorlinks=true,
      urlcolor=urlcolor,
      linkcolor=linkcolor,
      citecolor=citecolor,
      }
    % Slightly bigger margins than the latex defaults
    
    \geometry{verbose,tmargin=1in,bmargin=1in,lmargin=1in,rmargin=1in}
    
    

    \begin{document}
    
    
    \maketitle
    
    

    
    \section{1.8 bounding the error of
expressions}\label{bounding-the-error-of-expressions}

\subsection{modelling expressions with simple bivariate
functions}\label{modelling-expressions-with-simple-bivariate-functions}

\[\renewcommand{\R}{\mathbb{R}}\]

\begin{itemize}
\tightlist
\item
  let a set of integers \(i_1,\ldots,i_n\) and \(j_1,\ldots,j_n\)
  satisfy

  \begin{itemize}
  \tightlist
  \item
    either \(i_k=j_k=0\)
  \item
    or \(j_k < i_k < k\)
  \end{itemize}
\item
  let \(f_1,\ldots, f_n\) be bivariate real functions defined on compact
  domains

  \begin{itemize}
  \tightlist
  \item
    the functions \(f_k\) are either arithmetic binary operations or
    univariate functions
  \end{itemize}
\item
  let \(u_0=0\) and \(u_k\) be defined by the system of equations
  \[u_k = f_k(u_{i_k}, u_{j_k}), \quad k=1,\ldots,n\]
\end{itemize}

\subsection{evaluation of the
expression}\label{evaluation-of-the-expression}

\begin{itemize}
\tightlist
\item
  these equations are thus solved (i.e. all \(u_k\) computed) by
  substitution

  \begin{align*}
    u_1 &= f_1(u_0,u_0) = f_1(0,0) \\
    u_2 &= f_2(u_{i_2},u_0) = f_2(u_{i_2},0), \quad i_2\in \{0,1\}\\
    u_3 &= f_3(u_{i_3},u_{j_3}), \quad i_3 \in \{0,1,2\}, \; j_3 \in \{0,\ldots,i_3\}\\
    & \cdots \\
    u_n &= f_n(u_{i_n},u_{j_n}), \quad i_n \in \{0,\ldots,n-1\}, \; j_3 \in \{0,\ldots,i_n\}
  \end{align*}
\item
  with this we have modeled the evaluation of numerical expressions
  where \(u_n\) is the value of the expression and the other \(u_k\)
  intermediate results
\end{itemize}

\subsection{\texorpdfstring{example
\(\left(-p+\sqrt{p^2-4\,q}\right)/2\)}{example \textbackslash{}left(-p+\textbackslash{}sqrt\{p\^{}2-4\textbackslash{},q\}\textbackslash{}right)/2}}\label{example-left-psqrtp2-4qright2}

\begin{align*}
u_1 &= p \\
u_2 &= q \\
u_3 &= u_1^2 \\
u_4 &= u_3 - 4\, u_2 \\
u_5 &= \sqrt{u_4} \\
u_6 &= \left(-u_1+u_5\right)/2
\end{align*}

\subsection{the same with rounding errors at every
step}\label{the-same-with-rounding-errors-at-every-step}

\begin{itemize}
\tightlist
\item
  now let \(v_k\) be the numerical versions of \(u_k\) defined by
  \[v_k = (1+\delta_k)\,f_k(v_{i_k},v_{j_k}), \quad k=1,\ldots,n\] and
  \(v_0 = 0\)
\item
  as usual \(|\delta_k| \leq \epsilon\)
\item
  the relative error of \(v_k\), i.e., \((v_k-u_k)/u_k\) is denoted by
  \(\theta_k\) so that \[v_k = (1+\theta_k) u_k\]
\end{itemize}

\subsection{example with rounding
errors}\label{example-with-rounding-errors}

\begin{align*}
v_1 &= (1+\delta_1)p \\
v_2 &= (1+\delta_2)q \\
v_3 &= (1+\delta_3) v_1^2 \\
v_4 &= (1+\delta_4)(v_3 - 4\, v_2) \\
v_5 &= (1+\delta_5)\sqrt{v_4} \\
v_6 &= (1+\delta_6)\left(-v_1+v_5\right)/2
\end{align*}

\subsection{total error at every step -\/- for multiplication and
division}\label{total-error-at-every-step----for-multiplication-and-division}

\begin{itemize}
\item
  recall: \(f_k(x_i, x_j)\) is either an arithmetic binary operation
  (like sum) of \(x_i\) and \(x_f\) \emph{or} a unary operation
  \(f_k(x_i)\)
\item
  the simplest cases are multiplication and division
\item
  for multiplication \(f_k(v_i,v_j) = (1+\theta_i)(1+\theta_j) u_i u_j\)
  and so \[v_k = (1+\delta_k)(1+\theta_i)(1+\theta_j)\, u_k\]

  \begin{itemize}
  \tightlist
  \item
    multiplication:
    \[\theta_k = (1+\delta_k)(1+\theta_i)(1+\theta_j) - 1 \approx \theta_i + \theta_j + \delta_k\]
  \item
    division:
    \[\theta_k = (1+\delta_k)(1+\theta_i)/(1+\theta_j) - 1 \approx \theta_i - \theta_j + \delta_k\]
  \end{itemize}
\end{itemize}

\subsection{total error at every step -\/- for addition and
subtraction}\label{total-error-at-every-step----for-addition-and-subtraction}

\begin{itemize}
\item
  for addition \(f_k(v_i,v_j) = (1+\theta_i)u_i + (1+\theta_j)u_j\) and
  so
  \[v_k = (1+\delta_k)\left((1+\theta_i)\frac{u_i}{u_i+u_j} + (1+\theta_j)\frac{u_j}{u_i+u_j}\right)(u_i + u_j)\]

  \begin{itemize}
  \tightlist
  \item
    addition:
    \[\theta_k = (1+\delta_k)\left(1 + \zeta_k\theta_i + (1-\zeta_k)\theta_j\right)-1 \approx
      \zeta_k\theta_i + (1-\zeta_k)\theta_j + \delta_k\]
  \end{itemize}

  where \(\zeta_k = u_i/(u_i+u_j)\)
\item
  convex combination if \(u_i\) and \(u_j\) have equal sign
\item
  if different sign, error can be very large despite the fact that some
  times \(\delta_k = 0\) in this case
\item
  similar for subtraction
\end{itemize}

\subsection{total error at every step -\/- for univariate
function}\label{total-error-at-every-step----for-univariate-function}

\begin{itemize}
\item
  \(f_k(v_i) = f_k((1+\theta_i)u_i)\) and so

  \begin{align*} v_k &= (1+\delta_k)\, f_k((1+\theta_i)u_i) \\
                   &= (1+\delta_k)\, \left(1 + \frac{f_k((1+\theta_i)u_i) - f_k(u_i)}{f_k(u_i)}\right)u_k\\
                   &= (1+\delta_k)\, (1 + \zeta_k\theta_i) u_k
    \end{align*}

  where \(\zeta_k = \frac{f_k((1+\theta_i)u_i) - f_k(u_i)}{f_k(u_i)}\)
  and \[|\zeta_k| \leq \frac{L_k |u_i|}{|f(u_i)|}\] if \(L_k\) is
  Lipschitz constant of \(f_k\)
\item
  relative error of \(v_k\) is then
  \[\theta_k = (1+\delta_k)(1+\zeta_k \theta_i) - 1 \approx \zeta_k \theta_i + \delta_k\]
\end{itemize}

\subsection{relative errors for
example}\label{relative-errors-for-example}

\begin{align*}
  \theta_1 &= \delta_1 \\
  \theta_2 &= \delta_2 \\
  \theta_3 &= (1+\delta_3)(1+\theta_1)^2 - 1 \\
  \theta_4 &= (1+\delta_4)(1+\zeta_4\theta_3-(1-\zeta_4)\theta_2) - 1\\
  \theta_5 &= (1+\delta_5)(1+\zeta_5\theta_4)-1\\
  \theta_6 &= (1+\delta_6)(1-\zeta_6\theta_1+(1-\zeta_6)\theta_5)-1
\end{align*}

\begin{itemize}
\tightlist
\item
  homework: what are the \(\zeta_k\), get bounds and obtain a bound for
  \(\theta_6\)
\end{itemize}

\subsection{stability and growth
factor}\label{stability-and-growth-factor}

\begin{itemize}
\tightlist
\item
  we say that the \(f_k\) are \textbf{stable} for if there exists some
  \(L>0\) such that for all \(k\) one has
  \[|f_k(x_1,x_2) - f_k(y_1,y_2)| \leq L \max_i |x_i - y_i|\]
\item
  we assume that for \(k>0\) one has \(u_k \neq 0\)
\item
  then one can define a \emph{growth factor}
  \[\rho = \max \{ |u_j|/|u_k| \mid j < k \}\]
\end{itemize}

\subsection{a simple global error
bound}\label{a-simple-global-error-bound}

\textbf{Proposition} Let \(\alpha=(1+\epsilon)L\rho\) where \(L\) be as
defined above, \(\rho\) be the growth factor then

\[v_k = (1+\theta_k) u_k\]

where
\[|\theta_k| \leq \left(\frac{\alpha^{k+1}-1}{\alpha-1}\right) \epsilon\]

\begin{center}\rule{0.5\linewidth}{\linethickness}\end{center}

\textbf{proof.}

\begin{itemize}
\item
  induction
\item
  first one has \[v_1 = (1+\delta_1) u_1\] and thus
  \(\theta_1 = \delta_1\) and \(|\theta_1| = |\delta_1| \leq \epsilon\)
\item
  then

  \begin{align*} v_{k+1} &= (1+\delta_{k+1}) f_{k+1}(v_{i_{k+1}}, v_{j_{k+1}})\\
   &= (1+\theta_{k+1}) u_{k+1}
    \end{align*}

  where \[
    \theta_{k+1} = \delta_{k+1} + (1+\delta_{k+1})\frac{f_{k+1}(v_{i_{k+1}},v_{j_{k+1}})-f_{k+1}(u_{i_{k+1}},u_{j_{k+1}})}{u_{k+1}}
    \]
\end{itemize}

\begin{center}\rule{0.5\linewidth}{\linethickness}\end{center}

\begin{itemize}
\tightlist
\item
  the (absolute value of the) first term is bounded by \(\epsilon\) and
  for the second term one has for some \(0< i \leq k\):

  \begin{align*}
     (1+\delta_{k+1})\left|\frac{f_{k+1}(v_{i_{k+1}},v_{j_{k+1}})-f_{k+1}(u_{i_{k+1}},u_{j_{k+1}})}{u_{k+1}}\right|
     & \leq (1+\epsilon)L \frac{\left| v_i - u_i\right|}{\left|u_{k+1}\right|} \\
     & = \frac{(1+\epsilon)L|\theta_i|\cdot|u_i|}{|u_{k+1}|} \\ & \leq L(1+\epsilon)\frac{\alpha^{i+1}-1}{\alpha-1}\rho\epsilon \\ 
     &\leq \frac{\alpha^{k+2}-\alpha}{\alpha-1} \epsilon
    \end{align*}

  from which one gets
  \[|\theta_{k+1}| \leq \frac{\alpha^{k+2}-1}{\alpha-1}\epsilon\]
  \(\blacksquare\)
\end{itemize}


    % Add a bibliography block to the postdoc
    
    
    
    \end{document}
