
% Default to the notebook output style

    


% Inherit from the specified cell style.




    
\documentclass[11pt]{article}

    
    
    \usepackage[T1]{fontenc}
    % Nicer default font (+ math font) than Computer Modern for most use cases
    \usepackage{mathpazo}

    % Basic figure setup, for now with no caption control since it's done
    % automatically by Pandoc (which extracts ![](path) syntax from Markdown).
    \usepackage{graphicx}
    % We will generate all images so they have a width \maxwidth. This means
    % that they will get their normal width if they fit onto the page, but
    % are scaled down if they would overflow the margins.
    \makeatletter
    \def\maxwidth{\ifdim\Gin@nat@width>\linewidth\linewidth
    \else\Gin@nat@width\fi}
    \makeatother
    \let\Oldincludegraphics\includegraphics
    % Set max figure width to be 80% of text width, for now hardcoded.
    \renewcommand{\includegraphics}[1]{\Oldincludegraphics[width=.8\maxwidth]{#1}}
    % Ensure that by default, figures have no caption (until we provide a
    % proper Figure object with a Caption API and a way to capture that
    % in the conversion process - todo).
    \usepackage{caption}
    \DeclareCaptionLabelFormat{nolabel}{}
    \captionsetup{labelformat=nolabel}

    \usepackage{adjustbox} % Used to constrain images to a maximum size 
    \usepackage{xcolor} % Allow colors to be defined
    \usepackage{enumerate} % Needed for markdown enumerations to work
    \usepackage{geometry} % Used to adjust the document margins
    \usepackage{amsmath} % Equations
    \usepackage{amssymb} % Equations
    \usepackage{textcomp} % defines textquotesingle
    % Hack from http://tex.stackexchange.com/a/47451/13684:
    \AtBeginDocument{%
        \def\PYZsq{\textquotesingle}% Upright quotes in Pygmentized code
    }
    \usepackage{upquote} % Upright quotes for verbatim code
    \usepackage{eurosym} % defines \euro
    \usepackage[mathletters]{ucs} % Extended unicode (utf-8) support
    \usepackage[utf8x]{inputenc} % Allow utf-8 characters in the tex document
    \usepackage{fancyvrb} % verbatim replacement that allows latex
    \usepackage{grffile} % extends the file name processing of package graphics 
                         % to support a larger range 
    % The hyperref package gives us a pdf with properly built
    % internal navigation ('pdf bookmarks' for the table of contents,
    % internal cross-reference links, web links for URLs, etc.)
    \usepackage{hyperref}
    \usepackage{longtable} % longtable support required by pandoc >1.10
    \usepackage{booktabs}  % table support for pandoc > 1.12.2
    \usepackage[inline]{enumitem} % IRkernel/repr support (it uses the enumerate* environment)
    \usepackage[normalem]{ulem} % ulem is needed to support strikethroughs (\sout)
                                % normalem makes italics be italics, not underlines
    

    
    
    % Colors for the hyperref package
    \definecolor{urlcolor}{rgb}{0,.145,.698}
    \definecolor{linkcolor}{rgb}{.71,0.21,0.01}
    \definecolor{citecolor}{rgb}{.12,.54,.11}

    % ANSI colors
    \definecolor{ansi-black}{HTML}{3E424D}
    \definecolor{ansi-black-intense}{HTML}{282C36}
    \definecolor{ansi-red}{HTML}{E75C58}
    \definecolor{ansi-red-intense}{HTML}{B22B31}
    \definecolor{ansi-green}{HTML}{00A250}
    \definecolor{ansi-green-intense}{HTML}{007427}
    \definecolor{ansi-yellow}{HTML}{DDB62B}
    \definecolor{ansi-yellow-intense}{HTML}{B27D12}
    \definecolor{ansi-blue}{HTML}{208FFB}
    \definecolor{ansi-blue-intense}{HTML}{0065CA}
    \definecolor{ansi-magenta}{HTML}{D160C4}
    \definecolor{ansi-magenta-intense}{HTML}{A03196}
    \definecolor{ansi-cyan}{HTML}{60C6C8}
    \definecolor{ansi-cyan-intense}{HTML}{258F8F}
    \definecolor{ansi-white}{HTML}{C5C1B4}
    \definecolor{ansi-white-intense}{HTML}{A1A6B2}

    % commands and environments needed by pandoc snippets
    % extracted from the output of `pandoc -s`
    \providecommand{\tightlist}{%
      \setlength{\itemsep}{0pt}\setlength{\parskip}{0pt}}
    \DefineVerbatimEnvironment{Highlighting}{Verbatim}{commandchars=\\\{\}}
    % Add ',fontsize=\small' for more characters per line
    \newenvironment{Shaded}{}{}
    \newcommand{\KeywordTok}[1]{\textcolor[rgb]{0.00,0.44,0.13}{\textbf{{#1}}}}
    \newcommand{\DataTypeTok}[1]{\textcolor[rgb]{0.56,0.13,0.00}{{#1}}}
    \newcommand{\DecValTok}[1]{\textcolor[rgb]{0.25,0.63,0.44}{{#1}}}
    \newcommand{\BaseNTok}[1]{\textcolor[rgb]{0.25,0.63,0.44}{{#1}}}
    \newcommand{\FloatTok}[1]{\textcolor[rgb]{0.25,0.63,0.44}{{#1}}}
    \newcommand{\CharTok}[1]{\textcolor[rgb]{0.25,0.44,0.63}{{#1}}}
    \newcommand{\StringTok}[1]{\textcolor[rgb]{0.25,0.44,0.63}{{#1}}}
    \newcommand{\CommentTok}[1]{\textcolor[rgb]{0.38,0.63,0.69}{\textit{{#1}}}}
    \newcommand{\OtherTok}[1]{\textcolor[rgb]{0.00,0.44,0.13}{{#1}}}
    \newcommand{\AlertTok}[1]{\textcolor[rgb]{1.00,0.00,0.00}{\textbf{{#1}}}}
    \newcommand{\FunctionTok}[1]{\textcolor[rgb]{0.02,0.16,0.49}{{#1}}}
    \newcommand{\RegionMarkerTok}[1]{{#1}}
    \newcommand{\ErrorTok}[1]{\textcolor[rgb]{1.00,0.00,0.00}{\textbf{{#1}}}}
    \newcommand{\NormalTok}[1]{{#1}}
    
    % Additional commands for more recent versions of Pandoc
    \newcommand{\ConstantTok}[1]{\textcolor[rgb]{0.53,0.00,0.00}{{#1}}}
    \newcommand{\SpecialCharTok}[1]{\textcolor[rgb]{0.25,0.44,0.63}{{#1}}}
    \newcommand{\VerbatimStringTok}[1]{\textcolor[rgb]{0.25,0.44,0.63}{{#1}}}
    \newcommand{\SpecialStringTok}[1]{\textcolor[rgb]{0.73,0.40,0.53}{{#1}}}
    \newcommand{\ImportTok}[1]{{#1}}
    \newcommand{\DocumentationTok}[1]{\textcolor[rgb]{0.73,0.13,0.13}{\textit{{#1}}}}
    \newcommand{\AnnotationTok}[1]{\textcolor[rgb]{0.38,0.63,0.69}{\textbf{\textit{{#1}}}}}
    \newcommand{\CommentVarTok}[1]{\textcolor[rgb]{0.38,0.63,0.69}{\textbf{\textit{{#1}}}}}
    \newcommand{\VariableTok}[1]{\textcolor[rgb]{0.10,0.09,0.49}{{#1}}}
    \newcommand{\ControlFlowTok}[1]{\textcolor[rgb]{0.00,0.44,0.13}{\textbf{{#1}}}}
    \newcommand{\OperatorTok}[1]{\textcolor[rgb]{0.40,0.40,0.40}{{#1}}}
    \newcommand{\BuiltInTok}[1]{{#1}}
    \newcommand{\ExtensionTok}[1]{{#1}}
    \newcommand{\PreprocessorTok}[1]{\textcolor[rgb]{0.74,0.48,0.00}{{#1}}}
    \newcommand{\AttributeTok}[1]{\textcolor[rgb]{0.49,0.56,0.16}{{#1}}}
    \newcommand{\InformationTok}[1]{\textcolor[rgb]{0.38,0.63,0.69}{\textbf{\textit{{#1}}}}}
    \newcommand{\WarningTok}[1]{\textcolor[rgb]{0.38,0.63,0.69}{\textbf{\textit{{#1}}}}}
    
    
    % Define a nice break command that doesn't care if a line doesn't already
    % exist.
    \def\br{\hspace*{\fill} \\* }
    % Math Jax compatability definitions
    \def\gt{>}
    \def\lt{<}
    % Document parameters
    \title{ass2}
    
    
    

    % Pygments definitions
    
\makeatletter
\def\PY@reset{\let\PY@it=\relax \let\PY@bf=\relax%
    \let\PY@ul=\relax \let\PY@tc=\relax%
    \let\PY@bc=\relax \let\PY@ff=\relax}
\def\PY@tok#1{\csname PY@tok@#1\endcsname}
\def\PY@toks#1+{\ifx\relax#1\empty\else%
    \PY@tok{#1}\expandafter\PY@toks\fi}
\def\PY@do#1{\PY@bc{\PY@tc{\PY@ul{%
    \PY@it{\PY@bf{\PY@ff{#1}}}}}}}
\def\PY#1#2{\PY@reset\PY@toks#1+\relax+\PY@do{#2}}

\expandafter\def\csname PY@tok@w\endcsname{\def\PY@tc##1{\textcolor[rgb]{0.73,0.73,0.73}{##1}}}
\expandafter\def\csname PY@tok@c\endcsname{\let\PY@it=\textit\def\PY@tc##1{\textcolor[rgb]{0.25,0.50,0.50}{##1}}}
\expandafter\def\csname PY@tok@cp\endcsname{\def\PY@tc##1{\textcolor[rgb]{0.74,0.48,0.00}{##1}}}
\expandafter\def\csname PY@tok@k\endcsname{\let\PY@bf=\textbf\def\PY@tc##1{\textcolor[rgb]{0.00,0.50,0.00}{##1}}}
\expandafter\def\csname PY@tok@kp\endcsname{\def\PY@tc##1{\textcolor[rgb]{0.00,0.50,0.00}{##1}}}
\expandafter\def\csname PY@tok@kt\endcsname{\def\PY@tc##1{\textcolor[rgb]{0.69,0.00,0.25}{##1}}}
\expandafter\def\csname PY@tok@o\endcsname{\def\PY@tc##1{\textcolor[rgb]{0.40,0.40,0.40}{##1}}}
\expandafter\def\csname PY@tok@ow\endcsname{\let\PY@bf=\textbf\def\PY@tc##1{\textcolor[rgb]{0.67,0.13,1.00}{##1}}}
\expandafter\def\csname PY@tok@nb\endcsname{\def\PY@tc##1{\textcolor[rgb]{0.00,0.50,0.00}{##1}}}
\expandafter\def\csname PY@tok@nf\endcsname{\def\PY@tc##1{\textcolor[rgb]{0.00,0.00,1.00}{##1}}}
\expandafter\def\csname PY@tok@nc\endcsname{\let\PY@bf=\textbf\def\PY@tc##1{\textcolor[rgb]{0.00,0.00,1.00}{##1}}}
\expandafter\def\csname PY@tok@nn\endcsname{\let\PY@bf=\textbf\def\PY@tc##1{\textcolor[rgb]{0.00,0.00,1.00}{##1}}}
\expandafter\def\csname PY@tok@ne\endcsname{\let\PY@bf=\textbf\def\PY@tc##1{\textcolor[rgb]{0.82,0.25,0.23}{##1}}}
\expandafter\def\csname PY@tok@nv\endcsname{\def\PY@tc##1{\textcolor[rgb]{0.10,0.09,0.49}{##1}}}
\expandafter\def\csname PY@tok@no\endcsname{\def\PY@tc##1{\textcolor[rgb]{0.53,0.00,0.00}{##1}}}
\expandafter\def\csname PY@tok@nl\endcsname{\def\PY@tc##1{\textcolor[rgb]{0.63,0.63,0.00}{##1}}}
\expandafter\def\csname PY@tok@ni\endcsname{\let\PY@bf=\textbf\def\PY@tc##1{\textcolor[rgb]{0.60,0.60,0.60}{##1}}}
\expandafter\def\csname PY@tok@na\endcsname{\def\PY@tc##1{\textcolor[rgb]{0.49,0.56,0.16}{##1}}}
\expandafter\def\csname PY@tok@nt\endcsname{\let\PY@bf=\textbf\def\PY@tc##1{\textcolor[rgb]{0.00,0.50,0.00}{##1}}}
\expandafter\def\csname PY@tok@nd\endcsname{\def\PY@tc##1{\textcolor[rgb]{0.67,0.13,1.00}{##1}}}
\expandafter\def\csname PY@tok@s\endcsname{\def\PY@tc##1{\textcolor[rgb]{0.73,0.13,0.13}{##1}}}
\expandafter\def\csname PY@tok@sd\endcsname{\let\PY@it=\textit\def\PY@tc##1{\textcolor[rgb]{0.73,0.13,0.13}{##1}}}
\expandafter\def\csname PY@tok@si\endcsname{\let\PY@bf=\textbf\def\PY@tc##1{\textcolor[rgb]{0.73,0.40,0.53}{##1}}}
\expandafter\def\csname PY@tok@se\endcsname{\let\PY@bf=\textbf\def\PY@tc##1{\textcolor[rgb]{0.73,0.40,0.13}{##1}}}
\expandafter\def\csname PY@tok@sr\endcsname{\def\PY@tc##1{\textcolor[rgb]{0.73,0.40,0.53}{##1}}}
\expandafter\def\csname PY@tok@ss\endcsname{\def\PY@tc##1{\textcolor[rgb]{0.10,0.09,0.49}{##1}}}
\expandafter\def\csname PY@tok@sx\endcsname{\def\PY@tc##1{\textcolor[rgb]{0.00,0.50,0.00}{##1}}}
\expandafter\def\csname PY@tok@m\endcsname{\def\PY@tc##1{\textcolor[rgb]{0.40,0.40,0.40}{##1}}}
\expandafter\def\csname PY@tok@gh\endcsname{\let\PY@bf=\textbf\def\PY@tc##1{\textcolor[rgb]{0.00,0.00,0.50}{##1}}}
\expandafter\def\csname PY@tok@gu\endcsname{\let\PY@bf=\textbf\def\PY@tc##1{\textcolor[rgb]{0.50,0.00,0.50}{##1}}}
\expandafter\def\csname PY@tok@gd\endcsname{\def\PY@tc##1{\textcolor[rgb]{0.63,0.00,0.00}{##1}}}
\expandafter\def\csname PY@tok@gi\endcsname{\def\PY@tc##1{\textcolor[rgb]{0.00,0.63,0.00}{##1}}}
\expandafter\def\csname PY@tok@gr\endcsname{\def\PY@tc##1{\textcolor[rgb]{1.00,0.00,0.00}{##1}}}
\expandafter\def\csname PY@tok@ge\endcsname{\let\PY@it=\textit}
\expandafter\def\csname PY@tok@gs\endcsname{\let\PY@bf=\textbf}
\expandafter\def\csname PY@tok@gp\endcsname{\let\PY@bf=\textbf\def\PY@tc##1{\textcolor[rgb]{0.00,0.00,0.50}{##1}}}
\expandafter\def\csname PY@tok@go\endcsname{\def\PY@tc##1{\textcolor[rgb]{0.53,0.53,0.53}{##1}}}
\expandafter\def\csname PY@tok@gt\endcsname{\def\PY@tc##1{\textcolor[rgb]{0.00,0.27,0.87}{##1}}}
\expandafter\def\csname PY@tok@err\endcsname{\def\PY@bc##1{\setlength{\fboxsep}{0pt}\fcolorbox[rgb]{1.00,0.00,0.00}{1,1,1}{\strut ##1}}}
\expandafter\def\csname PY@tok@kc\endcsname{\let\PY@bf=\textbf\def\PY@tc##1{\textcolor[rgb]{0.00,0.50,0.00}{##1}}}
\expandafter\def\csname PY@tok@kd\endcsname{\let\PY@bf=\textbf\def\PY@tc##1{\textcolor[rgb]{0.00,0.50,0.00}{##1}}}
\expandafter\def\csname PY@tok@kn\endcsname{\let\PY@bf=\textbf\def\PY@tc##1{\textcolor[rgb]{0.00,0.50,0.00}{##1}}}
\expandafter\def\csname PY@tok@kr\endcsname{\let\PY@bf=\textbf\def\PY@tc##1{\textcolor[rgb]{0.00,0.50,0.00}{##1}}}
\expandafter\def\csname PY@tok@bp\endcsname{\def\PY@tc##1{\textcolor[rgb]{0.00,0.50,0.00}{##1}}}
\expandafter\def\csname PY@tok@fm\endcsname{\def\PY@tc##1{\textcolor[rgb]{0.00,0.00,1.00}{##1}}}
\expandafter\def\csname PY@tok@vc\endcsname{\def\PY@tc##1{\textcolor[rgb]{0.10,0.09,0.49}{##1}}}
\expandafter\def\csname PY@tok@vg\endcsname{\def\PY@tc##1{\textcolor[rgb]{0.10,0.09,0.49}{##1}}}
\expandafter\def\csname PY@tok@vi\endcsname{\def\PY@tc##1{\textcolor[rgb]{0.10,0.09,0.49}{##1}}}
\expandafter\def\csname PY@tok@vm\endcsname{\def\PY@tc##1{\textcolor[rgb]{0.10,0.09,0.49}{##1}}}
\expandafter\def\csname PY@tok@sa\endcsname{\def\PY@tc##1{\textcolor[rgb]{0.73,0.13,0.13}{##1}}}
\expandafter\def\csname PY@tok@sb\endcsname{\def\PY@tc##1{\textcolor[rgb]{0.73,0.13,0.13}{##1}}}
\expandafter\def\csname PY@tok@sc\endcsname{\def\PY@tc##1{\textcolor[rgb]{0.73,0.13,0.13}{##1}}}
\expandafter\def\csname PY@tok@dl\endcsname{\def\PY@tc##1{\textcolor[rgb]{0.73,0.13,0.13}{##1}}}
\expandafter\def\csname PY@tok@s2\endcsname{\def\PY@tc##1{\textcolor[rgb]{0.73,0.13,0.13}{##1}}}
\expandafter\def\csname PY@tok@sh\endcsname{\def\PY@tc##1{\textcolor[rgb]{0.73,0.13,0.13}{##1}}}
\expandafter\def\csname PY@tok@s1\endcsname{\def\PY@tc##1{\textcolor[rgb]{0.73,0.13,0.13}{##1}}}
\expandafter\def\csname PY@tok@mb\endcsname{\def\PY@tc##1{\textcolor[rgb]{0.40,0.40,0.40}{##1}}}
\expandafter\def\csname PY@tok@mf\endcsname{\def\PY@tc##1{\textcolor[rgb]{0.40,0.40,0.40}{##1}}}
\expandafter\def\csname PY@tok@mh\endcsname{\def\PY@tc##1{\textcolor[rgb]{0.40,0.40,0.40}{##1}}}
\expandafter\def\csname PY@tok@mi\endcsname{\def\PY@tc##1{\textcolor[rgb]{0.40,0.40,0.40}{##1}}}
\expandafter\def\csname PY@tok@il\endcsname{\def\PY@tc##1{\textcolor[rgb]{0.40,0.40,0.40}{##1}}}
\expandafter\def\csname PY@tok@mo\endcsname{\def\PY@tc##1{\textcolor[rgb]{0.40,0.40,0.40}{##1}}}
\expandafter\def\csname PY@tok@ch\endcsname{\let\PY@it=\textit\def\PY@tc##1{\textcolor[rgb]{0.25,0.50,0.50}{##1}}}
\expandafter\def\csname PY@tok@cm\endcsname{\let\PY@it=\textit\def\PY@tc##1{\textcolor[rgb]{0.25,0.50,0.50}{##1}}}
\expandafter\def\csname PY@tok@cpf\endcsname{\let\PY@it=\textit\def\PY@tc##1{\textcolor[rgb]{0.25,0.50,0.50}{##1}}}
\expandafter\def\csname PY@tok@c1\endcsname{\let\PY@it=\textit\def\PY@tc##1{\textcolor[rgb]{0.25,0.50,0.50}{##1}}}
\expandafter\def\csname PY@tok@cs\endcsname{\let\PY@it=\textit\def\PY@tc##1{\textcolor[rgb]{0.25,0.50,0.50}{##1}}}

\def\PYZbs{\char`\\}
\def\PYZus{\char`\_}
\def\PYZob{\char`\{}
\def\PYZcb{\char`\}}
\def\PYZca{\char`\^}
\def\PYZam{\char`\&}
\def\PYZlt{\char`\<}
\def\PYZgt{\char`\>}
\def\PYZsh{\char`\#}
\def\PYZpc{\char`\%}
\def\PYZdl{\char`\$}
\def\PYZhy{\char`\-}
\def\PYZsq{\char`\'}
\def\PYZdq{\char`\"}
\def\PYZti{\char`\~}
% for compatibility with earlier versions
\def\PYZat{@}
\def\PYZlb{[}
\def\PYZrb{]}
\makeatother


    % Exact colors from NB
    \definecolor{incolor}{rgb}{0.0, 0.0, 0.5}
    \definecolor{outcolor}{rgb}{0.545, 0.0, 0.0}



    
    % Prevent overflowing lines due to hard-to-break entities
    \sloppy 
    % Setup hyperref package
    \hypersetup{
      breaklinks=true,  % so long urls are correctly broken across lines
      colorlinks=true,
      urlcolor=urlcolor,
      linkcolor=linkcolor,
      citecolor=citecolor,
      }
    % Slightly bigger margins than the latex defaults
    
    \geometry{verbose,tmargin=1in,bmargin=1in,lmargin=1in,rmargin=1in}
    
    

    \begin{document}
    
    
    \maketitle
    
    

    
    remember to upload pdf Q1.1a

We can know that E is a elementary matrix. \[
E=\left[\begin{array}{cccc}
1 & 0 & 0 & 0\\
0 & 1 & 0 & 0\\
0 & {-\frac{x_3}{x_2}} & 1 & 0\\
0 & {-\frac{x_4}{x_2}} & 0 & 1
\end{array}\right]=I-ue_2^{T}
\]

Hence \[
ue_2^{T}=\left[\begin{array}{cccc}
0 & 0 & 0 & 0\\
0 & 0 & 0 & 0\\
0 & \frac{x_3}{x_2} & 0 & 0\\
0 & \frac{x_4}{x_2} & 0 & 0
\end{array}\right]
\]

Also, \[
e_2^{T}=\left[\begin{array}{cccc}
0 & 1 & 0 & 0\\
\end{array}\right]
\] Hence \[u=\left[\begin{array}{cccc}
0 & 0 & \frac{x_3}{x_2} & \frac{x_4}{x_2}\\
\end{array}\right]^T\]

    Q1.1b

\[
I-\alpha e_3e_2^{T}=\left[\begin{array}{cccc}
1 & 0 & 0 & 0\\
0 & 1 & 0 & 0\\
0 & 1-\alpha & 1 & 0\\
0 & 0 & 0 & 1
\end{array}\right]
\]

\[
I-\beta e_4e_2^{T}=\left[\begin{array}{cccc}
1 & 0 & 0 & 0\\
0 & 1 & 0 & 0\\
0 & 0 & 1 & 0\\
0 & 1-\beta & 0 & 1
\end{array}\right]
\]

\[
M=(I-\alpha e_3e_2^{T})(I-\beta e_4e_2^{T})=\left[\begin{array}{cccc}
1 & 0 & 0 & 0\\
0 & 1 & 0 & 0\\
0 & 1-\alpha & 1 & 0\\
0 & 1-\beta & 0 & 1
\end{array}\right]
\]

\(M=I-m_k e_k^T\) where \(m_k=[0\ 0\ \alpha-1\ \beta-1]^T\) so M is a
elementary matrix.

    Q1.c

\[x=M^{-1}\left[\begin{array}{c}
0 \\
4\\
0\\
0
\end{array}\right]=
\left[\begin{array}{c}
0 \\
4\\
4(\alpha-1)\\
4(\beta-1)
\end{array}\right]\]

    Q2.a

\(x^{k+1}-x^*=x^{k}-x^*-\omega A(x^{k}-x^*)=(I-\omega A)(x^k-x^*)\)

\(e^{(k+1)}=Ee^k\)

\(e^k=E^ke^0\), where \(E=I-\omega A\)

The eigenvalues for \(E=I-\omega A\) are
\(\lambda= 1-\omega ,1-\frac{1}{4}\omega ,1-\frac{1}{9}\omega, 1-\frac{1}{16}\omega\)

In order to make the method convergent \(\rho(E)=max(|\lambda|)<1\)

We need to require \(0<\omega<2\)

When \(\omega=\frac{32}{17}\) \(\rho(E)_{min}=\frac{15}{17}\) The best
possible value is 32/17

    Q2.b

For Jacobi \(M_J=D\) and \(E_J=I-D^{-1}A\) \[D=\left[\begin{array}{cccc}
4 & 0 & 0 & 0\\
0 & 5 & 0 & 0\\
0 & 0 & 6 & 0\\
0 & 0 & 0 & 3
\end{array}\right] \]

\[D^{-1}=\left[\begin{array}{cccc}
\frac{1}{4} & 0 & 0 & 0\\
0 & \frac{1}{5} & 0 & 0\\
0 & 0 & \frac{1}{6} & 0\\
0 & 0 & 0 & \frac{1}{3}
\end{array}\right] \]

\[E_J=I-D^{-1}A=\left[\begin{array}{cccc}
0& \frac{1}{4}  & 0 & 0\\
\frac{1}{5}  & 0 & \frac{1}{5}  & 0\\
0 & \frac{1}{6}  & 0 & \frac{1}{6} \\
0 & 0 & \frac{1}{3}  & 0
\end{array}\right]
\]

From the equation \[\begin{aligned}
    \rho(E_J)& = \rho(D E_J D^{-1}) = \rho((L+U) D^{-1}) \le \|(L+U) D^{-1}\|_1 \\
    & = \max_{1\le j\le n} \frac{1}{|a_{j,j}|} \sum_{\scriptstyle i=1 \atop \scriptstyle i\ne j}^n |a_{i,j}|.\end{aligned}\]

we know
\(\rho(E_J)=max\{\frac{1}{4},\frac{2}{5},\frac{1}{3}\}=\frac{2}{5}<1\)
so converge rate:2/5

    Q3.a

I prove \(d_k^Td_{k+1}=0\)

\[d_k^T d_{k+1}=d_k^T (b- A x_{k+1})=d_k^T (b- A x_{k}-A\alpha_k d_k)
=d_k^T (b- A x_{k})- d_k^T A (\displaystyle{\frac{d_k^T d_k}{d_k^T A d_k}}) d_k
=d_k^T d_k - d_k^T A d_k\displaystyle{\frac{d_k^T d_k}{d_k^T A d_k}}
=d_k^T d_k-d_k^T d_k=0\]

So the direction vectors d\(_k\) and d\(_{k+1}\) are orthogonal.

    \begin{Verbatim}[commandchars=\\\{\}]
{\color{incolor}In [{\color{incolor}44}]:} \PY{c+c1}{\PYZsh{}firstly we claculate the eigenvalue of the matrix A, and the ratio of the }
         \PY{c+c1}{\PYZsh{}maximum eigenvalue and minimum eigenvalue. }
         \PY{k+kn}{from} \PY{n+nn}{numpy} \PY{k}{import} \PY{n}{linalg} \PY{k}{as} \PY{n}{la}
         \PY{n}{A}\PY{o}{=}\PY{n}{np}\PY{o}{.}\PY{n}{array}\PY{p}{(}\PY{p}{[}\PY{p}{[}\PY{l+m+mi}{2}\PY{p}{,}\PY{o}{\PYZhy{}}\PY{l+m+mi}{1}\PY{p}{,}\PY{l+m+mi}{0}\PY{p}{,}\PY{l+m+mi}{0}\PY{p}{]}\PY{p}{,}
                    \PY{p}{[}\PY{o}{\PYZhy{}}\PY{l+m+mi}{1}\PY{p}{,}\PY{l+m+mi}{5}\PY{p}{,}\PY{o}{\PYZhy{}}\PY{l+m+mi}{1}\PY{p}{,}\PY{l+m+mi}{0}\PY{p}{]}\PY{p}{,}
                    \PY{p}{[}\PY{l+m+mi}{0}\PY{p}{,}\PY{o}{\PYZhy{}}\PY{l+m+mi}{1}\PY{p}{,}\PY{l+m+mi}{6}\PY{p}{,}\PY{o}{\PYZhy{}}\PY{l+m+mi}{1}\PY{p}{]}\PY{p}{,}
                    \PY{p}{[}\PY{l+m+mi}{0}\PY{p}{,}\PY{l+m+mi}{0}\PY{p}{,}\PY{o}{\PYZhy{}}\PY{l+m+mi}{1}\PY{p}{,}\PY{l+m+mi}{3}\PY{p}{]}\PY{p}{]}\PY{p}{)}
         \PY{n}{w}\PY{p}{,} \PY{n}{v} \PY{o}{=} \PY{n}{la}\PY{o}{.}\PY{n}{eig}\PY{p}{(}\PY{n}{A}\PY{p}{)}
         \PY{n+nb}{print}\PY{p}{(}\PY{n}{w}\PY{p}{)}
         \PY{n}{k}\PY{o}{=}\PY{n+nb}{max}\PY{p}{(}\PY{n}{w}\PY{p}{)}\PY{o}{/}\PY{n+nb}{min}\PY{p}{(}\PY{n}{w}\PY{p}{)}
         \PY{n}{k}
\end{Verbatim}


    \begin{Verbatim}[commandchars=\\\{\}]
[6.86247002 4.79195236 1.67195183 2.67362578]

    \end{Verbatim}

\begin{Verbatim}[commandchars=\\\{\}]
{\color{outcolor}Out[{\color{outcolor}44}]:} 4.104466337282004
\end{Verbatim}
            
    Q3.b

by defination \(x=[a, b, c, d]^T\)
\[x^TAx=[a,b,c,d]\left[\begin{array}{cccc}
2 & -1 & 0 & 0\\
-1 & 5 & -1  & 0\\
0 & -1 & 6 & -1 \\
0 & 0 & -1 & 3
\end{array}\right]
\left[\begin{array}{c}
a \\
b\\
c\\
d
\end{array}\right]
=[a,b,c,d]\left[\begin{array}{c}
2a-b \\
-a+5b-c\\
-b+6c-d\\
-c+3d
\end{array}\right]
=a^2+3b^2+4c^2+2d^2+(a-b)^2+(b-c)^2+(c-d)^2>0
if x\neq 0\] A is positive definite.

Or because the eigenvalues of A are all positive, hence A is positive
difinite.

By the calculation above, The condition number for A is given by
\[\kappa(A) = \frac{\lambda_{\max}(A)}{\lambda_{\min}(A)}\approx4.10\]
The
\(e^k\leq 2(\frac{\sqrt\kappa-1}{\sqrt\kappa+1})^k e^0\approx 2(0.34)^k e^0\)
\(2(0.34)^k\) is always less than 1 and convergent.

So this method is convergent.

    Q4.a

f(0)=-2 f(4)=222 f(0)f(4)\textless{}0 hence f(x) has a zero in the
intervall \([0, 4]\)

    Q4.b

We consider the problem in analytical way and numerical way:

error tolerance achieved when \((b_0-a_0)/2^{n+1}\leq \delta=0.01\) i.e.
\((4-0)/2^{n+1}\leq 0.01\)

\(n\geq2log_2(10)+1\)

The number of iteration steps required is that \(n\geq 8\)

    \begin{Verbatim}[commandchars=\\\{\}]
{\color{incolor}In [{\color{incolor}47}]:} \PY{c+c1}{\PYZsh{}numerical test}
         \PY{k}{def} \PY{n+nf}{bisect}\PY{p}{(}\PY{n}{f}\PY{p}{,}\PY{n}{a}\PY{p}{,}\PY{n}{b}\PY{p}{,}\PY{n}{n}\PY{p}{)}\PY{p}{:}
             \PY{c+c1}{\PYZsh{} first simple bisection code}
             \PY{n}{fa}\PY{p}{,} \PY{n}{fb} \PY{o}{=} \PY{n}{f}\PY{p}{(}\PY{n}{a}\PY{p}{)}\PY{p}{,} \PY{n}{f}\PY{p}{(}\PY{n}{b}\PY{p}{)}
             \PY{k}{for} \PY{n}{i} \PY{o+ow}{in} \PY{n+nb}{range}\PY{p}{(}\PY{n}{n}\PY{p}{)}\PY{p}{:}
                 \PY{n}{m} \PY{o}{=} \PY{p}{(}\PY{n}{a}\PY{o}{+}\PY{n}{b}\PY{p}{)}\PY{o}{/}\PY{l+m+mf}{2.0}
                 \PY{n}{fm} \PY{o}{=} \PY{n}{f}\PY{p}{(}\PY{n}{m}\PY{p}{)}
                 \PY{k}{if} \PY{n}{fa}\PY{o}{*}\PY{n}{fm} \PY{o}{\PYZlt{}}\PY{o}{=} \PY{l+m+mi}{0}\PY{p}{:}
                     \PY{n}{b}\PY{p}{,} \PY{n}{fb} \PY{o}{=} \PY{n}{m}\PY{p}{,} \PY{n}{fm}
                 \PY{k}{else}\PY{p}{:}
                     \PY{n}{a}\PY{p}{,} \PY{n}{fa} \PY{o}{=} \PY{n}{m}\PY{p}{,} \PY{n}{fm}        
             \PY{k}{return} \PY{p}{(}\PY{n}{a}\PY{o}{+}\PY{n}{b}\PY{p}{)}\PY{o}{/}\PY{l+m+mf}{2.0}\PY{p}{,} \PY{p}{(}\PY{n}{b}\PY{o}{\PYZhy{}}\PY{n}{a}\PY{p}{)}\PY{o}{/}\PY{l+m+mf}{2.0}\PY{p}{,} \PY{n}{f}\PY{p}{(}\PY{p}{(}\PY{n}{a}\PY{o}{+}\PY{n}{b}\PY{p}{)}\PY{o}{/}\PY{l+m+mi}{2}\PY{p}{)}  \PY{c+c1}{\PYZsh{} output}
         
         \PY{n}{f} \PY{o}{=} \PY{k}{lambda} \PY{n}{x} \PY{p}{:} \PY{n}{x}\PY{o}{*}\PY{o}{*}\PY{l+m+mi}{4}\PY{o}{\PYZhy{}}\PY{l+m+mi}{8}\PY{o}{*}\PY{n}{x}\PY{o}{\PYZhy{}}\PY{l+m+mi}{2}\PY{p}{;} \PY{n}{xex}\PY{o}{=}\PY{l+m+mf}{2.07722}
         \PY{n}{a} \PY{o}{=} \PY{l+m+mf}{0.0}\PY{p}{;} \PY{n}{b} \PY{o}{=} \PY{l+m+mf}{4.0}\PY{p}{;} \PY{n}{fa}\PY{p}{,} \PY{n}{fb} \PY{o}{=} \PY{n}{f}\PY{p}{(}\PY{n}{a}\PY{p}{)}\PY{p}{,} \PY{n}{f}\PY{p}{(}\PY{n}{b}\PY{p}{)}
         \PY{n+nb}{print}\PY{p}{(}\PY{l+s+s2}{\PYZdq{}}\PY{l+s+s2}{exact solution: }\PY{l+s+si}{\PYZpc{}g}\PY{l+s+s2}{\PYZdq{}}\PY{o}{\PYZpc{}}\PY{p}{(}\PY{n}{xex}\PY{p}{)}\PY{p}{)}
         \PY{n+nb}{print}\PY{p}{(}\PY{l+s+s2}{\PYZdq{}}\PY{l+s+s2}{interval [}\PY{l+s+si}{\PYZpc{}g}\PY{l+s+s2}{,}\PY{l+s+si}{\PYZpc{}g}\PY{l+s+s2}{] contains the zero? }\PY{l+s+si}{\PYZpc{}s}\PY{l+s+s2}{\PYZdq{}}\PY{o}{\PYZpc{}}\PY{p}{(}\PY{n}{a}\PY{p}{,}\PY{n}{b}\PY{p}{,}\PY{n}{fa}\PY{o}{*}\PY{n}{fb} \PY{o}{\PYZlt{}}\PY{o}{=} \PY{l+m+mi}{0}\PY{p}{)}\PY{p}{)} 
         \PY{k}{for} \PY{n}{k} \PY{o+ow}{in} \PY{n+nb}{range}\PY{p}{(}\PY{l+m+mi}{10}\PY{p}{)}\PY{p}{:}    
             \PY{n}{m}\PY{p}{,} \PY{n}{e}\PY{p}{,} \PY{n}{fm} \PY{o}{=} \PY{n}{bisect}\PY{p}{(}\PY{n}{f}\PY{p}{,}\PY{n}{a}\PY{p}{,}\PY{n}{b}\PY{p}{,}\PY{l+m+mi}{1}\PY{p}{)} \PY{c+c1}{\PYZsh{} one step}
             \PY{n}{a} \PY{o}{=} \PY{n}{m}\PY{o}{\PYZhy{}}\PY{n}{e}
             \PY{n}{b} \PY{o}{=} \PY{n}{m}\PY{o}{+}\PY{n}{e}
             \PY{n+nb}{print}\PY{p}{(}\PY{l+s+s2}{\PYZdq{}}\PY{l+s+si}{\PYZpc{}6.4g}\PY{l+s+s2}{ iterations: [a,b]=[}\PY{l+s+si}{\PYZpc{}6.4g}\PY{l+s+s2}{,}\PY{l+s+si}{\PYZpc{}6.4g}\PY{l+s+s2}{], x=}\PY{l+s+si}{\PYZpc{}8.5e}\PY{l+s+s2}{, e=}\PY{l+s+si}{\PYZpc{}8.1e}\PY{l+s+s2}{\PYZdq{}}\PY{o}{\PYZpc{}}\PY{p}{(}\PY{n}{k}\PY{p}{,}\PY{n}{a}\PY{p}{,}\PY{n}{b}\PY{p}{,}\PY{n}{m}\PY{p}{,}\PY{p}{(}\PY{n}{m}\PY{o}{\PYZhy{}}\PY{n}{xex}\PY{p}{)}\PY{p}{)}\PY{p}{)} 
\end{Verbatim}


    \begin{Verbatim}[commandchars=\\\{\}]
exact solution: 2.07722
interval [0,4] contains the zero? True
     0 iterations: [a,b]=[     2,     4], x=3.00000e+00, e= 9.2e-01
     1 iterations: [a,b]=[     2,     3], x=2.50000e+00, e= 4.2e-01
     2 iterations: [a,b]=[     2,   2.5], x=2.25000e+00, e= 1.7e-01
     3 iterations: [a,b]=[     2,  2.25], x=2.12500e+00, e= 4.8e-02
     4 iterations: [a,b]=[     2, 2.125], x=2.06250e+00, e=-1.5e-02
     5 iterations: [a,b]=[ 2.062, 2.125], x=2.09375e+00, e= 1.7e-02
     6 iterations: [a,b]=[ 2.062, 2.094], x=2.07812e+00, e= 9.0e-04
     7 iterations: [a,b]=[ 2.062, 2.078], x=2.07031e+00, e=-6.9e-03
     8 iterations: [a,b]=[  2.07, 2.078], x=2.07422e+00, e=-3.0e-03
     9 iterations: [a,b]=[ 2.074, 2.078], x=2.07617e+00, e=-1.0e-03

    \end{Verbatim}

    We can see that after 8 iterations the error is less than 0.01.

    Q5

If \(F\) is contractive on a real interval \([0,4]\) and
\(F([0,4]) \subset [0,4]\) then the sequence \(x_n\) defined by fixed
point iteration with \(F\) converges and the error satisfies
\[|e_n| \leq \lambda^n |e_0|.\] Then the method is convergent.

\(F(x) = x - \alpha f(x)\) needs to satisfy
\[|x-y-\alpha(f(x)-f(y))| \leq \lambda |x-y| \]for \(x,y \in [0, 4]\)

\[|1+8\alpha-\alpha(x^2+y^2)(x+y)||x-y|=\lambda |x-y|\leq|x-y|\]

Hence \[|1+8\alpha-\alpha(x^2+y^2)(x+y)|\leq1\]

for \(x,y \in [0, 4]\), \(0<(x^2+y^2)(x+y)<256\)

\(max(|1+8\alpha|,|1-248\alpha|)\leq1\)

So \(\alpha=0\) but \(\alpha\neq0\) so there is no \(\alpha\)

    Q6

\[F(x)=x-\frac{f(x)}{f'(x)}=x-\frac{x^4-8x-2}{4x^3-8}=\frac{3x^4+4}{4x^3-8}\]
\[F'(x)=\frac{12x^2(x^4-8x-2)}{16(x^3-2)^2}\]

If \(F\) is contractive on a real interval \([a,b]\) and
\(F([a,b]) \subset [a,b]\) then the sequence \(x_n\) defined by fixed
point iteration with \(F\) converges and the error satisfies
\[|e_n| \leq \lambda^n |e_0|.\] Then the method is convergent.

    We first calculate where \(F\) is contractive i.e.
\(-1\leq F'(x)\leq 1\) We plot F'(x) below.

    \begin{Verbatim}[commandchars=\\\{\}]
{\color{incolor}In [{\color{incolor}51}]:} \PY{k+kn}{import} \PY{n+nn}{math}
         \PY{k+kn}{import} \PY{n+nn}{numpy} \PY{k}{as} \PY{n+nn}{np}
         \PY{k+kn}{import} \PY{n+nn}{matplotlib}\PY{n+nn}{.}\PY{n+nn}{pyplot} \PY{k}{as} \PY{n+nn}{plt}
         
         \PY{n}{Fp}  \PY{o}{=} \PY{k}{lambda} \PY{n}{x}\PY{p}{:} \PY{l+m+mi}{12}\PY{o}{*}\PY{n}{x}\PY{o}{*}\PY{o}{*}\PY{l+m+mi}{2}\PY{o}{*}\PY{p}{(}\PY{n}{x}\PY{o}{*}\PY{o}{*}\PY{l+m+mi}{4}\PY{o}{\PYZhy{}}\PY{l+m+mi}{8}\PY{o}{*}\PY{n}{x}\PY{o}{\PYZhy{}}\PY{l+m+mi}{2}\PY{p}{)}\PY{o}{/}\PY{p}{(}\PY{l+m+mi}{16}\PY{o}{*}\PY{p}{(}\PY{n}{x}\PY{o}{*}\PY{o}{*}\PY{l+m+mi}{3}\PY{o}{\PYZhy{}}\PY{l+m+mi}{2}\PY{p}{)}\PY{o}{*}\PY{o}{*}\PY{l+m+mi}{2}\PY{p}{)}
         \PY{n}{xg}  \PY{o}{=} \PY{n}{np}\PY{o}{.}\PY{n}{linspace}\PY{p}{(}\PY{o}{\PYZhy{}}\PY{l+m+mi}{20}\PY{p}{,}\PY{l+m+mi}{20}\PY{p}{,}\PY{l+m+mi}{500}\PY{p}{)}
         \PY{n}{Fpxg} \PY{o}{=} \PY{n}{Fp}\PY{p}{(}\PY{n}{xg}\PY{p}{)}
         \PY{n}{plt}\PY{o}{.}\PY{n}{figure}\PY{p}{(}\PY{l+m+mi}{1}\PY{p}{)}
         \PY{n}{plt}\PY{o}{.}\PY{n}{plot}\PY{p}{(}\PY{n}{xg}\PY{p}{,}\PY{n}{Fpxg}\PY{p}{,} \PY{l+s+s1}{\PYZsq{}}\PY{l+s+s1}{r}\PY{l+s+s1}{\PYZsq{}}\PY{p}{)}
         \PY{n}{plt}\PY{o}{.}\PY{n}{axis}\PY{p}{(}\PY{p}{[}\PY{o}{\PYZhy{}}\PY{l+m+mi}{20}\PY{p}{,}\PY{l+m+mi}{20}\PY{p}{,}\PY{o}{\PYZhy{}}\PY{l+m+mi}{2}\PY{p}{,}\PY{l+m+mi}{1}\PY{p}{]}\PY{p}{)}\PY{p}{;} \PY{n}{plt}\PY{o}{.}\PY{n}{grid}\PY{p}{(}\PY{l+s+s1}{\PYZsq{}}\PY{l+s+s1}{on}\PY{l+s+s1}{\PYZsq{}}\PY{p}{)}
         
         \PY{n}{plt}\PY{o}{.}\PY{n}{figure}\PY{p}{(}\PY{l+m+mi}{2}\PY{p}{)}
         \PY{n}{plt}\PY{o}{.}\PY{n}{plot}\PY{p}{(}\PY{n}{xg}\PY{p}{,}\PY{n}{Fpxg}\PY{p}{,} \PY{l+s+s1}{\PYZsq{}}\PY{l+s+s1}{r}\PY{l+s+s1}{\PYZsq{}}\PY{p}{)}
         \PY{n}{plt}\PY{o}{.}\PY{n}{axis}\PY{p}{(}\PY{p}{[}\PY{l+m+mi}{0}\PY{p}{,}\PY{l+m+mi}{2}\PY{p}{,}\PY{o}{\PYZhy{}}\PY{l+m+mi}{2}\PY{p}{,}\PY{l+m+mi}{1}\PY{p}{]}\PY{p}{)}\PY{p}{;} \PY{n}{plt}\PY{o}{.}\PY{n}{grid}\PY{p}{(}\PY{l+s+s1}{\PYZsq{}}\PY{l+s+s1}{on}\PY{l+s+s1}{\PYZsq{}}\PY{p}{)}
\end{Verbatim}


    \begin{Verbatim}[commandchars=\\\{\}]
C:\textbackslash{}Anaconda3\textbackslash{}lib\textbackslash{}site-packages\textbackslash{}matplotlib\textbackslash{}cbook\textbackslash{}deprecation.py:107: MatplotlibDeprecationWarning: Passing one of 'on', 'true', 'off', 'false' as a boolean is deprecated; use an actual boolean (True/False) instead.
  warnings.warn(message, mplDeprecation, stacklevel=1)

    \end{Verbatim}

    \begin{center}
    \adjustimage{max size={0.9\linewidth}{0.9\paperheight}}{output_15_1.png}
    \end{center}
    { \hspace*{\fill} \\}
    
    \begin{center}
    \adjustimage{max size={0.9\linewidth}{0.9\paperheight}}{output_15_2.png}
    \end{center}
    { \hspace*{\fill} \\}
    
    In the limiting cases \$x \rightarrow \pm \infty \$,
\(F'(x)\rightarrow \frac{3}{4}\)

Hence from the plot above we know that only the points in \([x_1, x_2]\)
does not satisfy \(-1\leq F'(x)\leq 1\), where x\(_1\) x\(_2\) are the
roots of \(F'(x)=-1\)

Calculation from mathematica tells us that the approximate roots is
0.703 and 1.796

The points in \([0.703, 1.796]\) are not convergent.

    Next we show that the points in \([-\infty, 0.703]\) and
\([1.796,\infty]\) are convergent.

we will show that \(F(x)= x - f(x)/f^\prime(x)\) maps the two intervals
into itself

    \begin{Verbatim}[commandchars=\\\{\}]
{\color{incolor}In [{\color{incolor}52}]:} \PY{n}{f}  \PY{o}{=} \PY{k}{lambda} \PY{n}{x}\PY{p}{:} \PY{n}{x}\PY{o}{*}\PY{o}{*}\PY{l+m+mi}{4}\PY{o}{\PYZhy{}}\PY{l+m+mi}{8}\PY{o}{*}\PY{n}{x}\PY{o}{\PYZhy{}}\PY{l+m+mi}{2}
         \PY{n}{fp} \PY{o}{=} \PY{k}{lambda} \PY{n}{x}\PY{p}{:} \PY{l+m+mi}{4}\PY{o}{*}\PY{n}{x}\PY{o}{*}\PY{o}{*}\PY{l+m+mi}{3}\PY{o}{\PYZhy{}}\PY{l+m+mi}{8}
         \PY{n}{F}  \PY{o}{=} \PY{k}{lambda} \PY{n}{x}\PY{p}{:} \PY{n}{x} \PY{o}{\PYZhy{}} \PY{n}{f}\PY{p}{(}\PY{n}{x}\PY{p}{)}\PY{o}{/}\PY{n}{fp}\PY{p}{(}\PY{n}{x}\PY{p}{)}
         
         \PY{n}{xg}  \PY{o}{=} \PY{n}{np}\PY{o}{.}\PY{n}{linspace}\PY{p}{(}\PY{o}{\PYZhy{}}\PY{l+m+mi}{20}\PY{p}{,}\PY{l+m+mi}{20}\PY{p}{,}\PY{l+m+mi}{500}\PY{p}{)}
         \PY{n}{Fxg} \PY{o}{=} \PY{n}{F}\PY{p}{(}\PY{n}{xg}\PY{p}{)}
\end{Verbatim}


    \begin{Verbatim}[commandchars=\\\{\}]
{\color{incolor}In [{\color{incolor}55}]:} \PY{n}{plt}\PY{o}{.}\PY{n}{figure}\PY{p}{(}\PY{l+m+mi}{1}\PY{p}{)}\PY{p}{;}\PY{n}{plt}\PY{o}{.}\PY{n}{plot}\PY{p}{(}\PY{n}{xg}\PY{p}{,}\PY{n}{Fxg}\PY{p}{,} \PY{n}{xg}\PY{p}{,}\PY{n}{xg}\PY{p}{,} \PY{l+s+s1}{\PYZsq{}}\PY{l+s+s1}{r}\PY{l+s+s1}{\PYZsq{}}\PY{p}{)}\PY{p}{;} \PY{n}{plt}\PY{o}{.}\PY{n}{axis}\PY{p}{(}\PY{p}{[}\PY{o}{\PYZhy{}}\PY{l+m+mi}{20}\PY{p}{,}\PY{l+m+mf}{0.703}\PY{p}{,}\PY{o}{\PYZhy{}}\PY{l+m+mi}{20}\PY{p}{,}\PY{l+m+mf}{0.703}\PY{p}{]}\PY{p}{)}\PY{p}{;} \PY{n}{plt}\PY{o}{.}\PY{n}{grid}\PY{p}{(}\PY{l+s+s1}{\PYZsq{}}\PY{l+s+s1}{on}\PY{l+s+s1}{\PYZsq{}}\PY{p}{)}
         \PY{n}{plt}\PY{o}{.}\PY{n}{figure}\PY{p}{(}\PY{l+m+mi}{2}\PY{p}{)}\PY{p}{;}\PY{n}{plt}\PY{o}{.}\PY{n}{plot}\PY{p}{(}\PY{n}{xg}\PY{p}{,}\PY{n}{Fxg}\PY{p}{,} \PY{n}{xg}\PY{p}{,}\PY{n}{xg}\PY{p}{,} \PY{l+s+s1}{\PYZsq{}}\PY{l+s+s1}{r}\PY{l+s+s1}{\PYZsq{}}\PY{p}{)}\PY{p}{;} \PY{n}{plt}\PY{o}{.}\PY{n}{axis}\PY{p}{(}\PY{p}{[}\PY{l+m+mf}{1.796}\PY{p}{,}\PY{l+m+mi}{20}\PY{p}{,}\PY{l+m+mf}{1.796}\PY{p}{,}\PY{l+m+mi}{20}\PY{p}{]}\PY{p}{)}\PY{p}{;} \PY{n}{plt}\PY{o}{.}\PY{n}{grid}\PY{p}{(}\PY{l+s+s1}{\PYZsq{}}\PY{l+s+s1}{on}\PY{l+s+s1}{\PYZsq{}}\PY{p}{)}
\end{Verbatim}


    \begin{Verbatim}[commandchars=\\\{\}]
C:\textbackslash{}Anaconda3\textbackslash{}lib\textbackslash{}site-packages\textbackslash{}matplotlib\textbackslash{}cbook\textbackslash{}deprecation.py:107: MatplotlibDeprecationWarning: Passing one of 'on', 'true', 'off', 'false' as a boolean is deprecated; use an actual boolean (True/False) instead.
  warnings.warn(message, mplDeprecation, stacklevel=1)

    \end{Verbatim}

    \begin{center}
    \adjustimage{max size={0.9\linewidth}{0.9\paperheight}}{output_19_1.png}
    \end{center}
    { \hspace*{\fill} \\}
    
    \begin{center}
    \adjustimage{max size={0.9\linewidth}{0.9\paperheight}}{output_19_2.png}
    \end{center}
    { \hspace*{\fill} \\}
    
    So the Newton's method for \(f(x) = x^4 − 8x − 2\) does not converge for
any point.

The points in \([0.703, 1.796]\) are not convergent.


    % Add a bibliography block to the postdoc
    
    
    
    \end{document}
