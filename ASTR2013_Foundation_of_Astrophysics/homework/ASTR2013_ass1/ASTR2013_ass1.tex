
% Default to the notebook output style

    


% Inherit from the specified cell style.




    
\documentclass[11pt]{article}

    
    
    \usepackage[T1]{fontenc}
    % Nicer default font (+ math font) than Computer Modern for most use cases
    \usepackage{mathpazo}

    % Basic figure setup, for now with no caption control since it's done
    % automatically by Pandoc (which extracts ![](path) syntax from Markdown).
    \usepackage{graphicx}
    % We will generate all images so they have a width \maxwidth. This means
    % that they will get their normal width if they fit onto the page, but
    % are scaled down if they would overflow the margins.
    \makeatletter
    \def\maxwidth{\ifdim\Gin@nat@width>\linewidth\linewidth
    \else\Gin@nat@width\fi}
    \makeatother
    \let\Oldincludegraphics\includegraphics
    % Set max figure width to be 80% of text width, for now hardcoded.
    \renewcommand{\includegraphics}[1]{\Oldincludegraphics[width=.8\maxwidth]{#1}}
    % Ensure that by default, figures have no caption (until we provide a
    % proper Figure object with a Caption API and a way to capture that
    % in the conversion process - todo).
    \usepackage{caption}
    \DeclareCaptionLabelFormat{nolabel}{}
    \captionsetup{labelformat=nolabel}

    \usepackage{adjustbox} % Used to constrain images to a maximum size 
    \usepackage{xcolor} % Allow colors to be defined
    \usepackage{enumerate} % Needed for markdown enumerations to work
    \usepackage{geometry} % Used to adjust the document margins
    \usepackage{amsmath} % Equations
    \usepackage{amssymb} % Equations
    \usepackage{textcomp} % defines textquotesingle
    % Hack from http://tex.stackexchange.com/a/47451/13684:
    \AtBeginDocument{%
        \def\PYZsq{\textquotesingle}% Upright quotes in Pygmentized code
    }
    \usepackage{upquote} % Upright quotes for verbatim code
    \usepackage{eurosym} % defines \euro
    \usepackage[mathletters]{ucs} % Extended unicode (utf-8) support
    \usepackage[utf8x]{inputenc} % Allow utf-8 characters in the tex document
    \usepackage{fancyvrb} % verbatim replacement that allows latex
    \usepackage{grffile} % extends the file name processing of package graphics 
                         % to support a larger range 
    % The hyperref package gives us a pdf with properly built
    % internal navigation ('pdf bookmarks' for the table of contents,
    % internal cross-reference links, web links for URLs, etc.)
    \usepackage{hyperref}
    \usepackage{longtable} % longtable support required by pandoc >1.10
    \usepackage{booktabs}  % table support for pandoc > 1.12.2
    \usepackage[inline]{enumitem} % IRkernel/repr support (it uses the enumerate* environment)
    \usepackage[normalem]{ulem} % ulem is needed to support strikethroughs (\sout)
                                % normalem makes italics be italics, not underlines
    \usepackage{mathrsfs}
    

    
    
    % Colors for the hyperref package
    \definecolor{urlcolor}{rgb}{0,.145,.698}
    \definecolor{linkcolor}{rgb}{.71,0.21,0.01}
    \definecolor{citecolor}{rgb}{.12,.54,.11}

    % ANSI colors
    \definecolor{ansi-black}{HTML}{3E424D}
    \definecolor{ansi-black-intense}{HTML}{282C36}
    \definecolor{ansi-red}{HTML}{E75C58}
    \definecolor{ansi-red-intense}{HTML}{B22B31}
    \definecolor{ansi-green}{HTML}{00A250}
    \definecolor{ansi-green-intense}{HTML}{007427}
    \definecolor{ansi-yellow}{HTML}{DDB62B}
    \definecolor{ansi-yellow-intense}{HTML}{B27D12}
    \definecolor{ansi-blue}{HTML}{208FFB}
    \definecolor{ansi-blue-intense}{HTML}{0065CA}
    \definecolor{ansi-magenta}{HTML}{D160C4}
    \definecolor{ansi-magenta-intense}{HTML}{A03196}
    \definecolor{ansi-cyan}{HTML}{60C6C8}
    \definecolor{ansi-cyan-intense}{HTML}{258F8F}
    \definecolor{ansi-white}{HTML}{C5C1B4}
    \definecolor{ansi-white-intense}{HTML}{A1A6B2}
    \definecolor{ansi-default-inverse-fg}{HTML}{FFFFFF}
    \definecolor{ansi-default-inverse-bg}{HTML}{000000}

    % commands and environments needed by pandoc snippets
    % extracted from the output of `pandoc -s`
    \providecommand{\tightlist}{%
      \setlength{\itemsep}{0pt}\setlength{\parskip}{0pt}}
    \DefineVerbatimEnvironment{Highlighting}{Verbatim}{commandchars=\\\{\}}
    % Add ',fontsize=\small' for more characters per line
    \newenvironment{Shaded}{}{}
    \newcommand{\KeywordTok}[1]{\textcolor[rgb]{0.00,0.44,0.13}{\textbf{{#1}}}}
    \newcommand{\DataTypeTok}[1]{\textcolor[rgb]{0.56,0.13,0.00}{{#1}}}
    \newcommand{\DecValTok}[1]{\textcolor[rgb]{0.25,0.63,0.44}{{#1}}}
    \newcommand{\BaseNTok}[1]{\textcolor[rgb]{0.25,0.63,0.44}{{#1}}}
    \newcommand{\FloatTok}[1]{\textcolor[rgb]{0.25,0.63,0.44}{{#1}}}
    \newcommand{\CharTok}[1]{\textcolor[rgb]{0.25,0.44,0.63}{{#1}}}
    \newcommand{\StringTok}[1]{\textcolor[rgb]{0.25,0.44,0.63}{{#1}}}
    \newcommand{\CommentTok}[1]{\textcolor[rgb]{0.38,0.63,0.69}{\textit{{#1}}}}
    \newcommand{\OtherTok}[1]{\textcolor[rgb]{0.00,0.44,0.13}{{#1}}}
    \newcommand{\AlertTok}[1]{\textcolor[rgb]{1.00,0.00,0.00}{\textbf{{#1}}}}
    \newcommand{\FunctionTok}[1]{\textcolor[rgb]{0.02,0.16,0.49}{{#1}}}
    \newcommand{\RegionMarkerTok}[1]{{#1}}
    \newcommand{\ErrorTok}[1]{\textcolor[rgb]{1.00,0.00,0.00}{\textbf{{#1}}}}
    \newcommand{\NormalTok}[1]{{#1}}
    
    % Additional commands for more recent versions of Pandoc
    \newcommand{\ConstantTok}[1]{\textcolor[rgb]{0.53,0.00,0.00}{{#1}}}
    \newcommand{\SpecialCharTok}[1]{\textcolor[rgb]{0.25,0.44,0.63}{{#1}}}
    \newcommand{\VerbatimStringTok}[1]{\textcolor[rgb]{0.25,0.44,0.63}{{#1}}}
    \newcommand{\SpecialStringTok}[1]{\textcolor[rgb]{0.73,0.40,0.53}{{#1}}}
    \newcommand{\ImportTok}[1]{{#1}}
    \newcommand{\DocumentationTok}[1]{\textcolor[rgb]{0.73,0.13,0.13}{\textit{{#1}}}}
    \newcommand{\AnnotationTok}[1]{\textcolor[rgb]{0.38,0.63,0.69}{\textbf{\textit{{#1}}}}}
    \newcommand{\CommentVarTok}[1]{\textcolor[rgb]{0.38,0.63,0.69}{\textbf{\textit{{#1}}}}}
    \newcommand{\VariableTok}[1]{\textcolor[rgb]{0.10,0.09,0.49}{{#1}}}
    \newcommand{\ControlFlowTok}[1]{\textcolor[rgb]{0.00,0.44,0.13}{\textbf{{#1}}}}
    \newcommand{\OperatorTok}[1]{\textcolor[rgb]{0.40,0.40,0.40}{{#1}}}
    \newcommand{\BuiltInTok}[1]{{#1}}
    \newcommand{\ExtensionTok}[1]{{#1}}
    \newcommand{\PreprocessorTok}[1]{\textcolor[rgb]{0.74,0.48,0.00}{{#1}}}
    \newcommand{\AttributeTok}[1]{\textcolor[rgb]{0.49,0.56,0.16}{{#1}}}
    \newcommand{\InformationTok}[1]{\textcolor[rgb]{0.38,0.63,0.69}{\textbf{\textit{{#1}}}}}
    \newcommand{\WarningTok}[1]{\textcolor[rgb]{0.38,0.63,0.69}{\textbf{\textit{{#1}}}}}
    
    
    % Define a nice break command that doesn't care if a line doesn't already
    % exist.
    \def\br{\hspace*{\fill} \\* }
    % Math Jax compatibility definitions
    \def\gt{>}
    \def\lt{<}
    \let\Oldtex\TeX
    \let\Oldlatex\LaTeX
    \renewcommand{\TeX}{\textrm{\Oldtex}}
    \renewcommand{\LaTeX}{\textrm{\Oldlatex}}
    % Document parameters
    % Document title
    \title{ASTR2013\_ass1}
    
    
    
    
    

    % Pygments definitions
    
\makeatletter
\def\PY@reset{\let\PY@it=\relax \let\PY@bf=\relax%
    \let\PY@ul=\relax \let\PY@tc=\relax%
    \let\PY@bc=\relax \let\PY@ff=\relax}
\def\PY@tok#1{\csname PY@tok@#1\endcsname}
\def\PY@toks#1+{\ifx\relax#1\empty\else%
    \PY@tok{#1}\expandafter\PY@toks\fi}
\def\PY@do#1{\PY@bc{\PY@tc{\PY@ul{%
    \PY@it{\PY@bf{\PY@ff{#1}}}}}}}
\def\PY#1#2{\PY@reset\PY@toks#1+\relax+\PY@do{#2}}

\expandafter\def\csname PY@tok@w\endcsname{\def\PY@tc##1{\textcolor[rgb]{0.73,0.73,0.73}{##1}}}
\expandafter\def\csname PY@tok@c\endcsname{\let\PY@it=\textit\def\PY@tc##1{\textcolor[rgb]{0.25,0.50,0.50}{##1}}}
\expandafter\def\csname PY@tok@cp\endcsname{\def\PY@tc##1{\textcolor[rgb]{0.74,0.48,0.00}{##1}}}
\expandafter\def\csname PY@tok@k\endcsname{\let\PY@bf=\textbf\def\PY@tc##1{\textcolor[rgb]{0.00,0.50,0.00}{##1}}}
\expandafter\def\csname PY@tok@kp\endcsname{\def\PY@tc##1{\textcolor[rgb]{0.00,0.50,0.00}{##1}}}
\expandafter\def\csname PY@tok@kt\endcsname{\def\PY@tc##1{\textcolor[rgb]{0.69,0.00,0.25}{##1}}}
\expandafter\def\csname PY@tok@o\endcsname{\def\PY@tc##1{\textcolor[rgb]{0.40,0.40,0.40}{##1}}}
\expandafter\def\csname PY@tok@ow\endcsname{\let\PY@bf=\textbf\def\PY@tc##1{\textcolor[rgb]{0.67,0.13,1.00}{##1}}}
\expandafter\def\csname PY@tok@nb\endcsname{\def\PY@tc##1{\textcolor[rgb]{0.00,0.50,0.00}{##1}}}
\expandafter\def\csname PY@tok@nf\endcsname{\def\PY@tc##1{\textcolor[rgb]{0.00,0.00,1.00}{##1}}}
\expandafter\def\csname PY@tok@nc\endcsname{\let\PY@bf=\textbf\def\PY@tc##1{\textcolor[rgb]{0.00,0.00,1.00}{##1}}}
\expandafter\def\csname PY@tok@nn\endcsname{\let\PY@bf=\textbf\def\PY@tc##1{\textcolor[rgb]{0.00,0.00,1.00}{##1}}}
\expandafter\def\csname PY@tok@ne\endcsname{\let\PY@bf=\textbf\def\PY@tc##1{\textcolor[rgb]{0.82,0.25,0.23}{##1}}}
\expandafter\def\csname PY@tok@nv\endcsname{\def\PY@tc##1{\textcolor[rgb]{0.10,0.09,0.49}{##1}}}
\expandafter\def\csname PY@tok@no\endcsname{\def\PY@tc##1{\textcolor[rgb]{0.53,0.00,0.00}{##1}}}
\expandafter\def\csname PY@tok@nl\endcsname{\def\PY@tc##1{\textcolor[rgb]{0.63,0.63,0.00}{##1}}}
\expandafter\def\csname PY@tok@ni\endcsname{\let\PY@bf=\textbf\def\PY@tc##1{\textcolor[rgb]{0.60,0.60,0.60}{##1}}}
\expandafter\def\csname PY@tok@na\endcsname{\def\PY@tc##1{\textcolor[rgb]{0.49,0.56,0.16}{##1}}}
\expandafter\def\csname PY@tok@nt\endcsname{\let\PY@bf=\textbf\def\PY@tc##1{\textcolor[rgb]{0.00,0.50,0.00}{##1}}}
\expandafter\def\csname PY@tok@nd\endcsname{\def\PY@tc##1{\textcolor[rgb]{0.67,0.13,1.00}{##1}}}
\expandafter\def\csname PY@tok@s\endcsname{\def\PY@tc##1{\textcolor[rgb]{0.73,0.13,0.13}{##1}}}
\expandafter\def\csname PY@tok@sd\endcsname{\let\PY@it=\textit\def\PY@tc##1{\textcolor[rgb]{0.73,0.13,0.13}{##1}}}
\expandafter\def\csname PY@tok@si\endcsname{\let\PY@bf=\textbf\def\PY@tc##1{\textcolor[rgb]{0.73,0.40,0.53}{##1}}}
\expandafter\def\csname PY@tok@se\endcsname{\let\PY@bf=\textbf\def\PY@tc##1{\textcolor[rgb]{0.73,0.40,0.13}{##1}}}
\expandafter\def\csname PY@tok@sr\endcsname{\def\PY@tc##1{\textcolor[rgb]{0.73,0.40,0.53}{##1}}}
\expandafter\def\csname PY@tok@ss\endcsname{\def\PY@tc##1{\textcolor[rgb]{0.10,0.09,0.49}{##1}}}
\expandafter\def\csname PY@tok@sx\endcsname{\def\PY@tc##1{\textcolor[rgb]{0.00,0.50,0.00}{##1}}}
\expandafter\def\csname PY@tok@m\endcsname{\def\PY@tc##1{\textcolor[rgb]{0.40,0.40,0.40}{##1}}}
\expandafter\def\csname PY@tok@gh\endcsname{\let\PY@bf=\textbf\def\PY@tc##1{\textcolor[rgb]{0.00,0.00,0.50}{##1}}}
\expandafter\def\csname PY@tok@gu\endcsname{\let\PY@bf=\textbf\def\PY@tc##1{\textcolor[rgb]{0.50,0.00,0.50}{##1}}}
\expandafter\def\csname PY@tok@gd\endcsname{\def\PY@tc##1{\textcolor[rgb]{0.63,0.00,0.00}{##1}}}
\expandafter\def\csname PY@tok@gi\endcsname{\def\PY@tc##1{\textcolor[rgb]{0.00,0.63,0.00}{##1}}}
\expandafter\def\csname PY@tok@gr\endcsname{\def\PY@tc##1{\textcolor[rgb]{1.00,0.00,0.00}{##1}}}
\expandafter\def\csname PY@tok@ge\endcsname{\let\PY@it=\textit}
\expandafter\def\csname PY@tok@gs\endcsname{\let\PY@bf=\textbf}
\expandafter\def\csname PY@tok@gp\endcsname{\let\PY@bf=\textbf\def\PY@tc##1{\textcolor[rgb]{0.00,0.00,0.50}{##1}}}
\expandafter\def\csname PY@tok@go\endcsname{\def\PY@tc##1{\textcolor[rgb]{0.53,0.53,0.53}{##1}}}
\expandafter\def\csname PY@tok@gt\endcsname{\def\PY@tc##1{\textcolor[rgb]{0.00,0.27,0.87}{##1}}}
\expandafter\def\csname PY@tok@err\endcsname{\def\PY@bc##1{\setlength{\fboxsep}{0pt}\fcolorbox[rgb]{1.00,0.00,0.00}{1,1,1}{\strut ##1}}}
\expandafter\def\csname PY@tok@kc\endcsname{\let\PY@bf=\textbf\def\PY@tc##1{\textcolor[rgb]{0.00,0.50,0.00}{##1}}}
\expandafter\def\csname PY@tok@kd\endcsname{\let\PY@bf=\textbf\def\PY@tc##1{\textcolor[rgb]{0.00,0.50,0.00}{##1}}}
\expandafter\def\csname PY@tok@kn\endcsname{\let\PY@bf=\textbf\def\PY@tc##1{\textcolor[rgb]{0.00,0.50,0.00}{##1}}}
\expandafter\def\csname PY@tok@kr\endcsname{\let\PY@bf=\textbf\def\PY@tc##1{\textcolor[rgb]{0.00,0.50,0.00}{##1}}}
\expandafter\def\csname PY@tok@bp\endcsname{\def\PY@tc##1{\textcolor[rgb]{0.00,0.50,0.00}{##1}}}
\expandafter\def\csname PY@tok@fm\endcsname{\def\PY@tc##1{\textcolor[rgb]{0.00,0.00,1.00}{##1}}}
\expandafter\def\csname PY@tok@vc\endcsname{\def\PY@tc##1{\textcolor[rgb]{0.10,0.09,0.49}{##1}}}
\expandafter\def\csname PY@tok@vg\endcsname{\def\PY@tc##1{\textcolor[rgb]{0.10,0.09,0.49}{##1}}}
\expandafter\def\csname PY@tok@vi\endcsname{\def\PY@tc##1{\textcolor[rgb]{0.10,0.09,0.49}{##1}}}
\expandafter\def\csname PY@tok@vm\endcsname{\def\PY@tc##1{\textcolor[rgb]{0.10,0.09,0.49}{##1}}}
\expandafter\def\csname PY@tok@sa\endcsname{\def\PY@tc##1{\textcolor[rgb]{0.73,0.13,0.13}{##1}}}
\expandafter\def\csname PY@tok@sb\endcsname{\def\PY@tc##1{\textcolor[rgb]{0.73,0.13,0.13}{##1}}}
\expandafter\def\csname PY@tok@sc\endcsname{\def\PY@tc##1{\textcolor[rgb]{0.73,0.13,0.13}{##1}}}
\expandafter\def\csname PY@tok@dl\endcsname{\def\PY@tc##1{\textcolor[rgb]{0.73,0.13,0.13}{##1}}}
\expandafter\def\csname PY@tok@s2\endcsname{\def\PY@tc##1{\textcolor[rgb]{0.73,0.13,0.13}{##1}}}
\expandafter\def\csname PY@tok@sh\endcsname{\def\PY@tc##1{\textcolor[rgb]{0.73,0.13,0.13}{##1}}}
\expandafter\def\csname PY@tok@s1\endcsname{\def\PY@tc##1{\textcolor[rgb]{0.73,0.13,0.13}{##1}}}
\expandafter\def\csname PY@tok@mb\endcsname{\def\PY@tc##1{\textcolor[rgb]{0.40,0.40,0.40}{##1}}}
\expandafter\def\csname PY@tok@mf\endcsname{\def\PY@tc##1{\textcolor[rgb]{0.40,0.40,0.40}{##1}}}
\expandafter\def\csname PY@tok@mh\endcsname{\def\PY@tc##1{\textcolor[rgb]{0.40,0.40,0.40}{##1}}}
\expandafter\def\csname PY@tok@mi\endcsname{\def\PY@tc##1{\textcolor[rgb]{0.40,0.40,0.40}{##1}}}
\expandafter\def\csname PY@tok@il\endcsname{\def\PY@tc##1{\textcolor[rgb]{0.40,0.40,0.40}{##1}}}
\expandafter\def\csname PY@tok@mo\endcsname{\def\PY@tc##1{\textcolor[rgb]{0.40,0.40,0.40}{##1}}}
\expandafter\def\csname PY@tok@ch\endcsname{\let\PY@it=\textit\def\PY@tc##1{\textcolor[rgb]{0.25,0.50,0.50}{##1}}}
\expandafter\def\csname PY@tok@cm\endcsname{\let\PY@it=\textit\def\PY@tc##1{\textcolor[rgb]{0.25,0.50,0.50}{##1}}}
\expandafter\def\csname PY@tok@cpf\endcsname{\let\PY@it=\textit\def\PY@tc##1{\textcolor[rgb]{0.25,0.50,0.50}{##1}}}
\expandafter\def\csname PY@tok@c1\endcsname{\let\PY@it=\textit\def\PY@tc##1{\textcolor[rgb]{0.25,0.50,0.50}{##1}}}
\expandafter\def\csname PY@tok@cs\endcsname{\let\PY@it=\textit\def\PY@tc##1{\textcolor[rgb]{0.25,0.50,0.50}{##1}}}

\def\PYZbs{\char`\\}
\def\PYZus{\char`\_}
\def\PYZob{\char`\{}
\def\PYZcb{\char`\}}
\def\PYZca{\char`\^}
\def\PYZam{\char`\&}
\def\PYZlt{\char`\<}
\def\PYZgt{\char`\>}
\def\PYZsh{\char`\#}
\def\PYZpc{\char`\%}
\def\PYZdl{\char`\$}
\def\PYZhy{\char`\-}
\def\PYZsq{\char`\'}
\def\PYZdq{\char`\"}
\def\PYZti{\char`\~}
% for compatibility with earlier versions
\def\PYZat{@}
\def\PYZlb{[}
\def\PYZrb{]}
\makeatother


    % Exact colors from NB
    \definecolor{incolor}{rgb}{0.0, 0.0, 0.5}
    \definecolor{outcolor}{rgb}{0.545, 0.0, 0.0}



    
    % Prevent overflowing lines due to hard-to-break entities
    \sloppy 
    % Setup hyperref package
    \hypersetup{
      breaklinks=true,  % so long urls are correctly broken across lines
      colorlinks=true,
      urlcolor=urlcolor,
      linkcolor=linkcolor,
      citecolor=citecolor,
      }
    % Slightly bigger margins than the latex defaults
    
    \geometry{verbose,tmargin=1in,bmargin=1in,lmargin=1in,rmargin=1in}
    
    

    \begin{document}
    
    
    \maketitle
    
    

    
    \hypertarget{q1}{%
\section{Q1}\label{q1}}

    From chain rule we get: \[
f_{\nu}=\frac{d f}{d \nu}=\frac{d f}{d \lambda} \frac{d \lambda}{d \nu}=-\frac{d \lambda}{d \nu} f_{\lambda}
\]

where we use the minus sign for correcting for making \(f_\lambda\)
positive.

As \[
\lambda=\frac{c}{\nu} \quad \frac{d \lambda}{d \nu}=-\frac{c}{\nu^{2}}
\]

We get the conversion formula: \[
f_{\nu}=\frac{c}{\nu^{2}} f_{\lambda}=\frac{\lambda^{2}}{c} f_{\lambda}
\]

    \hypertarget{q2}{%
\section{Q2}\label{q2}}

    \textbf{(a)}

    \begin{Verbatim}[commandchars=\\\{\}]
{\color{incolor}In [{\color{incolor}1}]:} \PY{k+kn}{import} \PY{n+nn}{astropy}\PY{n+nn}{.}\PY{n+nn}{constants} \PY{k}{as} \PY{n+nn}{c}\PY{o}{,} \PY{n+nn}{astropy}\PY{n+nn}{.}\PY{n+nn}{units} \PY{k}{as} \PY{n+nn}{u}
        \PY{k+kn}{import} \PY{n+nn}{numpy} \PY{k}{as} \PY{n+nn}{np}\PY{o}{,} \PY{n+nn}{matplotlib}\PY{n+nn}{.}\PY{n+nn}{pyplot} \PY{k}{as} \PY{n+nn}{plt}
\end{Verbatim}

    \begin{Verbatim}[commandchars=\\\{\}]
{\color{incolor}In [{\color{incolor}2}]:} \PY{n}{lambd\PYZus{}1} \PY{o}{=} \PY{l+m+mi}{440}\PY{o}{*}\PY{n}{u}\PY{o}{.}\PY{n}{nm}
        \PY{n}{lambd\PYZus{}2} \PY{o}{=} \PY{l+m+mi}{550}\PY{o}{*}\PY{n}{u}\PY{o}{.}\PY{n}{nm}
        \PY{n}{T} \PY{o}{=} \PY{n}{np}\PY{o}{.}\PY{n}{logspace}\PY{p}{(}\PY{n}{np}\PY{o}{.}\PY{n}{log10}\PY{p}{(}\PY{l+m+mi}{3000}\PY{p}{)}\PY{p}{,}\PY{n}{np}\PY{o}{.}\PY{n}{log10}\PY{p}{(}\PY{l+m+mi}{30000}\PY{p}{)}\PY{p}{,}\PY{l+m+mi}{100}\PY{p}{)}\PY{o}{*}\PY{n}{u}\PY{o}{.}\PY{n}{K}
\end{Verbatim}

    \begin{Verbatim}[commandchars=\\\{\}]
{\color{incolor}In [{\color{incolor}3}]:} \PY{n}{F} \PY{o}{=} \PY{k}{lambda} \PY{n}{lambd}\PY{p}{,} \PY{n}{T}\PY{p}{:} \PY{l+m+mi}{2}\PY{o}{*}\PY{n}{c}\PY{o}{.}\PY{n}{h}\PY{o}{*}\PY{n}{c}\PY{o}{.}\PY{n}{c}\PY{o}{*}\PY{o}{*}\PY{l+m+mi}{2}\PY{o}{/}\PY{n}{lambd}\PY{o}{*}\PY{o}{*}\PY{l+m+mi}{5}\PY{o}{/}\PY{p}{(}\PY{n}{np}\PY{o}{.}\PY{n}{exp}\PY{p}{(}\PY{n}{c}\PY{o}{.}\PY{n}{h}\PY{o}{*}\PY{n}{c}\PY{o}{.}\PY{n}{c}\PY{o}{/}\PY{n}{lambd}\PY{o}{/}\PY{n}{c}\PY{o}{.}\PY{n}{k\PYZus{}B}\PY{o}{/}\PY{n}{T}\PY{p}{)}\PY{o}{\PYZhy{}}\PY{l+m+mi}{1}\PY{p}{)}
        \PY{n}{F\PYZus{}1} \PY{o}{=} \PY{n}{F}\PY{p}{(}\PY{n}{lambd\PYZus{}1}\PY{p}{,} \PY{n}{T}\PY{p}{)}
        \PY{n}{F\PYZus{}2} \PY{o}{=} \PY{n}{F}\PY{p}{(}\PY{n}{lambd\PYZus{}2}\PY{p}{,} \PY{n}{T}\PY{p}{)}
\end{Verbatim}

    Log-log plot of temperature vs measured flux ratio with temperature
varying between 3000 and 30000K.

    \begin{Verbatim}[commandchars=\\\{\}]
{\color{incolor}In [{\color{incolor}4}]:} \PY{n}{plt}\PY{o}{.}\PY{n}{loglog}\PY{p}{(}\PY{n}{F\PYZus{}2}\PY{o}{/}\PY{n}{F\PYZus{}1}\PY{p}{,} \PY{n}{T}\PY{p}{)}
        \PY{n}{plt}\PY{o}{.}\PY{n}{xlabel}\PY{p}{(}\PY{l+s+s1}{\PYZsq{}}\PY{l+s+s1}{flux ratio}\PY{l+s+s1}{\PYZsq{}}\PY{p}{)}\PY{p}{;}\PY{n}{plt}\PY{o}{.}\PY{n}{ylabel}\PY{p}{(}\PY{l+s+s1}{\PYZsq{}}\PY{l+s+s1}{T(K)}\PY{l+s+s1}{\PYZsq{}}\PY{p}{)}\PY{p}{;}\PY{n}{plt}\PY{o}{.}\PY{n}{legend}\PY{p}{(}\PY{p}{)}
        \PY{n}{plt}\PY{o}{.}\PY{n}{axis}\PY{p}{(}\PY{p}{[}\PY{l+m+mf}{4e\PYZhy{}1}\PY{p}{,} \PY{l+m+mf}{3e0}\PY{p}{,}\PY{l+m+mi}{3000}\PY{p}{,}\PY{l+m+mi}{30000}\PY{p}{]}\PY{p}{)}
\end{Verbatim}

    \begin{Verbatim}[commandchars=\\\{\}]
No handles with labels found to put in legend.

    \end{Verbatim}

\begin{Verbatim}[commandchars=\\\{\}]
{\color{outcolor}Out[{\color{outcolor}4}]:} [0.4, 3.0, 3000, 30000]
\end{Verbatim}
            
    \begin{center}
    \adjustimage{max size={0.9\linewidth}{0.9\paperheight}}{output_8_2.png}
    \end{center}
    { \hspace*{\fill} \\}
    
    \textbf{(b)}

    In Wien's limit, flux ratio writes: \[
f_{\lambda}=\frac{2 h c^{2}}{\lambda^{5}} e^{-\frac{h c}{\lambda k T}}
\]

For two wavelength, we get: \[
f_{1}=\frac{2 h c^{2}}{\lambda_{1}^{5}} e^{-\frac{h c}{\lambda_{1} k T}}
\]

\[
f_{2}=\frac{2 h c^{2}}{\lambda_{2}^{5}} e^{-\frac{h c}{\lambda_{2} k T}}
\]

Hence the flux ratio writes: \[
\frac{f_2}{f_1}=\frac{\lambda_{1}^5}{\lambda_{2}^{5}} e^{\frac{h c}{k T}(\frac{1}{\lambda_1}-\frac{1}{\lambda_2})}
\]

After a few algebra, we can get the temperature is given by:

\[
T=\frac{h c}{k_{B}}\left(\frac{1}{\lambda_{1}}-\frac{1}{\lambda_{2}}\right) \frac{1}{\ln \left(f_{2} / f_{1}\right)+5 \ln \left(\lambda_{2} / \lambda_{1}\right)}
\]

    Plot of approximation formula and exact formula

    \begin{Verbatim}[commandchars=\\\{\}]
{\color{incolor}In [{\color{incolor}5}]:} \PY{n}{T\PYZus{}wien} \PY{o}{=} \PY{k}{lambda} \PY{n}{lambd\PYZus{}1}\PY{p}{,} \PY{n}{lambd\PYZus{}2}\PY{p}{,} \PY{n}{f\PYZus{}r}\PY{p}{:} \PY{p}{(}\PY{n}{c}\PY{o}{.}\PY{n}{h}\PY{o}{*}\PY{n}{c}\PY{o}{.}\PY{n}{c}\PY{o}{/}\PY{n}{c}\PY{o}{.}\PY{n}{k\PYZus{}B}\PY{o}{*}\PY{p}{(}\PY{l+m+mi}{1}\PY{o}{/}\PY{n}{lambd\PYZus{}1}\PY{o}{\PYZhy{}}\PY{l+m+mi}{1}\PY{o}{/}\PY{n}{lambd\PYZus{}2}\PY{p}{)}
                                                \PY{o}{/}\PY{p}{(}\PY{n}{np}\PY{o}{.}\PY{n}{log}\PY{p}{(}\PY{n}{f\PYZus{}r}\PY{p}{)}\PY{o}{+}\PY{l+m+mi}{5}\PY{o}{*}\PY{n}{np}\PY{o}{.}\PY{n}{log}\PY{p}{(}\PY{n}{lambd\PYZus{}2}\PY{o}{/}\PY{n}{lambd\PYZus{}1}\PY{p}{)}\PY{p}{)}\PY{p}{)}\PY{o}{.}\PY{n}{to}\PY{p}{(}\PY{n}{u}\PY{o}{.}\PY{n}{K}\PY{p}{)}
\end{Verbatim}

    \begin{Verbatim}[commandchars=\\\{\}]
{\color{incolor}In [{\color{incolor}6}]:} \PY{n}{f\PYZus{}r} \PY{o}{=} \PY{n}{np}\PY{o}{.}\PY{n}{logspace}\PY{p}{(}\PY{o}{\PYZhy{}}\PY{l+m+mi}{1}\PY{p}{,}\PY{l+m+mi}{1}\PY{p}{,}\PY{l+m+mi}{100}\PY{p}{)}
        \PY{n}{T\PYZus{}w} \PY{o}{=} \PY{n}{T\PYZus{}wien}\PY{p}{(}\PY{n}{lambd\PYZus{}1}\PY{p}{,}\PY{n}{lambd\PYZus{}2}\PY{p}{,} \PY{n}{f\PYZus{}r}\PY{p}{)}
        \PY{c+c1}{\PYZsh{}R(lambd\PYZus{}1, lambd\PYZus{}2, T)}
\end{Verbatim}

    \begin{Verbatim}[commandchars=\\\{\}]
{\color{incolor}In [{\color{incolor}7}]:} \PY{n}{plt}\PY{o}{.}\PY{n}{loglog}\PY{p}{(}\PY{n}{F\PYZus{}2}\PY{o}{/}\PY{n}{F\PYZus{}1}\PY{p}{,} \PY{n}{T}\PY{p}{,}\PY{n}{label}\PY{o}{=}\PY{l+s+s2}{\PYZdq{}}\PY{l+s+s2}{exact}\PY{l+s+s2}{\PYZdq{}}\PY{p}{)}
        \PY{n}{plt}\PY{o}{.}\PY{n}{loglog}\PY{p}{(}\PY{n}{f\PYZus{}r}\PY{p}{,} \PY{n}{T\PYZus{}w} \PY{p}{,}\PY{n}{label}\PY{o}{=}\PY{l+s+s2}{\PYZdq{}}\PY{l+s+s2}{Wien}\PY{l+s+s2}{\PYZsq{}}\PY{l+s+s2}{s limit}\PY{l+s+s2}{\PYZdq{}}\PY{p}{)}
        \PY{n}{plt}\PY{o}{.}\PY{n}{xlabel}\PY{p}{(}\PY{l+s+s1}{\PYZsq{}}\PY{l+s+s1}{ratio}\PY{l+s+s1}{\PYZsq{}}\PY{p}{)}\PY{p}{;}\PY{n}{plt}\PY{o}{.}\PY{n}{ylabel}\PY{p}{(}\PY{l+s+s1}{\PYZsq{}}\PY{l+s+s1}{T}\PY{l+s+s1}{\PYZsq{}}\PY{p}{)}\PY{p}{;}\PY{n}{plt}\PY{o}{.}\PY{n}{legend}\PY{p}{(}\PY{p}{)}
        \PY{n}{plt}\PY{o}{.}\PY{n}{axis}\PY{p}{(}\PY{p}{[}\PY{l+m+mf}{4e\PYZhy{}1}\PY{p}{,} \PY{l+m+mf}{3e0}\PY{p}{,} \PY{l+m+mi}{3000}\PY{p}{,}\PY{l+m+mi}{30000}\PY{p}{]}\PY{p}{)}
\end{Verbatim}

\begin{Verbatim}[commandchars=\\\{\}]
{\color{outcolor}Out[{\color{outcolor}7}]:} [0.4, 3.0, 3000, 30000]
\end{Verbatim}
            
    \begin{center}
    \adjustimage{max size={0.9\linewidth}{0.9\paperheight}}{output_14_1.png}
    \end{center}
    { \hspace*{\fill} \\}
    
    we can see that when temperature is high or the flux ratio is low,
wien's limit is not very good.

    \textbf{(c)}

    From Poisson statistics, the uncertainty of measurement is\\
\(\sqrt{N} \approx\sqrt{10000}=100\)

Then precision of measurement \(\approx 100/10000 =0.01\)

precision of flux ratio is approximately
\(\sqrt{0.01^2+0.01^2} \approx 0.01414\)

    we use the following code to illustate the accuracy of temperature
measurement using flux ratio.

At low temperature, we use the wien approimation as it gives a good
approximation to the exact case.

At high temperature, we use intepolation module to calculate the
temperature.

    \begin{Verbatim}[commandchars=\\\{\}]
{\color{incolor}In [{\color{incolor}8}]:} \PY{k+kn}{from} \PY{n+nn}{scipy} \PY{k}{import} \PY{n}{interpolate}
        \PY{n}{err}\PY{o}{=}\PY{n}{np}\PY{o}{.}\PY{n}{sqrt}\PY{p}{(}\PY{l+m+mf}{0.01}\PY{o}{*}\PY{o}{*}\PY{l+m+mi}{2}\PY{o}{+}\PY{l+m+mf}{0.01}\PY{o}{*}\PY{o}{*}\PY{l+m+mi}{2}\PY{p}{)}
        
        \PY{n}{T\PYZus{}ex} \PY{o}{=} \PY{n}{interpolate}\PY{o}{.}\PY{n}{interp1d}\PY{p}{(}\PY{n}{F\PYZus{}2}\PY{o}{/}\PY{n}{F\PYZus{}1}\PY{p}{,}\PY{n}{T}\PY{p}{)}
        
        \PY{n}{T\PYZus{}1} \PY{o}{=} \PY{l+m+mi}{4000}\PY{o}{*}\PY{n}{u}\PY{o}{.}\PY{n}{K}
        \PY{n}{T\PYZus{}2} \PY{o}{=} \PY{l+m+mi}{20000}\PY{o}{*}\PY{n}{u}\PY{o}{.}\PY{n}{K}
        \PY{n}{r\PYZus{}1} \PY{o}{=} \PY{n}{F}\PY{p}{(}\PY{n}{lambd\PYZus{}2}\PY{p}{,} \PY{n}{T\PYZus{}1}\PY{p}{)}\PY{o}{/}\PY{n}{F}\PY{p}{(}\PY{n}{lambd\PYZus{}1}\PY{p}{,} \PY{n}{T\PYZus{}1}\PY{p}{)} 
        \PY{n}{r\PYZus{}2} \PY{o}{=} \PY{n}{F}\PY{p}{(}\PY{n}{lambd\PYZus{}2}\PY{p}{,} \PY{n}{T\PYZus{}2}\PY{p}{)}\PY{o}{/}\PY{n}{F}\PY{p}{(}\PY{n}{lambd\PYZus{}1}\PY{p}{,} \PY{n}{T\PYZus{}2}\PY{p}{)} 
        
        \PY{n}{T\PYZus{}4000\PYZus{}lower} \PY{o}{=} \PY{n}{T\PYZus{}wien}\PY{p}{(}\PY{n}{lambd\PYZus{}1}\PY{p}{,} \PY{n}{lambd\PYZus{}2}\PY{p}{,} \PY{n}{r\PYZus{}1}\PY{o}{*}\PY{p}{(}\PY{l+m+mi}{1}\PY{o}{+}\PY{n}{err}\PY{p}{)}\PY{p}{)}
        \PY{n}{T\PYZus{}4000\PYZus{}higher} \PY{o}{=} \PY{n}{T\PYZus{}wien}\PY{p}{(}\PY{n}{lambd\PYZus{}1}\PY{p}{,} \PY{n}{lambd\PYZus{}2}\PY{p}{,} \PY{n}{r\PYZus{}1}\PY{o}{*}\PY{p}{(}\PY{l+m+mi}{1}\PY{o}{\PYZhy{}}\PY{n}{err}\PY{p}{)}\PY{p}{)}
        
        \PY{n}{T\PYZus{}20000\PYZus{}lower} \PY{o}{=} \PY{n}{T\PYZus{}ex}\PY{p}{(}\PY{n}{r\PYZus{}2}\PY{o}{*}\PY{p}{(}\PY{l+m+mi}{1}\PY{o}{+}\PY{n}{err}\PY{p}{)}\PY{p}{)}\PY{o}{*}\PY{n}{u}\PY{o}{.}\PY{n}{K}
        \PY{n}{T\PYZus{}20000\PYZus{}higher} \PY{o}{=} \PY{n}{T\PYZus{}ex}\PY{p}{(}\PY{n}{r\PYZus{}2}\PY{o}{*}\PY{p}{(}\PY{l+m+mi}{1}\PY{o}{\PYZhy{}}\PY{n}{err}\PY{p}{)}\PY{p}{)}\PY{o}{*}\PY{n}{u}\PY{o}{.}\PY{n}{K}
        
        \PY{n+nb}{print}\PY{p}{(}\PY{l+s+s1}{\PYZsq{}}\PY{l+s+s1}{Temperature measurement of 4000K fall in range between }\PY{l+s+si}{\PYZob{}:.2f\PYZcb{}}\PY{l+s+s1}{ to }\PY{l+s+si}{\PYZob{}:.2f\PYZcb{}}\PY{l+s+s1}{:}\PY{l+s+s1}{\PYZsq{}}
              \PY{o}{.}\PY{n}{format}\PY{p}{(}\PY{n}{T\PYZus{}4000\PYZus{}lower}\PY{p}{,}\PY{n}{T\PYZus{}4000\PYZus{}higher}\PY{p}{)}\PY{p}{)}
        \PY{n+nb}{print}\PY{p}{(}\PY{l+s+s1}{\PYZsq{}}\PY{l+s+s1}{Temperature measurement of 20000K fall in range between }\PY{l+s+si}{\PYZob{}:.2f\PYZcb{}}\PY{l+s+s1}{ to }\PY{l+s+si}{\PYZob{}:.2f\PYZcb{}}\PY{l+s+s1}{:}\PY{l+s+s1}{\PYZsq{}}
              \PY{o}{.}\PY{n}{format}\PY{p}{(}\PY{n}{T\PYZus{}20000\PYZus{}lower}\PY{p}{,}\PY{n}{T\PYZus{}20000\PYZus{}higher}\PY{p}{)}\PY{p}{)}
\end{Verbatim}

    \begin{Verbatim}[commandchars=\\\{\}]
Temperature measurement of 4000K fall in range between 3963.14 K to 4032.26 K:
Temperature measurement of 20000K fall in range between 18896.47 K to 21285.47 K:

    \end{Verbatim}

    We can see that the precision of measurement of 4000K is about
30/4000=0.75\%, whereas for 20000K, the precision is about
1200/20000=6\%. The measurement of 4000K is more precise.

    \hypertarget{q3}{%
\section{Q3}\label{q3}}

    \textbf{(a)}

    We use interpolation to calculate the temperature of star

    \begin{Verbatim}[commandchars=\\\{\}]
{\color{incolor}In [{\color{incolor}9}]:} \PY{n}{lambd\PYZus{}1} \PY{o}{=} \PY{l+m+mi}{532}\PY{o}{*}\PY{n}{u}\PY{o}{.}\PY{n}{nm}
        \PY{n}{lambd\PYZus{}2} \PY{o}{=} \PY{l+m+mi}{797}\PY{o}{*}\PY{n}{u}\PY{o}{.}\PY{n}{nm}
        \PY{n}{T} \PY{o}{=} \PY{n}{np}\PY{o}{.}\PY{n}{logspace}\PY{p}{(}\PY{n}{np}\PY{o}{.}\PY{n}{log10}\PY{p}{(}\PY{l+m+mi}{3000}\PY{p}{)}\PY{p}{,}\PY{n}{np}\PY{o}{.}\PY{n}{log10}\PY{p}{(}\PY{l+m+mi}{10000}\PY{p}{)}\PY{p}{,}\PY{l+m+mi}{100}\PY{p}{)}\PY{o}{*}\PY{n}{u}\PY{o}{.}\PY{n}{K}
        \PY{n}{F\PYZus{}1} \PY{o}{=} \PY{n}{F}\PY{p}{(}\PY{n}{lambd\PYZus{}1}\PY{p}{,} \PY{n}{T}\PY{p}{)}
        \PY{n}{F\PYZus{}2} \PY{o}{=} \PY{n}{F}\PY{p}{(}\PY{n}{lambd\PYZus{}2}\PY{p}{,} \PY{n}{T}\PY{p}{)}
        \PY{n}{T\PYZus{}ex} \PY{o}{=} \PY{n}{interpolate}\PY{o}{.}\PY{n}{interp1d}\PY{p}{(}\PY{n}{F\PYZus{}2}\PY{o}{/}\PY{n}{F\PYZus{}1}\PY{p}{,}\PY{n}{T}\PY{p}{)}
        
        \PY{n}{f\PYZus{}1} \PY{o}{=} \PY{l+m+mf}{5.9e\PYZhy{}13}\PY{o}{*}\PY{n}{u}\PY{o}{.}\PY{n}{erg}\PY{o}{/}\PY{n}{u}\PY{o}{.}\PY{n}{s}\PY{o}{/}\PY{n}{u}\PY{o}{.}\PY{n}{cm}\PY{o}{*}\PY{o}{*}\PY{l+m+mi}{2}\PY{o}{/}\PY{n}{u}\PY{o}{.}\PY{n}{Angstrom}
        \PY{n}{f\PYZus{}2} \PY{o}{=} \PY{l+m+mf}{4.8e\PYZhy{}13}\PY{o}{*}\PY{n}{u}\PY{o}{.}\PY{n}{erg}\PY{o}{/}\PY{n}{u}\PY{o}{.}\PY{n}{s}\PY{o}{/}\PY{n}{u}\PY{o}{.}\PY{n}{cm}\PY{o}{*}\PY{o}{*}\PY{l+m+mi}{2}\PY{o}{/}\PY{n}{u}\PY{o}{.}\PY{n}{Angstrom}
        \PY{n}{T\PYZus{}star} \PY{o}{=} \PY{n}{T\PYZus{}ex}\PY{p}{(}\PY{n}{f\PYZus{}2}\PY{o}{/}\PY{n}{f\PYZus{}1}\PY{p}{)}\PY{o}{*}\PY{n}{u}\PY{o}{.}\PY{n}{K}
        \PY{n+nb}{print}\PY{p}{(}\PY{l+s+s1}{\PYZsq{}}\PY{l+s+s1}{The temperature of star is:}\PY{l+s+si}{\PYZob{}:.2f\PYZcb{}}\PY{l+s+s1}{\PYZsq{}}\PY{o}{.}\PY{n}{format}\PY{p}{(}\PY{n}{T\PYZus{}star}\PY{p}{)}\PY{p}{)}
\end{Verbatim}

    \begin{Verbatim}[commandchars=\\\{\}]
The temperature of star is:5019.73 K

    \end{Verbatim}

    \textbf{(b)}

    \begin{Verbatim}[commandchars=\\\{\}]
{\color{incolor}In [{\color{incolor}10}]:} \PY{n}{D} \PY{o}{=} \PY{l+m+mi}{1}\PY{o}{/}\PY{l+m+mf}{19e\PYZhy{}3}\PY{o}{*}\PY{n}{u}\PY{o}{.}\PY{n}{pc}
         \PY{n+nb}{print}\PY{p}{(}\PY{l+s+s1}{\PYZsq{}}\PY{l+s+s1}{The distance to the star is:}\PY{l+s+si}{\PYZob{}:.2f\PYZcb{}}\PY{l+s+s1}{\PYZsq{}}\PY{o}{.}\PY{n}{format}\PY{p}{(}\PY{n}{D}\PY{p}{)}\PY{p}{)}
         \PY{n+nb}{print}\PY{p}{(}\PY{l+s+s1}{\PYZsq{}}\PY{l+s+s1}{The distance to the star is:}\PY{l+s+si}{\PYZob{}:.2f\PYZcb{}}\PY{l+s+s1}{\PYZsq{}}\PY{o}{.}\PY{n}{format}\PY{p}{(}\PY{n}{D}\PY{o}{.}\PY{n}{to}\PY{p}{(}\PY{n}{u}\PY{o}{.}\PY{n}{km}\PY{p}{)}\PY{p}{)}\PY{p}{)}
\end{Verbatim}

    \begin{Verbatim}[commandchars=\\\{\}]
The distance to the star is:52.63 pc
The distance to the star is:1624040832351153.50 km

    \end{Verbatim}

    \textbf{(c)}

    Using inverse square law:
\[f_{earth} = \frac{f_\lambda R_{sun}^2}{D^2}\]

we get: \[R_{sun} = D \sqrt{\frac{f_{earth}}{f_\lambda}}\]

We perform this calculation for two filters.

    \begin{Verbatim}[commandchars=\\\{\}]
{\color{incolor}In [{\color{incolor}11}]:} \PY{c+c1}{\PYZsh{}(c)}
         \PY{n}{lambd} \PY{o}{=} \PY{l+m+mi}{532}\PY{o}{*}\PY{n}{u}\PY{o}{.}\PY{n}{nm}
         \PY{n}{F\PYZus{}earth} \PY{o}{=} \PY{l+m+mf}{5.9e\PYZhy{}13}\PY{o}{*}\PY{n}{u}\PY{o}{.}\PY{n}{erg}\PY{o}{/}\PY{n}{u}\PY{o}{.}\PY{n}{s}\PY{o}{/}\PY{n}{u}\PY{o}{.}\PY{n}{cm}\PY{o}{*}\PY{o}{*}\PY{l+m+mi}{2}\PY{o}{/}\PY{n}{u}\PY{o}{.}\PY{n}{Angstrom}
         \PY{n}{ratio} \PY{o}{=} \PY{n}{F\PYZus{}earth}\PY{o}{/}\PY{n}{F}\PY{p}{(}\PY{n}{lambd}\PY{p}{,} \PY{n}{T\PYZus{}star}\PY{p}{)}
         \PY{n}{R\PYZus{}star} \PY{o}{=} \PY{n}{D}\PY{o}{*}\PY{n}{np}\PY{o}{.}\PY{n}{sqrt}\PY{p}{(}\PY{n}{ratio}\PY{p}{)}
         \PY{n+nb}{print}\PY{p}{(}\PY{l+s+s1}{\PYZsq{}}\PY{l+s+s1}{The radius of the star is: }\PY{l+s+si}{\PYZob{}:.2f\PYZcb{}}\PY{l+s+s1}{\PYZsq{}}\PY{o}{.}\PY{n}{format}\PY{p}{(}\PY{n}{R\PYZus{}star}\PY{o}{.}\PY{n}{to}\PY{p}{(}\PY{n}{u}\PY{o}{.}\PY{n}{R\PYZus{}sun}\PY{p}{)}\PY{p}{)}\PY{p}{)}
\end{Verbatim}

    \begin{Verbatim}[commandchars=\\\{\}]
The radius of the star is: 1.58 solRad

    \end{Verbatim}

    \begin{Verbatim}[commandchars=\\\{\}]
{\color{incolor}In [{\color{incolor}12}]:} \PY{n}{lambd} \PY{o}{=} \PY{l+m+mi}{797}\PY{o}{*}\PY{n}{u}\PY{o}{.}\PY{n}{nm}
         \PY{n}{F\PYZus{}earth} \PY{o}{=} \PY{l+m+mf}{4.8e\PYZhy{}13}\PY{o}{*}\PY{n}{u}\PY{o}{.}\PY{n}{erg}\PY{o}{/}\PY{n}{u}\PY{o}{.}\PY{n}{s}\PY{o}{/}\PY{n}{u}\PY{o}{.}\PY{n}{cm}\PY{o}{*}\PY{o}{*}\PY{l+m+mi}{2}\PY{o}{/}\PY{n}{u}\PY{o}{.}\PY{n}{Angstrom}
         \PY{n}{ratio} \PY{o}{=} \PY{n}{F\PYZus{}earth}\PY{o}{/}\PY{n}{F}\PY{p}{(}\PY{n}{lambd}\PY{p}{,} \PY{n}{T\PYZus{}star}\PY{p}{)}
         \PY{n}{R\PYZus{}star} \PY{o}{=} \PY{n}{D}\PY{o}{*}\PY{n}{np}\PY{o}{.}\PY{n}{sqrt}\PY{p}{(}\PY{n}{ratio}\PY{p}{)}
         \PY{n+nb}{print}\PY{p}{(}\PY{l+s+s1}{\PYZsq{}}\PY{l+s+s1}{The radius of the star is: }\PY{l+s+si}{\PYZob{}:.2f\PYZcb{}}\PY{l+s+s1}{\PYZsq{}}\PY{o}{.}\PY{n}{format}\PY{p}{(}\PY{n}{R\PYZus{}star}\PY{o}{.}\PY{n}{to}\PY{p}{(}\PY{n}{u}\PY{o}{.}\PY{n}{R\PYZus{}sun}\PY{p}{)}\PY{p}{)}\PY{p}{)}
\end{Verbatim}

    \begin{Verbatim}[commandchars=\\\{\}]
The radius of the star is: 1.58 solRad

    \end{Verbatim}

    We can see that they give approximately the same answer.

    \textbf{(d)}

    We calculate Luminosity using Stefan-Boltzmann law:

    \begin{Verbatim}[commandchars=\\\{\}]
{\color{incolor}In [{\color{incolor}13}]:} \PY{n}{L} \PY{o}{=} \PY{n}{c}\PY{o}{.}\PY{n}{sigma\PYZus{}sb}\PY{o}{*}\PY{n}{T\PYZus{}star}\PY{o}{*}\PY{o}{*}\PY{l+m+mi}{4}\PY{o}{*}\PY{l+m+mi}{4}\PY{o}{*}\PY{n}{np}\PY{o}{.}\PY{n}{pi}\PY{o}{*}\PY{n}{R\PYZus{}star}\PY{o}{*}\PY{o}{*}\PY{l+m+mi}{2}
         \PY{n+nb}{print}\PY{p}{(}\PY{l+s+s1}{\PYZsq{}}\PY{l+s+s1}{The Luminosity of the star is: }\PY{l+s+si}{\PYZob{}:.2f\PYZcb{}}\PY{l+s+s1}{\PYZsq{}}\PY{o}{.}\PY{n}{format}\PY{p}{(}\PY{n}{L}\PY{o}{.}\PY{n}{to}\PY{p}{(}\PY{n}{u}\PY{o}{.}\PY{n}{L\PYZus{}sun}\PY{p}{)}\PY{p}{)}\PY{p}{)}
\end{Verbatim}

    \begin{Verbatim}[commandchars=\\\{\}]
The Luminosity of the star is: 1.43 solLum

    \end{Verbatim}

    From the temperature and luminosity of the star, we see that it lies in
the main sequence in the HR-diagram. They are main sequence star.


    % Add a bibliography block to the postdoc
    
    
    
    \end{document}
